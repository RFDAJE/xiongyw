% $Id: juan07.tex,v 1.4 2011/08/09 09:08:36 xiongyw Exp $


阿難,汝問攝心,我今先說入三摩地修學妙門;求菩薩道,要先持此四種律儀,皎如冰霜。自不能生一切枝葉;心三口四,生必無因。阿難,如是四事,若不遺失;心尚不緣色、香、味、觸,一切魔事,云何發生?若有宿習,不能滅除。汝教是人,一心誦我,佛頂光明摩訶薩怛哆般怛囉無上神呪。斯是如來無見頂相無為心佛,從頂發輝,坐寶蓮華,所說神呪。且汝宿世與摩登伽歷劫因緣,恩愛習氣,非是一生及與一劫。我一宣揚,愛心永脫,成阿羅漢。彼尚婬女,無心修行;神力冥資,速證無學。云何汝等在會聲聞,求最上乘,決定成佛?譬如以塵揚於順風,有何艱險?若有末世,欲坐道場,先持比丘清淨禁戒。要當選擇戒清淨者,第一沙門以為其師。若其不遇眞清淨僧,汝戒律儀必不成就。戒成已後,著新淨衣,然香閑居,誦此心佛所說神呪一百八徧,然後結界,建立道場。求於十方現住國土無上如來,放大悲光來灌其頂。阿難,如是末世清淨比丘,若比丘尼、白衣檀越,心滅貪婬,持佛淨戒,於道場中,發菩薩願。出入澡浴,六時行道,如是不寐,經三七日。我自現身至其人前,摩頂安慰,令其開悟。

阿難白佛言:世尊,我蒙如來無上悲誨,心已開悟,自知修證,無學道成。末法修行,建立道場,云何結界,合佛世尊清淨軌則?

佛告阿難:若末世人,願立道場。先取雪山大力白牛,食其山中肥膩香草;此牛唯飲雪山清水,其糞微細。可取其糞,和合旃檀以泥其地。若非雪山,其牛臭穢,不堪塗地。別於平原,穿去地皮五尺已下,取其黃土。和上旃檀、沈水、蘇合、薰陸、鬱金、白膠、青木、零陵、甘松,及雞舌香;以此十種細羅為粉,合土成泥,以塗場地。方圓丈六,為八角壇。壇心置一金、銀、銅、木,所造蓮華。華中安缽,缽中先盛八月露水,水中隨安所有華葉。取八圓鏡,各安其方,圍繞華缽。鏡外建立十六蓮華、十六香鑪;間華鋪設,莊嚴香鑪。純燒沈水,無令見火。取白牛乳置十六器,乳為煎餅,並諸砂糖、油餅、乳糜、蘇合、蜜薑、純酥、純蜜;於蓮華外各各十六,繞圍華外,以奉諸佛及大菩薩。每以食時,若在中夜,取蜜半升,用酥三合,壇前別安一小火鑪,以兜樓婆香,煎取香水,沐浴其炭,然令猛熾;投是酥蜜於炎鑪內,燒令煙盡,享佛菩薩。令其四外,徧懸旛華。於壇室中,四壁敷設十方如來,及諸菩薩所有形像。應於當陽張盧舍那、釋迦、彌勒、阿閦、彌陀;諸大變化觀音形像,兼金剛藏,安其左右。帝釋、梵王、烏芻瑟摩,並藍地迦、諸軍荼利;與毗俱胝、四天王等,頻那、夜迦,張於門側,左右安置。又取八鏡覆懸虛空,與壇場中所安之鏡,方面相對,使其形影重重相涉。

於初七中,至誠頂禮十方如來、諸大菩薩、阿羅漢號,恆於六時誦呪圍壇,至心行道,一時常行一百八徧。第二七中,一向專心發菩薩願,心無間斷,我毗奈耶先有願教。第三七中,於十二時,一向持佛,般怛囉呪。至第七日,十方如來,一時出現,鏡交光處,承佛摩頂。卽於道場,修三摩地。能令如是,末世修學,身心明淨,猶如璢璃。阿難,若此比丘本受戒師,及同會中十比丘等,其中有一不清淨者,如是道場,多不成就。從三七後,端坐安居經一百日;有利根者,不起於座,得須陀洹。縱其身心聖果未成,決定自知成佛不謬。汝問道場,建立如是。

阿難頂禮佛足,而白佛言:自我出家,恃佛憍愛,求多聞故,未證無為。遭彼梵天邪術所禁,心雖明了,力不自由。賴遇文殊,令我解脫。雖蒙如來佛頂神呪,冥獲其力,尚未親聞。惟願大慈重為宣說,悲救此會諸修行輩,末及當來在輪迴者,承佛密音,身意解脫。

於時,會中一切大眾普皆作禮,佇聞如來祕密章句。

爾時,世尊從肉髻中,涌百寶光,光中涌出千葉寶蓮,有化如來坐寶華中,頂放十道百寶光明,一一光明,皆徧示現十恆河沙金剛密跡,擎山持杵,徧虛空界。大眾仰觀,畏愛兼抱,求佛哀祐,一心聽佛無見頂相,放光如來宣說神呪:

\begin{ZHOUYU}
% $Id: mantra.tex,v 1.3 2011/08/09 07:14:15 xiongyw Exp $


%------------------------------
% 楞严咒:序号,音译,大意
%------------------------------

\ifthenelse{\boolean{fulu}}{\begin{center}\heiti\zihao{1} 楞嚴呪大意 \end{center}}{}

%------------------------------
% 第一会
%------------------------------


\ifthenelse{\boolean{fulu}}{\begin{center}\fangsong\zihao{2} 第一會 \end{center}}{}

\ifthenelse{\boolean{fulu}}{\zydy}{\zy}{1}{南無薩怛他蘇伽多耶阿囉訶帝三藐三菩陀寫}{敬禮世尊善逝應供正等覺}
\ifthenelse{\boolean{fulu}}{\zydy}{\zy}{2}{薩怛他佛陀俱胝瑟尼釤}{敬禮如來百千億佛頂}
\ifthenelse{\boolean{fulu}}{\zydy}{\zy}{3}{南無薩婆勃陀勃地薩跢鞞弊}{敬禮一切佛陀菩薩眾}
\ifthenelse{\boolean{fulu}}{\zydy}{\zy}{4}{南無薩多南三藐三菩陀俱知南}{敬禮七俱胝正等覺}
\ifthenelse{\boolean{fulu}}{\zydy}{\zy}{5}{娑舍囉婆迦僧伽喃}{及聲聞僧伽眾}
\ifthenelse{\boolean{fulu}}{\zydy}{\zy}{6}{南無盧雞阿羅漢跢喃}{敬禮世間阿羅漢眾}
\ifthenelse{\boolean{fulu}}{\zydy}{\zy}{7}{南無蘇盧多波那喃}{敬禮預流眾}
\ifthenelse{\boolean{fulu}}{\zydy}{\zy}{8}{南無娑羯唎陀伽彌喃}{敬禮一來眾}
\ifthenelse{\boolean{fulu}}{\zydy}{\zy}{9}{南無盧雞三藐伽跢喃}{敬禮世間正至眾}
\ifthenelse{\boolean{fulu}}{\zydy}{\zy}{10}{三藐伽波囉底波多那喃}{正行眾}
\ifthenelse{\boolean{fulu}}{\zydy}{\zy}{11}{南無提婆離瑟𧹞}{敬禮天仙眾}
\ifthenelse{\boolean{fulu}}{\zydy}{\zy}{12}{南無悉陀耶毗地耶陀囉離瑟𧹞}{敬禮持明成就仙眾}
\ifthenelse{\boolean{fulu}}{\zydy}{\zy}{13}{舍波奴揭囉訶娑訶娑囉摩他喃}{堪能降伏惡呪眾}
\ifthenelse{\boolean{fulu}}{\zydy}{\zy}{14}{南無跋囉訶摩泥}{敬禮梵天}
\ifthenelse{\boolean{fulu}}{\zydy}{\zy}{15}{南無因陀囉耶}{敬禮帝釋天}
\ifthenelse{\boolean{fulu}}{\zydy}{\zy}{16}{南無婆伽婆帝}{敬禮世尊}
\ifthenelse{\boolean{fulu}}{\zydy}{\zy}{17}{盧陀囉耶}{大自在天}
\ifthenelse{\boolean{fulu}}{\zydy}{\zy}{18}{烏摩般帝}{烏摩天後}
\ifthenelse{\boolean{fulu}}{\zydy}{\zy}{19}{娑醯夜耶}{及眷屬眾}
\ifthenelse{\boolean{fulu}}{\zydy}{\zy}{20}{南無婆伽婆帝}{敬禮世尊}
\ifthenelse{\boolean{fulu}}{\zydy}{\zy}{21}{那囉野拏耶}{那羅延天}
\ifthenelse{\boolean{fulu}}{\zydy}{\zy}{22}{槃遮摩訶三慕陀囉}{五大印女}
\ifthenelse{\boolean{fulu}}{\zydy}{\zy}{23}{南無悉羯唎多耶}{所敬禮處}
\ifthenelse{\boolean{fulu}}{\zydy}{\zy}{24}{南無婆伽婆帝}{敬禮世尊}
\ifthenelse{\boolean{fulu}}{\zydy}{\zy}{25}{摩訶迦羅耶}{大黑天}
\ifthenelse{\boolean{fulu}}{\zydy}{\zy}{26}{地唎般剌那伽囉}{三重(金、銀、鐵)城}
\ifthenelse{\boolean{fulu}}{\zydy}{\zy}{27}{毗陀囉波拏迦囉耶}{摧壞能作}
\ifthenelse{\boolean{fulu}}{\zydy}{\zy}{28}{阿地目帝}{樂於}
\ifthenelse{\boolean{fulu}}{\zydy}{\zy}{29}{尸摩舍那泥婆悉泥}{墓塚居住}
\ifthenelse{\boolean{fulu}}{\zydy}{\zy}{30}{摩怛唎伽拏}{鬼母眾}
\ifthenelse{\boolean{fulu}}{\zydy}{\zy}{31}{南無悉羯唎多耶}{所敬禮處}
\ifthenelse{\boolean{fulu}}{\zydy}{\zy}{32}{南無婆伽婆帝}{敬禮世尊}
\ifthenelse{\boolean{fulu}}{\zydy}{\zy}{33}{多他伽跢俱囉耶}{如來部}
\ifthenelse{\boolean{fulu}}{\zydy}{\zy}{34}{南無般頭摩俱囉耶}{敬禮蓮華部}
\ifthenelse{\boolean{fulu}}{\zydy}{\zy}{35}{南無跋闍羅俱囉耶}{敬禮金剛部}
\ifthenelse{\boolean{fulu}}{\zydy}{\zy}{36}{南無摩尼俱囉耶}{敬禮寶部}
\ifthenelse{\boolean{fulu}}{\zydy}{\zy}{37}{南無伽闍俱囉耶}{敬禮象部}
\ifthenelse{\boolean{fulu}}{\zydy}{\zy}{38}{南無婆伽婆帝}{敬禮世尊}
\ifthenelse{\boolean{fulu}}{\zydy}{\zy}{39}{帝唎茶輸囉西那}{勇猛軍隊}
\ifthenelse{\boolean{fulu}}{\zydy}{\zy}{40}{波囉訶囉拏囉闍耶}{鬥戰王}
\ifthenelse{\boolean{fulu}}{\zydy}{\zy}{41}{跢他伽多耶}{如來}
\ifthenelse{\boolean{fulu}}{\zydy}{\zy}{42}{南無婆伽婆帝}{敬禮世尊}
\ifthenelse{\boolean{fulu}}{\zydy}{\zy}{43}{南無阿彌多婆耶}{敬禮阿彌陀佛}
\ifthenelse{\boolean{fulu}}{\zydy}{\zy}{44}{跢他伽多耶}{如來}
\ifthenelse{\boolean{fulu}}{\zydy}{\zy}{45}{阿囉訶帝}{應供}
\ifthenelse{\boolean{fulu}}{\zydy}{\zy}{46}{三藐三菩陀耶}{正等覺}
\ifthenelse{\boolean{fulu}}{\zydy}{\zy}{47}{南無婆伽婆帝}{敬禮世尊}
\ifthenelse{\boolean{fulu}}{\zydy}{\zy}{48}{阿芻鞞耶}{不動(阿閦)}
\ifthenelse{\boolean{fulu}}{\zydy}{\zy}{49}{跢他伽多耶}{如來}
\ifthenelse{\boolean{fulu}}{\zydy}{\zy}{50}{阿囉訶帝}{應供}
\ifthenelse{\boolean{fulu}}{\zydy}{\zy}{51}{三藐三菩陀耶}{正等覺}
\ifthenelse{\boolean{fulu}}{\zydy}{\zy}{52}{南無婆伽婆帝}{敬禮世尊}
\ifthenelse{\boolean{fulu}}{\zydy}{\zy}{53}{鞞沙闍耶俱盧吠柱唎耶}{藥師璢璃}
\ifthenelse{\boolean{fulu}}{\zydy}{\zy}{54}{般囉婆囉闍耶}{光王}
\ifthenelse{\boolean{fulu}}{\zydy}{\zy}{55}{跢他伽多耶}{如來}
\ifthenelse{\boolean{fulu}}{\zydy}{\zy}{56}{南無婆伽婆帝}{敬禮世尊}
\ifthenelse{\boolean{fulu}}{\zydy}{\zy}{57}{三補師毖多}{開敷蓮華}
\ifthenelse{\boolean{fulu}}{\zydy}{\zy}{58}{薩憐捺囉剌闍耶}{娑羅樹王}
\ifthenelse{\boolean{fulu}}{\zydy}{\zy}{59}{跢他伽多耶}{如來}
\ifthenelse{\boolean{fulu}}{\zydy}{\zy}{60}{阿囉訶帝}{應供}
\ifthenelse{\boolean{fulu}}{\zydy}{\zy}{61}{三藐三菩陀耶}{正等覺}
\ifthenelse{\boolean{fulu}}{\zydy}{\zy}{62}{南無婆伽婆帝}{敬禮世尊}
\ifthenelse{\boolean{fulu}}{\zydy}{\zy}{63}{舍雞野母那曳}{釋迦牟尼}
\ifthenelse{\boolean{fulu}}{\zydy}{\zy}{64}{跢他伽多耶}{如來}
\ifthenelse{\boolean{fulu}}{\zydy}{\zy}{65}{阿囉訶帝}{應供}
\ifthenelse{\boolean{fulu}}{\zydy}{\zy}{66}{三藐三菩陀耶}{正等覺}
\ifthenelse{\boolean{fulu}}{\zydy}{\zy}{67}{南無婆伽婆帝}{敬禮世尊}
\ifthenelse{\boolean{fulu}}{\zydy}{\zy}{68}{剌怛那雞都囉闍耶}{寶幢王}
\ifthenelse{\boolean{fulu}}{\zydy}{\zy}{69}{跢他伽多耶}{如來}
\ifthenelse{\boolean{fulu}}{\zydy}{\zy}{70}{阿囉訶帝}{應供}
\ifthenelse{\boolean{fulu}}{\zydy}{\zy}{71}{三藐三菩陀耶}{正等覺}
\ifthenelse{\boolean{fulu}}{\zydy}{\zy}{72}{帝瓢南無薩羯唎多}{如是敬禮已}
\ifthenelse{\boolean{fulu}}{\zydy}{\zy}{73}{翳曇婆伽婆多}{此諸世尊}
\ifthenelse{\boolean{fulu}}{\zydy}{\zy}{74}{薩怛他伽都瑟尼釤}{如來頂髻}
\ifthenelse{\boolean{fulu}}{\zydy}{\zy}{75}{薩怛多般怛㘕}{白傘蓋}
\ifthenelse{\boolean{fulu}}{\zydy}{\zy}{76}{南無阿婆囉視耽}{名爲無有能及}
\ifthenelse{\boolean{fulu}}{\zydy}{\zy}{77}{般囉帝揚岐囉}{甚能調伏}
\ifthenelse{\boolean{fulu}}{\zydy}{\zy}{78}{薩囉婆部多揭囉訶}{一切鬼魅惡星}
\ifthenelse{\boolean{fulu}}{\zydy}{\zy}{79}{尼羯囉訶揭迦囉訶尼}{降伏能令作}
\ifthenelse{\boolean{fulu}}{\zydy}{\zy}{80}{跋囉毖地耶叱陀你}{能令斷滅仇敵惡呪}
\ifthenelse{\boolean{fulu}}{\zydy}{\zy}{81}{阿迦囉密唎柱}{夭折橫死}
\ifthenelse{\boolean{fulu}}{\zydy}{\zy}{82}{般唎怛囉耶儜揭唎}{救護能作}
\ifthenelse{\boolean{fulu}}{\zydy}{\zy}{83}{薩囉婆槃陀那目叉尼}{能令解脫一切煩惱縛結}
\ifthenelse{\boolean{fulu}}{\zydy}{\zy}{84}{薩囉婆突瑟吒}{一切極惡}
\ifthenelse{\boolean{fulu}}{\zydy}{\zy}{85}{突悉乏般那你伐囉尼}{夢魇 能令止息}
\ifthenelse{\boolean{fulu}}{\zydy}{\zy}{86}{赭都囉失帝南}{八十四}
\ifthenelse{\boolean{fulu}}{\zydy}{\zy}{87}{羯囉訶娑訶薩囉若闍}{惡星邪魔千}
\ifthenelse{\boolean{fulu}}{\zydy}{\zy}{88}{毗多崩娑那羯唎}{敗壞能令}
\ifthenelse{\boolean{fulu}}{\zydy}{\zy}{89}{阿瑟吒氷舍帝南}{二十八}
\ifthenelse{\boolean{fulu}}{\zydy}{\zy}{90}{那叉剎怛囉若闍}{星宿}
\ifthenelse{\boolean{fulu}}{\zydy}{\zy}{91}{波囉薩陀那羯唎}{清淨能令}
\ifthenelse{\boolean{fulu}}{\zydy}{\zy}{92}{阿瑟吒南}{八}
\ifthenelse{\boolean{fulu}}{\zydy}{\zy}{93}{摩訶揭囉訶若闍}{大執曜}
\ifthenelse{\boolean{fulu}}{\zydy}{\zy}{94}{毗多崩薩那羯唎}{摧伏能令}
\ifthenelse{\boolean{fulu}}{\zydy}{\zy}{95}{薩婆舍都嚧你婆囉若闍}{能令遮止一切怨敵}
\ifthenelse{\boolean{fulu}}{\zydy}{\zy}{96}{呼藍突悉乏難遮那舍尼}{能令消滅魔呪法及諸惡夢}
\ifthenelse{\boolean{fulu}}{\zydy}{\zy}{97}{毖沙舍悉怛囉}{毒藥刀兵}
\ifthenelse{\boolean{fulu}}{\zydy}{\zy}{98}{阿吉尼烏陀迦囉若闍}{火災水災能令救護}
\ifthenelse{\boolean{fulu}}{\zydy}{\zy}{99}{阿般囉視多具囉}{威力猛烈無能及}
\ifthenelse{\boolean{fulu}}{\zydy}{\zy}{100}{摩訶般囉戰持}{大力天女可畏女}
\ifthenelse{\boolean{fulu}}{\zydy}{\zy}{101}{摩訶疊多}{大熾燃}
\ifthenelse{\boolean{fulu}}{\zydy}{\zy}{102}{摩訶帝闍}{大威力}
\ifthenelse{\boolean{fulu}}{\zydy}{\zy}{103}{摩訶稅多闍婆囉}{大白輝女及光焰}
\ifthenelse{\boolean{fulu}}{\zydy}{\zy}{104}{摩訶跋囉槃陀囉婆悉你}{大力天女白衣女}
\ifthenelse{\boolean{fulu}}{\zydy}{\zy}{105}{阿唎耶多囉}{多羅天女}
\ifthenelse{\boolean{fulu}}{\zydy}{\zy}{106}{毗唎俱知}{毗俱胝天女}
\ifthenelse{\boolean{fulu}}{\zydy}{\zy}{107}{誓婆毗闍耶}{及如是最勝}
\ifthenelse{\boolean{fulu}}{\zydy}{\zy}{108}{跋闍囉摩禮底}{金剛鬘}
\ifthenelse{\boolean{fulu}}{\zydy}{\zy}{109}{毗舍嚧多}{名稱}
\ifthenelse{\boolean{fulu}}{\zydy}{\zy}{110}{勃騰罔迦}{蓮花}
\ifthenelse{\boolean{fulu}}{\zydy}{\zy}{111}{跋闍囉制喝那阿遮}{金剛舌及}
\ifthenelse{\boolean{fulu}}{\zydy}{\zy}{112}{摩囉制婆般囉質多}{花鬘及如是無能勝}
\ifthenelse{\boolean{fulu}}{\zydy}{\zy}{113}{跋闍囉擅持}{金剛杵}
\ifthenelse{\boolean{fulu}}{\zydy}{\zy}{114}{毗舍囉遮}{廣大及}
\ifthenelse{\boolean{fulu}}{\zydy}{\zy}{115}{扇多舍鞞提婆補視多}{寂靜勝身諸供養}
\ifthenelse{\boolean{fulu}}{\zydy}{\zy}{116}{蘇摩嚧波}{月光相}
\ifthenelse{\boolean{fulu}}{\zydy}{\zy}{117}{摩訶稅多}{大白}
\ifthenelse{\boolean{fulu}}{\zydy}{\zy}{118}{阿唎耶多囉}{聖救度}
\ifthenelse{\boolean{fulu}}{\zydy}{\zy}{119}{摩訶婆囉阿般囉}{大力不歿}
\ifthenelse{\boolean{fulu}}{\zydy}{\zy}{120}{跋闍囉商羯囉制婆}{金剛鎖及如是}
\ifthenelse{\boolean{fulu}}{\zydy}{\zy}{121}{跋闍囉俱摩唎}{金剛童女}
\ifthenelse{\boolean{fulu}}{\zydy}{\zy}{122}{俱藍陀唎}{部持女}
\ifthenelse{\boolean{fulu}}{\zydy}{\zy}{123}{跋闍囉喝薩多遮}{金剛手及}
\ifthenelse{\boolean{fulu}}{\zydy}{\zy}{124}{毗地耶乾遮那摩唎迦}{明女金花鬘}
\ifthenelse{\boolean{fulu}}{\zydy}{\zy}{125}{啒蘇母婆羯囉跢那}{紅寶珠}
\ifthenelse{\boolean{fulu}}{\zydy}{\zy}{126}{鞞嚧遮那俱唎耶}{遍照種族}
\ifthenelse{\boolean{fulu}}{\zydy}{\zy}{127}{夜囉菟瑟尼釤}{諸利益頂髻}
\ifthenelse{\boolean{fulu}}{\zydy}{\zy}{128}{毗折藍婆摩尼遮}{細眉開展及}
\ifthenelse{\boolean{fulu}}{\zydy}{\zy}{129}{跋闍囉迦那迦波囉婆}{金剛金光}
\ifthenelse{\boolean{fulu}}{\zydy}{\zy}{130}{嚧闍那跋闍囉頓稚遮}{眼金剛嘴及}
\ifthenelse{\boolean{fulu}}{\zydy}{\zy}{131}{稅多遮迦摩囉}{白色及蓮花眼}
\ifthenelse{\boolean{fulu}}{\zydy}{\zy}{132}{剎奢尸波囉婆}{月光}
\ifthenelse{\boolean{fulu}}{\zydy}{\zy}{133}{翳帝夷帝}{如是}
\ifthenelse{\boolean{fulu}}{\zydy}{\zy}{134}{母陀囉羯拏}{印眾}
\ifthenelse{\boolean{fulu}}{\zydy}{\zy}{135}{沙鞞囉懺}{一切守護}
\ifthenelse{\boolean{fulu}}{\zydy}{\zy}{136}{掘梵都}{願作}
\ifthenelse{\boolean{fulu}}{\zydy}{\zy}{137}{印兔那麼麼寫}{此我等}

%------------------------------
% 第二会
%------------------------------
\ifthenelse{\boolean{fulu}}{\begin{center}\fangsong\zihao{2} 第二會 \end{center}}{}

\ifthenelse{\boolean{fulu}}{\zydy}{\zy}{138}{烏𤙖}{三身}
\ifthenelse{\boolean{fulu}}{\zydy}{\zy}{139}{唎瑟揭拏}{仙眾}
\ifthenelse{\boolean{fulu}}{\zydy}{\zy}{140}{般剌舍悉多}{讚歎}
\ifthenelse{\boolean{fulu}}{\zydy}{\zy}{141}{薩怛他伽都瑟尼釤}{如來頂髻}
\ifthenelse{\boolean{fulu}}{\zydy}{\zy}{142}{虎𤙖}{}
\ifthenelse{\boolean{fulu}}{\zydy}{\zy}{143}{都嚧雍}{}
\ifthenelse{\boolean{fulu}}{\zydy}{\zy}{144}{瞻婆那}{破碎}
\ifthenelse{\boolean{fulu}}{\zydy}{\zy}{145}{虎𤙖}{}
\ifthenelse{\boolean{fulu}}{\zydy}{\zy}{146}{都嚧雍}{}
\ifthenelse{\boolean{fulu}}{\zydy}{\zy}{147}{悉耽婆那}{降伏}
\ifthenelse{\boolean{fulu}}{\zydy}{\zy}{148}{虎𤙖}{}
\ifthenelse{\boolean{fulu}}{\zydy}{\zy}{149}{都嚧雍}{}
\ifthenelse{\boolean{fulu}}{\zydy}{\zy}{150}{波羅瑟地耶三般叉拏羯囉}{齊令破壞(外道)最勝呪術}
\ifthenelse{\boolean{fulu}}{\zydy}{\zy}{151}{虎𤙖}{}
\ifthenelse{\boolean{fulu}}{\zydy}{\zy}{152}{都嚧雍}{}
\ifthenelse{\boolean{fulu}}{\zydy}{\zy}{153}{薩婆藥叉喝囉剎娑}{一切夜叉羅剎}
\ifthenelse{\boolean{fulu}}{\zydy}{\zy}{154}{揭囉訶若闍}{惡星}
\ifthenelse{\boolean{fulu}}{\zydy}{\zy}{155}{毗騰崩薩那羯囉}{敗壞令作}
\ifthenelse{\boolean{fulu}}{\zydy}{\zy}{156}{虎𤙖}{}
\ifthenelse{\boolean{fulu}}{\zydy}{\zy}{157}{都嚧雍}{}
\ifthenelse{\boolean{fulu}}{\zydy}{\zy}{158}{者都囉尸底南}{八十四}
\ifthenelse{\boolean{fulu}}{\zydy}{\zy}{159}{揭囉訶娑訶薩囉南}{惡星鬼魅千}
\ifthenelse{\boolean{fulu}}{\zydy}{\zy}{160}{毗騰崩薩那囉}{降伏令作}
\ifthenelse{\boolean{fulu}}{\zydy}{\zy}{161}{虎𤙖}{}
\ifthenelse{\boolean{fulu}}{\zydy}{\zy}{162}{都嚧雍}{}
\ifthenelse{\boolean{fulu}}{\zydy}{\zy}{163}{囉叉}{守護}
\ifthenelse{\boolean{fulu}}{\zydy}{\zy}{164}{婆伽梵}{世尊}
\ifthenelse{\boolean{fulu}}{\zydy}{\zy}{165}{薩怛他伽都瑟尼釤}{如來頂髻}
\ifthenelse{\boolean{fulu}}{\zydy}{\zy}{166}{波囉點闍吉唎}{甚能調伏}
\ifthenelse{\boolean{fulu}}{\zydy}{\zy}{167}{摩訶娑訶薩囉}{大千}
\ifthenelse{\boolean{fulu}}{\zydy}{\zy}{168}{勃樹娑訶薩囉室唎沙}{臂千頭}
\ifthenelse{\boolean{fulu}}{\zydy}{\zy}{169}{俱知娑訶薩泥帝㘑}{俱胝百千諸眼}
\ifthenelse{\boolean{fulu}}{\zydy}{\zy}{170}{阿弊提視婆唎多}{不毀光輝}
\ifthenelse{\boolean{fulu}}{\zydy}{\zy}{171}{吒吒甖迦}{無邊岸}
\ifthenelse{\boolean{fulu}}{\zydy}{\zy}{172}{摩訶跋闍嚧陀囉}{大金剛殊妙}
\ifthenelse{\boolean{fulu}}{\zydy}{\zy}{173}{帝唎菩婆那}{三界}
\ifthenelse{\boolean{fulu}}{\zydy}{\zy}{174}{曼茶囉}{壇場}
\ifthenelse{\boolean{fulu}}{\zydy}{\zy}{175}{烏𤙖}{三身}
\ifthenelse{\boolean{fulu}}{\zydy}{\zy}{176}{莎悉帝薄婆都}{福佑請作}
\ifthenelse{\boolean{fulu}}{\zydy}{\zy}{177}{麼麼}{我}
\ifthenelse{\boolean{fulu}}{\zydy}{\zy}{178}{印兔那麼麼寫}{此等我的}

%------------------------------
% 第三会
%------------------------------
\ifthenelse{\boolean{fulu}}{\begin{center}\fangsong\zihao{2} 第三會 \end{center}}{}


\ifthenelse{\boolean{fulu}}{\zydy}{\zy}{179}{囉闍婆夜}{諸王難}
\ifthenelse{\boolean{fulu}}{\zydy}{\zy}{180}{主囉跋夜}{諸賊難}
\ifthenelse{\boolean{fulu}}{\zydy}{\zy}{181}{阿祇尼婆夜}{諸火難}
\ifthenelse{\boolean{fulu}}{\zydy}{\zy}{182}{烏陀迦婆夜}{諸水難}
\ifthenelse{\boolean{fulu}}{\zydy}{\zy}{183}{毗沙婆夜}{諸毒難}
\ifthenelse{\boolean{fulu}}{\zydy}{\zy}{184}{舍薩多囉婆夜}{諸刀兵難}
\ifthenelse{\boolean{fulu}}{\zydy}{\zy}{185}{婆囉斫羯囉婆夜}{諸怨敵難}
\ifthenelse{\boolean{fulu}}{\zydy}{\zy}{186}{突瑟叉婆夜}{諸飢饉難}
\ifthenelse{\boolean{fulu}}{\zydy}{\zy}{187}{阿舍你婆夜}{諸雷電難}
\ifthenelse{\boolean{fulu}}{\zydy}{\zy}{188}{阿迦囉密唎柱婆夜}{諸橫死難}
\ifthenelse{\boolean{fulu}}{\zydy}{\zy}{189}{陀囉尼部彌劍波伽波陀婆夜}{諸大地地震難}
\ifthenelse{\boolean{fulu}}{\zydy}{\zy}{190}{烏囉迦婆多婆夜}{諸流星崩落難}
\ifthenelse{\boolean{fulu}}{\zydy}{\zy}{191}{剌闍壇茶婆夜}{諸王刀仗難}
\ifthenelse{\boolean{fulu}}{\zydy}{\zy}{192}{那伽婆夜}{諸龍難}
\ifthenelse{\boolean{fulu}}{\zydy}{\zy}{193}{毗條怛婆夜}{諸電光難}
\ifthenelse{\boolean{fulu}}{\zydy}{\zy}{194}{蘇波囉拏婆夜}{諸大猛禽難}
\ifthenelse{\boolean{fulu}}{\zydy}{\zy}{195}{藥叉揭囉訶}{諸夜叉所持}
\ifthenelse{\boolean{fulu}}{\zydy}{\zy}{196}{囉叉私揭囉訶}{諸羅剎所持}
\ifthenelse{\boolean{fulu}}{\zydy}{\zy}{197}{畢唎多揭囉訶}{諸餓鬼所持}
\ifthenelse{\boolean{fulu}}{\zydy}{\zy}{198}{毗舍遮揭囉訶}{諸毗舍遮所持}
\ifthenelse{\boolean{fulu}}{\zydy}{\zy}{199}{部多揭囉訶}{諸部多所魅}
\ifthenelse{\boolean{fulu}}{\zydy}{\zy}{200}{鳩槃茶揭囉訶}{諸鳩槃茶所魅}
\ifthenelse{\boolean{fulu}}{\zydy}{\zy}{201}{補丹那揭囉訶}{諸富單那所魅}
\ifthenelse{\boolean{fulu}}{\zydy}{\zy}{202}{迦吒補丹那揭囉訶}{諸迦吒富單那所魅}
\ifthenelse{\boolean{fulu}}{\zydy}{\zy}{203}{悉乾度揭囉訶}{諸騫陀所魅}
\ifthenelse{\boolean{fulu}}{\zydy}{\zy}{204}{阿播悉摩囉揭囉訶}{諸阿波悉魔羅所魅}
\ifthenelse{\boolean{fulu}}{\zydy}{\zy}{205}{烏檀摩陀揭囉訶}{諸醉鬼所魅}
\ifthenelse{\boolean{fulu}}{\zydy}{\zy}{206}{車夜揭囉訶}{諸陰鬼所魅}
\ifthenelse{\boolean{fulu}}{\zydy}{\zy}{207}{醯唎婆帝揭囉訶}{諸黎婆坻所魅}
\ifthenelse{\boolean{fulu}}{\zydy}{\zy}{208}{社多訶唎南}{諸食生鬼}
\ifthenelse{\boolean{fulu}}{\zydy}{\zy}{209}{揭婆訶唎南}{諸食胎鬼}
\ifthenelse{\boolean{fulu}}{\zydy}{\zy}{210}{嚧地囉訶唎南}{諸食血鬼}
\ifthenelse{\boolean{fulu}}{\zydy}{\zy}{211}{忙娑訶唎南}{諸食肉鬼}
\ifthenelse{\boolean{fulu}}{\zydy}{\zy}{212}{謎陀訶唎南}{諸食脂鬼}
\ifthenelse{\boolean{fulu}}{\zydy}{\zy}{213}{摩闍訶唎南}{諸食髓鬼}
\ifthenelse{\boolean{fulu}}{\zydy}{\zy}{214}{闍多訶唎女}{諸食生鬼}
\ifthenelse{\boolean{fulu}}{\zydy}{\zy}{215}{視比多訶唎南}{諸食命鬼}
\ifthenelse{\boolean{fulu}}{\zydy}{\zy}{216}{毗多訶唎南}{諸食飲鬼}
\ifthenelse{\boolean{fulu}}{\zydy}{\zy}{217}{婆多訶唎南}{諸食吐鬼}
\ifthenelse{\boolean{fulu}}{\zydy}{\zy}{218}{阿輸遮訶唎女}{諸食不淨物鬼}
\ifthenelse{\boolean{fulu}}{\zydy}{\zy}{219}{質多訶唎女}{諸食心鬼}
\ifthenelse{\boolean{fulu}}{\zydy}{\zy}{220}{帝釤薩鞞釤}{如是一切等}
\ifthenelse{\boolean{fulu}}{\zydy}{\zy}{221}{薩婆揭囉訶南}{一切諸鬼魅眾}
\ifthenelse{\boolean{fulu}}{\zydy}{\zy}{222}{毗陀耶闍瞋陀夜彌}{呪術我今悉使斷除}
\ifthenelse{\boolean{fulu}}{\zydy}{\zy}{223}{雞囉夜彌}{悉使釘住}
\ifthenelse{\boolean{fulu}}{\zydy}{\zy}{224}{波唎跋囉者迦訖唎擔}{波立婆外道所造}
\ifthenelse{\boolean{fulu}}{\zydy}{\zy}{225}{毗陀夜闍瞋陀夜彌}{呪術我今悉使斷除}
\ifthenelse{\boolean{fulu}}{\zydy}{\zy}{226}{雞囉夜彌}{悉使釘住}
\ifthenelse{\boolean{fulu}}{\zydy}{\zy}{227}{茶演尼訖唎擔}{食人肉女鬼所造}
\ifthenelse{\boolean{fulu}}{\zydy}{\zy}{228}{毗陀夜闍瞋陀夜彌}{呪術我今悉使斷除}
\ifthenelse{\boolean{fulu}}{\zydy}{\zy}{229}{雞囉夜彌}{悉使釘住}
\ifthenelse{\boolean{fulu}}{\zydy}{\zy}{230}{摩訶般輸般怛夜}{大獸主}
\ifthenelse{\boolean{fulu}}{\zydy}{\zy}{231}{嚧陀囉訖唎擔}{濕婆神所造}
\ifthenelse{\boolean{fulu}}{\zydy}{\zy}{232}{毗陀夜闍瞋陀夜彌}{呪術 我今悉使斷除}
\ifthenelse{\boolean{fulu}}{\zydy}{\zy}{233}{雞囉夜彌}{悉使釘住}
\ifthenelse{\boolean{fulu}}{\zydy}{\zy}{234}{那囉夜拏訖唎擔}{毗濕奴神所造}
\ifthenelse{\boolean{fulu}}{\zydy}{\zy}{235}{毗陀夜闍瞋陀夜彌}{呪術我今悉使斷除}
\ifthenelse{\boolean{fulu}}{\zydy}{\zy}{236}{雞囉夜彌}{悉使釘住}
\ifthenelse{\boolean{fulu}}{\zydy}{\zy}{237}{怛埵伽嚧茶西訖唎擔}{眞實迦樓羅鳥所造}
\ifthenelse{\boolean{fulu}}{\zydy}{\zy}{238}{毗陀夜闍瞋陀夜彌}{呪術我今悉使斷除}
\ifthenelse{\boolean{fulu}}{\zydy}{\zy}{239}{雞囉夜彌}{悉使釘住}
\ifthenelse{\boolean{fulu}}{\zydy}{\zy}{240}{摩訶迦囉摩怛唎伽拏訖唎擔}{大黑天鬼母眾所造}
\ifthenelse{\boolean{fulu}}{\zydy}{\zy}{241}{毗陀夜闍瞋陀夜彌}{呪術我今悉使斷除}
\ifthenelse{\boolean{fulu}}{\zydy}{\zy}{242}{雞囉夜彌}{悉使釘住}
\ifthenelse{\boolean{fulu}}{\zydy}{\zy}{243}{迦波唎迦訖唎擔}{骷髏外道所造}
\ifthenelse{\boolean{fulu}}{\zydy}{\zy}{244}{毗陀夜闍瞋陀夜彌}{呪術我今悉使斷除}
\ifthenelse{\boolean{fulu}}{\zydy}{\zy}{245}{雞囉夜彌}{悉使釘住}
\ifthenelse{\boolean{fulu}}{\zydy}{\zy}{246}{闍耶羯囉摩度羯囉}{勝作蜜作}
\ifthenelse{\boolean{fulu}}{\zydy}{\zy}{247}{薩婆囉他娑達那訖唎擔}{諸事業成辦者所造}
\ifthenelse{\boolean{fulu}}{\zydy}{\zy}{248}{毗陀夜闍瞋陀夜彌}{呪術我今悉使斷除}
\ifthenelse{\boolean{fulu}}{\zydy}{\zy}{249}{雞囉夜彌}{悉使釘住}
\ifthenelse{\boolean{fulu}}{\zydy}{\zy}{250}{赭咄囉婆耆你訖唎擔}{四姊妹所造}
\ifthenelse{\boolean{fulu}}{\zydy}{\zy}{251}{毗陀耶闍瞋陀夜彌}{呪術我今悉使斷除}
\ifthenelse{\boolean{fulu}}{\zydy}{\zy}{252}{雞囉夜彌}{悉使釘住}
\ifthenelse{\boolean{fulu}}{\zydy}{\zy}{253}{毗唎羊訖唎知}{大自在天隨從}
\ifthenelse{\boolean{fulu}}{\zydy}{\zy}{254}{難陀雞沙囉伽拏般帝}{歡喜自在天眾主}
\ifthenelse{\boolean{fulu}}{\zydy}{\zy}{255}{索醯夜訖唎擔}{眷屬所造}
\ifthenelse{\boolean{fulu}}{\zydy}{\zy}{256}{毗陀夜闍瞋陀夜彌}{呪術我今悉使斷除}
\ifthenelse{\boolean{fulu}}{\zydy}{\zy}{257}{雞囉夜彌}{悉使釘住}
\ifthenelse{\boolean{fulu}}{\zydy}{\zy}{258}{那揭那舍囉婆拏訖唎擔}{裸形無衣苦行外道所造}
\ifthenelse{\boolean{fulu}}{\zydy}{\zy}{259}{毗陀夜闍瞋陀夜彌}{呪術我今悉使斷除}
\ifthenelse{\boolean{fulu}}{\zydy}{\zy}{260}{雞囉夜彌}{悉使釘住}
\ifthenelse{\boolean{fulu}}{\zydy}{\zy}{261}{阿羅漢訖唎擔毗陀夜闍瞋陀夜彌}{(外道)阿羅漢所造呪術我今悉使斷除}
\ifthenelse{\boolean{fulu}}{\zydy}{\zy}{262}{雞囉夜彌}{悉使釘住}
\ifthenelse{\boolean{fulu}}{\zydy}{\zy}{263}{毗多囉伽訖唎擔}{(外道)離欲所造}
\ifthenelse{\boolean{fulu}}{\zydy}{\zy}{264}{毗陀夜闍瞋陀夜彌}{呪術我今悉使斷除}
\ifthenelse{\boolean{fulu}}{\zydy}{\zy}{265}{雞囉夜彌跋闍囉波你}{(外道)執金剛神}
\ifthenelse{\boolean{fulu}}{\zydy}{\zy}{266}{具醯夜具醯夜}{秘密}
\ifthenelse{\boolean{fulu}}{\zydy}{\zy}{267}{迦地般帝訖唎擔}{主 所造}
\ifthenelse{\boolean{fulu}}{\zydy}{\zy}{268}{毗陀夜闍瞋陀夜彌}{呪術我今悉使斷除}
\ifthenelse{\boolean{fulu}}{\zydy}{\zy}{269}{雞囉夜彌}{悉使釘住}
\ifthenelse{\boolean{fulu}}{\zydy}{\zy}{270}{囉叉罔}{守護我}
\ifthenelse{\boolean{fulu}}{\zydy}{\zy}{271}{婆伽梵}{世尊}
\ifthenelse{\boolean{fulu}}{\zydy}{\zy}{272}{印兔那麼麼寫}{如是我彼}

%------------------------------
% 第四会
%------------------------------
\ifthenelse{\boolean{fulu}}{\begin{center}\fangsong\zihao{2} 第四會 \end{center}}{}


\ifthenelse{\boolean{fulu}}{\zydy}{\zy}{273}{婆伽梵}{世尊}
\ifthenelse{\boolean{fulu}}{\zydy}{\zy}{274}{薩怛多般怛囉}{白傘蓋}
\ifthenelse{\boolean{fulu}}{\zydy}{\zy}{275}{南無粹都帝}{我敬禮稱讚}
\ifthenelse{\boolean{fulu}}{\zydy}{\zy}{276}{阿悉多那囉剌迦}{火甘露火日光}
\ifthenelse{\boolean{fulu}}{\zydy}{\zy}{277}{波囉婆悉普吒}{放光普照}
\ifthenelse{\boolean{fulu}}{\zydy}{\zy}{278}{毗迦薩怛多鉢帝唎}{開展白傘蓋}
\ifthenelse{\boolean{fulu}}{\zydy}{\zy}{279}{什佛囉什佛囉}{光明熾盛}
\ifthenelse{\boolean{fulu}}{\zydy}{\zy}{280}{陀囉陀囉}{摧破裂開}
\ifthenelse{\boolean{fulu}}{\zydy}{\zy}{281}{頻陀囉頻陀囉瞋陀瞋陀}{摧破裂開切斷斷裂}
\ifthenelse{\boolean{fulu}}{\zydy}{\zy}{282}{虎𤙖}{}
\ifthenelse{\boolean{fulu}}{\zydy}{\zy}{283}{虎𤙖}{}
\ifthenelse{\boolean{fulu}}{\zydy}{\zy}{284}{泮吒}{}
\ifthenelse{\boolean{fulu}}{\zydy}{\zy}{285}{泮吒泮吒泮吒泮吒}{}
\ifthenelse{\boolean{fulu}}{\zydy}{\zy}{286}{娑訶}{成就圓滿}
\ifthenelse{\boolean{fulu}}{\zydy}{\zy}{287}{醯醯泮}{善來善來}
\ifthenelse{\boolean{fulu}}{\zydy}{\zy}{288}{阿牟迦耶泮}{不空}
\ifthenelse{\boolean{fulu}}{\zydy}{\zy}{289}{阿波囉提訶多泮}{無礙}
\ifthenelse{\boolean{fulu}}{\zydy}{\zy}{290}{婆囉波囉陀泮}{與願}
\ifthenelse{\boolean{fulu}}{\zydy}{\zy}{291}{阿素囉毗陀囉波迦泮}{阿脩羅切裂}
\ifthenelse{\boolean{fulu}}{\zydy}{\zy}{292}{薩婆提鞞弊泮}{一切諸天眾}
\ifthenelse{\boolean{fulu}}{\zydy}{\zy}{293}{薩婆那伽弊泮}{一切諸龍眾}
\ifthenelse{\boolean{fulu}}{\zydy}{\zy}{294}{薩婆藥叉弊泮}{一切夜叉眾}
\ifthenelse{\boolean{fulu}}{\zydy}{\zy}{295}{薩婆乾闥婆弊泮}{一切諸乾闥婆眾}
\ifthenelse{\boolean{fulu}}{\zydy}{\zy}{296}{薩婆補丹那弊泮}{一切諸富單那眾}
\ifthenelse{\boolean{fulu}}{\zydy}{\zy}{297}{迦吒補丹那弊泮}{一切諸迦吒富單那眾}
\ifthenelse{\boolean{fulu}}{\zydy}{\zy}{298}{薩婆突狼枳帝弊泮}{一切諸誤想過眾}
\ifthenelse{\boolean{fulu}}{\zydy}{\zy}{299}{薩婆突澀比𠼐訖瑟帝弊泮}{一切諸懊見過眾}
\ifthenelse{\boolean{fulu}}{\zydy}{\zy}{300}{薩婆什婆唎弊泮}{一切諸瘟疫眾}
\ifthenelse{\boolean{fulu}}{\zydy}{\zy}{301}{薩婆阿播悉摩𠼐弊泮}{一切諸阿波悉魔羅眾}
\ifthenelse{\boolean{fulu}}{\zydy}{\zy}{302}{薩婆舍囉婆拏弊泮}{一切諸苦行眾}
\ifthenelse{\boolean{fulu}}{\zydy}{\zy}{303}{薩婆地帝雞弊泮}{一切諸外道師眾}
\ifthenelse{\boolean{fulu}}{\zydy}{\zy}{304}{薩婆怛摩陀繼弊泮}{一切諸醉鬼眾}
\ifthenelse{\boolean{fulu}}{\zydy}{\zy}{305}{薩婆毗陀耶囉誓遮𠼐弊泮}{一切諸呪王師眾}
\ifthenelse{\boolean{fulu}}{\zydy}{\zy}{306}{闍夜羯囉摩度羯囉}{作勝作蜜}
\ifthenelse{\boolean{fulu}}{\zydy}{\zy}{307}{薩婆羅他娑陀雞弊泮}{成辦諸事業者眾}
\ifthenelse{\boolean{fulu}}{\zydy}{\zy}{308}{毗地夜遮唎弊泮}{諸呪師眾}
\ifthenelse{\boolean{fulu}}{\zydy}{\zy}{309}{者都囉縛耆你弊泮}{四姊妹女天眾}
\ifthenelse{\boolean{fulu}}{\zydy}{\zy}{310}{跋闍囉俱摩唎}{金剛嬌魔哩}
\ifthenelse{\boolean{fulu}}{\zydy}{\zy}{311}{毗陀夜囉誓弊泮}{呪王眾}
\ifthenelse{\boolean{fulu}}{\zydy}{\zy}{312}{摩訶波囉丁羊乂耆唎弊泮}{大甚能調伏眾}
\ifthenelse{\boolean{fulu}}{\zydy}{\zy}{313}{跋闍囉商羯囉夜}{金剛鎖}
\ifthenelse{\boolean{fulu}}{\zydy}{\zy}{314}{波囉丈耆囉闍耶泮}{甚能調伏王}
\ifthenelse{\boolean{fulu}}{\zydy}{\zy}{315}{摩訶迦囉夜}{大黑天}
\ifthenelse{\boolean{fulu}}{\zydy}{\zy}{316}{摩訶末怛唎迦拏}{大鬼母眾眷屬}
\ifthenelse{\boolean{fulu}}{\zydy}{\zy}{317}{南無娑羯唎多夜泮}{作禮敬}
\ifthenelse{\boolean{fulu}}{\zydy}{\zy}{318}{毖瑟拏婢曳泮}{吠紐天妃}
\ifthenelse{\boolean{fulu}}{\zydy}{\zy}{319}{勃囉訶牟尼曳泮}{梵天妃}
\ifthenelse{\boolean{fulu}}{\zydy}{\zy}{320}{阿耆尼曳泮}{火天妃}
\ifthenelse{\boolean{fulu}}{\zydy}{\zy}{321}{摩訶羯唎曳泮}{大黑天妃}
\ifthenelse{\boolean{fulu}}{\zydy}{\zy}{322}{羯囉檀遲曳泮}{死天妃}
\ifthenelse{\boolean{fulu}}{\zydy}{\zy}{323}{蔑怛唎曳泮}{帝释天妃}
\ifthenelse{\boolean{fulu}}{\zydy}{\zy}{324}{嘮怛唎曳泮}{自在天妃}
\ifthenelse{\boolean{fulu}}{\zydy}{\zy}{325}{遮文茶曳泮}{左悶拏天妃}
\ifthenelse{\boolean{fulu}}{\zydy}{\zy}{326}{羯邏囉怛唎曳泮}{黑夜天妃}
\ifthenelse{\boolean{fulu}}{\zydy}{\zy}{327}{迦般唎曳泮}{骷髏外道女}
\ifthenelse{\boolean{fulu}}{\zydy}{\zy}{328}{阿地目質多迦尸摩舍那}{樂墓塚}
\ifthenelse{\boolean{fulu}}{\zydy}{\zy}{329}{婆私你曳泮}{居住鬼母}
\ifthenelse{\boolean{fulu}}{\zydy}{\zy}{330}{演吉質}{若心}
\ifthenelse{\boolean{fulu}}{\zydy}{\zy}{331}{薩埵婆寫}{眾生}
\ifthenelse{\boolean{fulu}}{\zydy}{\zy}{332}{麼麼印兔那麼麼寫}{於我此等我的}

%------------------------------
% 第五会
%------------------------------
\ifthenelse{\boolean{fulu}}{\begin{center}\fangsong\zihao{2} 第五會 \end{center}}{}


\ifthenelse{\boolean{fulu}}{\zydy}{\zy}{333}{突瑟吒質多}{惡心}
\ifthenelse{\boolean{fulu}}{\zydy}{\zy}{334}{阿末怛唎質多}{無慈心}
\ifthenelse{\boolean{fulu}}{\zydy}{\zy}{335}{烏闍訶囉}{食精氣鬼眾}
\ifthenelse{\boolean{fulu}}{\zydy}{\zy}{336}{伽婆訶囉}{食胎鬼眾}
\ifthenelse{\boolean{fulu}}{\zydy}{\zy}{337}{嚧地囉訶囉}{食血鬼眾}
\ifthenelse{\boolean{fulu}}{\zydy}{\zy}{338}{婆娑訶囉}{食膏鬼眾}
\ifthenelse{\boolean{fulu}}{\zydy}{\zy}{339}{摩闍訶囉}{食髓鬼眾}
\ifthenelse{\boolean{fulu}}{\zydy}{\zy}{340}{闍多訶囉}{食生(子息、胎兒)鬼眾}
\ifthenelse{\boolean{fulu}}{\zydy}{\zy}{341}{視毖多訶囉}{食命鬼眾}
\ifthenelse{\boolean{fulu}}{\zydy}{\zy}{342}{跋略夜訶囉}{食鬘鬼眾}
\ifthenelse{\boolean{fulu}}{\zydy}{\zy}{343}{乾陀訶囉}{食香鬼眾}
\ifthenelse{\boolean{fulu}}{\zydy}{\zy}{344}{布史波訶囉}{食花鬼眾}
\ifthenelse{\boolean{fulu}}{\zydy}{\zy}{345}{頗囉訶囉}{食果鬼眾}
\ifthenelse{\boolean{fulu}}{\zydy}{\zy}{346}{婆寫訶囉}{食苗稼鬼眾}
\ifthenelse{\boolean{fulu}}{\zydy}{\zy}{347}{般波質多}{不善心}
\ifthenelse{\boolean{fulu}}{\zydy}{\zy}{348}{突瑟吒質多}{惡心}
\ifthenelse{\boolean{fulu}}{\zydy}{\zy}{349}{嘮陀囉質多}{兇暴心}
\ifthenelse{\boolean{fulu}}{\zydy}{\zy}{350}{藥叉揭囉訶}{諸夜叉所持}
\ifthenelse{\boolean{fulu}}{\zydy}{\zy}{351}{囉剎娑揭囉訶}{諸羅剎所持}
\ifthenelse{\boolean{fulu}}{\zydy}{\zy}{352}{閉㘑多揭囉訶}{諸餓鬼所持}
\ifthenelse{\boolean{fulu}}{\zydy}{\zy}{353}{毗舍遮揭囉訶}{諸毗舍遮所持}
\ifthenelse{\boolean{fulu}}{\zydy}{\zy}{354}{部多揭囉訶}{諸部多所魅}
\ifthenelse{\boolean{fulu}}{\zydy}{\zy}{355}{鳩槃茶揭囉訶}{諸鳩槃荼所魅}
\ifthenelse{\boolean{fulu}}{\zydy}{\zy}{356}{悉乾陀揭囉訶}{諸騫陀所魅}
\ifthenelse{\boolean{fulu}}{\zydy}{\zy}{357}{烏怛摩陀揭囉訶}{諸醉鬼所魅}
\ifthenelse{\boolean{fulu}}{\zydy}{\zy}{358}{車夜揭囉訶}{諸陰鬼所魅}
\ifthenelse{\boolean{fulu}}{\zydy}{\zy}{359}{阿播薩摩囉揭囉訶}{諸阿波悉魔羅所魅}
\ifthenelse{\boolean{fulu}}{\zydy}{\zy}{360}{宅袪革茶耆尼揭囉訶}{荼加荼枳尼鬼所魅}
\ifthenelse{\boolean{fulu}}{\zydy}{\zy}{361}{唎佛帝揭囉訶}{諸黎婆坻所魅}
\ifthenelse{\boolean{fulu}}{\zydy}{\zy}{362}{闍彌迦揭囉訶}{諸閻彌迦所魅}
\ifthenelse{\boolean{fulu}}{\zydy}{\zy}{363}{舍俱尼揭囉訶}{諸舍究尼所魅}
\ifthenelse{\boolean{fulu}}{\zydy}{\zy}{364}{姥陀囉難地迦揭囉訶}{諸曼多難提所魅}
\ifthenelse{\boolean{fulu}}{\zydy}{\zy}{365}{阿藍婆揭囉訶}{諸藍婆所魅}
\ifthenelse{\boolean{fulu}}{\zydy}{\zy}{366}{乾度波尼揭囉訶}{諸乾吒婆尼所魅}
\ifthenelse{\boolean{fulu}}{\zydy}{\zy}{367}{什伐囉堙迦醯迦}{諸熱病:一日}
\ifthenelse{\boolean{fulu}}{\zydy}{\zy}{368}{墜帝藥迦}{二日}
\ifthenelse{\boolean{fulu}}{\zydy}{\zy}{369}{怛隷帝藥迦}{三日}
\ifthenelse{\boolean{fulu}}{\zydy}{\zy}{370}{者突託迦}{四日}
\ifthenelse{\boolean{fulu}}{\zydy}{\zy}{371}{昵提什伐囉毖釤摩什伐囉}{常熱病不盡熱病}
\ifthenelse{\boolean{fulu}}{\zydy}{\zy}{372}{薄底迦}{風病}
\ifthenelse{\boolean{fulu}}{\zydy}{\zy}{373}{鼻底迦}{黃病}
\ifthenelse{\boolean{fulu}}{\zydy}{\zy}{374}{室隷瑟密迦}{痰病}
\ifthenelse{\boolean{fulu}}{\zydy}{\zy}{375}{娑你般帝迦}{三集病}
\ifthenelse{\boolean{fulu}}{\zydy}{\zy}{376}{薩婆什伐囉}{一切病苦}
\ifthenelse{\boolean{fulu}}{\zydy}{\zy}{377}{室嚧吉帝}{頭痛}
\ifthenelse{\boolean{fulu}}{\zydy}{\zy}{378}{末陀鞞達嚧制劍}{半痛飲食不消}
\ifthenelse{\boolean{fulu}}{\zydy}{\zy}{379}{阿綺嚧鉗}{眼病}
\ifthenelse{\boolean{fulu}}{\zydy}{\zy}{380}{目佉嚧鉗}{口病}
\ifthenelse{\boolean{fulu}}{\zydy}{\zy}{381}{羯唎突嚧鉗}{心病}
\ifthenelse{\boolean{fulu}}{\zydy}{\zy}{382}{揭囉訶揭藍}{咽喉痛}
\ifthenelse{\boolean{fulu}}{\zydy}{\zy}{383}{羯拏輸藍}{耳痛}
\ifthenelse{\boolean{fulu}}{\zydy}{\zy}{384}{憚多輸藍}{牙齒痛}
\ifthenelse{\boolean{fulu}}{\zydy}{\zy}{385}{迄唎夜輸藍}{心痛}
\ifthenelse{\boolean{fulu}}{\zydy}{\zy}{386}{末麼輸藍}{關節痛}
\ifthenelse{\boolean{fulu}}{\zydy}{\zy}{387}{跋唎室婆輸藍}{脅痛}
\ifthenelse{\boolean{fulu}}{\zydy}{\zy}{388}{毖栗瑟吒輸藍}{背痛}
\ifthenelse{\boolean{fulu}}{\zydy}{\zy}{389}{烏陀囉輸藍}{肚痛}
\ifthenelse{\boolean{fulu}}{\zydy}{\zy}{390}{羯知輸藍}{腰痛}
\ifthenelse{\boolean{fulu}}{\zydy}{\zy}{391}{跋悉帝輸藍}{隱密處痛}
\ifthenelse{\boolean{fulu}}{\zydy}{\zy}{392}{鄔嚧輸藍}{髀痛}
\ifthenelse{\boolean{fulu}}{\zydy}{\zy}{393}{常伽輸藍}{脛痛}
\ifthenelse{\boolean{fulu}}{\zydy}{\zy}{394}{喝悉多輸藍}{手痛}
\ifthenelse{\boolean{fulu}}{\zydy}{\zy}{395}{跋陀輸藍}{腳痛}
\ifthenelse{\boolean{fulu}}{\zydy}{\zy}{396}{娑房盎伽般囉丈伽輸藍}{遍身疼痛}
\ifthenelse{\boolean{fulu}}{\zydy}{\zy}{397}{部多毖跢茶}{部多鬼起屍鬼}
\ifthenelse{\boolean{fulu}}{\zydy}{\zy}{398}{茶耆尼什婆囉}{荼枳尼鬼病}
\ifthenelse{\boolean{fulu}}{\zydy}{\zy}{399}{陀突嚧迦建咄嚧吉知婆路多毗}{癬癩疥蒼痘疹蜘蛛蒼}
\ifthenelse{\boolean{fulu}}{\zydy}{\zy}{400}{薩般嚧訶凌伽}{火蒼疔蒼}
\ifthenelse{\boolean{fulu}}{\zydy}{\zy}{401}{輸沙怛囉娑那羯囉}{乾消驚怖毒病}
\ifthenelse{\boolean{fulu}}{\zydy}{\zy}{402}{毗沙喻迦}{毒藥厭禱}
\ifthenelse{\boolean{fulu}}{\zydy}{\zy}{403}{阿耆尼烏陀迦}{火災水災}
\ifthenelse{\boolean{fulu}}{\zydy}{\zy}{404}{末囉鞞囉建跢囉}{疫病怨敵險難}
\ifthenelse{\boolean{fulu}}{\zydy}{\zy}{405}{阿迦囉密唎咄怛斂部迦}{夭死土蜂}
\ifthenelse{\boolean{fulu}}{\zydy}{\zy}{406}{地栗剌吒}{馬蜂}
\ifthenelse{\boolean{fulu}}{\zydy}{\zy}{407}{毖唎瑟質迦}{蠍}
\ifthenelse{\boolean{fulu}}{\zydy}{\zy}{408}{薩婆那俱囉}{蛇黃鼠}
\ifthenelse{\boolean{fulu}}{\zydy}{\zy}{409}{肆引伽弊揭囉唎藥叉怛囉芻}{獅子虎熊豺}
\ifthenelse{\boolean{fulu}}{\zydy}{\zy}{410}{末囉視吠帝釤娑鞞釤}{害命者彼等一切}
\ifthenelse{\boolean{fulu}}{\zydy}{\zy}{411}{悉怛多鉢怛囉}{白傘蓋}
\ifthenelse{\boolean{fulu}}{\zydy}{\zy}{412}{摩訶跋闍嚧瑟尼釤}{大金剛頂髻}
\ifthenelse{\boolean{fulu}}{\zydy}{\zy}{413}{摩訶般賴丈耆藍}{大甚能調伏}
\ifthenelse{\boolean{fulu}}{\zydy}{\zy}{414}{夜波突陀舍喻闍那}{所有十二由旬}
\ifthenelse{\boolean{fulu}}{\zydy}{\zy}{415}{辮怛隷拏}{內}
\ifthenelse{\boolean{fulu}}{\zydy}{\zy}{416}{毗陀耶槃曇迦嚧彌}{明呪結界我今作之}
\ifthenelse{\boolean{fulu}}{\zydy}{\zy}{417}{帝殊槃曇迦嚧彌}{威神結界我今作之}
\ifthenelse{\boolean{fulu}}{\zydy}{\zy}{418}{般囉毗陀槃曇迦嚧彌}{最勝明呪結界我今作之}
\ifthenelse{\boolean{fulu}}{\zydy}{\zy}{419}{跢姪他}{卽說呪曰}
\ifthenelse{\boolean{fulu}}{\zydy}{\zy}{420}{唵}{}
\ifthenelse{\boolean{fulu}}{\zydy}{\zy}{421}{阿那隷}{甘露火}
\ifthenelse{\boolean{fulu}}{\zydy}{\zy}{422}{毗舍提}{光明輝矅}
\ifthenelse{\boolean{fulu}}{\zydy}{\zy}{423}{鞞囉跋闍囉陀唎}{勇猛持金剛者}
\ifthenelse{\boolean{fulu}}{\zydy}{\zy}{424}{槃陀槃陀你}{禁縛結界}
\ifthenelse{\boolean{fulu}}{\zydy}{\zy}{425}{跋闍囉謗尼泮}{金剛手}
\ifthenelse{\boolean{fulu}}{\zydy}{\zy}{426}{虎𤙖都嚧甕泮}{}
\ifthenelse{\boolean{fulu}}{\zydy}{\zy}{427}{莎婆訶}{成就福智圓滿}

\end{ZHOUYU}

阿難,是佛頂光聚、悉怛多般怛囉、祕密伽陀微妙章句,出生十方一切諸佛。十方如來因此呪心,得成無上正徧知覺。十方如來執此呪心,降伏諸魔,制諸外道。十方如來乘此呪心,坐寶蓮華,應微塵國;十方如來含此呪心,於微塵國,轉大法輪。十方如來持此呪心,能於十方摩頂授記;自果未成,亦於十方蒙佛授記。十方如來依此呪心,能於十方拔濟群苦。所謂地獄、餓鬼、畜生、盲、聾、瘖、瘂、怨憎會苦、愛別離苦、求不得苦、五陰熾盛,大小諸橫,同時解脫。賊難、兵難、王難、獄難、風火水難,飢渴、貧窮,應念銷散。十方如來隨此呪心,能於十方事善知識。四威儀中供養如意。恆沙如來會中,推為大法王子。十方如來行此呪心,能於十方攝受親因,令諸小乘聞祕密藏,不生驚怖。十方如來誦此呪心,成無上覺;坐菩提樹,入大涅槃。十方如來傳此呪心,於滅度後,付佛法事,究竟住持。嚴淨戒律,悉得清淨。若我說是佛頂光聚般怛羅呪,從旦至暮,音聲相聯,字句中間,亦不重疊,經恆沙劫,終不能盡。亦說此呪名如來頂。汝等有學,未盡輪迴,發心至誠取阿羅漢,不持此呪而坐道場,令其身心遠諸魔事,無有是處。

阿難,若諸世界隨所國土所有眾生,隨國所生樺皮、貝葉、紙素、白㲲,書寫此呪,貯於香囊,是人心昏,未能誦憶;或帶身上,或書宅中,當知是人,盡其生年,一切諸毒所不能害。阿難,我今為汝更說此呪救護世間,得大無畏,成就眾生出世間智。若我滅後,末世眾生,有能自誦,若教他誦,當知如是誦持眾生,火不能燒,水不能溺,大毒小毒所不能害。如是乃至天龍鬼神、精祇、魔魅,所有惡呪皆不能著。心得正受,一切呪詛、厭蠱毒藥、金毒銀毒、草木蟲蛇萬物毒氣,入此人口,成甘露味。一切惡星,並諸鬼神,磣心毒人,於如是人不能起惡。頻那、夜迦,諸惡鬼王,並其眷屬,皆領深恩,常加守護。

阿難當知,是呪常有八萬四千那由他恆河沙俱胝金剛藏王菩薩種族,一一皆有諸金剛眾而為眷屬,晝夜隨侍。設有眾生於散亂心,非三摩地,心憶口持。是金剛王常隨從彼諸善男子;何況決定菩提心者。此諸金剛菩薩藏王,精心陰速,發彼神識。是人應時心能記憶八萬四千恆河沙劫;周徧了知,得無疑惑。從第一劫,乃至後身。生生不生藥叉、羅剎,及富單那、迦吒富單那、鳩槃茶、毗舍遮等,並諸餓鬼,有形、無形、有想、無想,如是惡處。是善男子若讀、若誦、若書、若寫、若帶、若藏,諸色供養,劫劫不生貧窮下賤不可樂處。此諸眾生,縱其自身不作福業;十方如來,所有功德悉與此人。由是得於恆河沙阿僧祇不可說不可說劫,常與諸佛同生一處。無量功德如惡叉聚,同處熏修,永無分散。是故能令破戒之人,戒根清淨;未得戒者,令其得戒;未精進者,令其精進;無智慧者,令得智慧;不清淨者,速得清淨;不持齋戒,自成齋戒。

阿難,是善男子持此呪時,設犯禁戒於未受時,持呪之後,眾破戒罪,無問輕重,一時銷滅。縱經飲酒,食噉五辛,種種不淨,一切諸佛、菩薩、金剛、天仙、鬼神,不將為過。設著不淨破弊衣服,一行一住,悉同清淨。縱不作壇,不入道場,亦不行道,誦持此呪,還同入壇行道,功德無有異也。若造五逆無間重罪,及諸比丘、比丘尼,四棄八棄,誦此呪已,如是重業,猶如猛風吹散沙聚,悉皆消滅,更無毫髮。

阿難,若有眾生從無量無數劫來,所有一切輕重罪障,從前世來,未及懺悔;若能讀誦書寫此呪,身上帶持,若安住處莊宅園館,如是積業,猶湯消雪。不久皆得悟無生忍。

復次,阿難,若有女人未生男女,欲求孕者;若能至心憶念斯呪,或能身上帶此悉怛多般怛囉者,便生福德智慧男女。求長命者,卽得長命;欲求果報速圓滿者,速得圓滿;身命色力,亦復如是。命終之後,隨願往生十方國土;必定不生邊地下賤,何況雜形?

阿難,若諸國土州縣聚落,飢荒疫癘,或復刀兵、賊難、鬬諍,兼餘一切厄難之地,寫此神呪安城四門,並諸支提,或脫闍上;令其國土所有眾生,奉迎斯呪,禮拜恭敬,一心供養,令其人民各各身佩,或各各安所居宅地,一切災厄,悉皆銷滅。阿難,在在處處國土眾生,隨有此呪,天龍歡喜,風雨順時,五穀豐殷,兆庶安樂。亦復能鎮一切惡星、隨方變怪,災障不起,人無橫夭;杻械枷鎖不著其身,晝夜安眠,常無惡夢。

阿難,是娑婆界有八萬四千災變惡星,二十八大惡星而為上首。復有八大惡星以為其主,作種種形;出現世時,能生眾生種種災異。有此呪地,悉皆消滅。十二由旬成結界地,諸惡災祥永不能入。是故如來宣示此呪於未來世,保護初學諸修行者入三摩地,身心泰然,得大安隱。更無一切諸魔鬼神,及無始來冤橫宿殃、舊業陳債,來相惱害。汝及眾中諸有學人,及未來世諸修行者,依我壇場,如法持戒;所受戒主,逢清淨僧。持此呪心,不生疑悔。是善男子,於此父母所生之身不得心通,十方如來便為妄語。

說是語已,會中無量百千金剛,一時佛前合掌頂禮而白佛言:如佛所說,我當誠心保護如是修菩提者。爾時,梵王並天帝釋、四天大王,亦於佛前同時頂禮而白佛言:審有如是修學善人,我當盡心至誠保護,令其一生所作如願。復有無量藥叉大將、諸羅剎王、富單那王、鳩槃茶王、毗舍遮王、頻那夜迦、諸大鬼王,及諸鬼帥,亦於佛前合掌頂禮,我亦誓願護持是人,令菩提心速得圓滿。復有無量日月天子、風師、雨師、雲師、雷師,並電伯等,年歲巡官,諸星眷屬,亦於會中頂禮佛足而白佛言:我亦保護是修行人,安立道場,得無所畏。復有無量山神、海神,一切土地,水、陸、空行,萬物精祇,並風神王、無色界天,於如來前,同時稽首而白佛言:我亦保護是修行人,得成菩提,永無魔事。

爾時,八萬四千那由他恆河沙俱胝金剛藏王菩薩,在大會中,卽從座起,頂禮佛足而白佛言:世尊,如我等輩所修功業,久成菩提,不取涅槃;常隨此呪,救護末世修三摩地正修行者。世尊,如是修心求正定人,若在道場,及餘經行,乃至散心遊戲聚落,我等徒眾,常當隨從侍衛此人。縱令魔王、大自在天,求其方便,終不可得。諸小鬼神,去此善人十由旬外;除彼發心,樂修禪者。世尊,如是惡魔,若魔眷屬欲來侵擾是善人者,我以寶杵隕碎其首,猶如微塵。恆令此人所作如願。

阿難卽從座起,頂禮佛足而白佛言:我輩愚鈍,好為多聞,於諸漏心,未求出離;蒙佛慈誨,得正熏修,身心快然,獲大饒益。世尊,如是修證佛三摩提,未到涅槃。云何名為乾慧之地?四十四心,至何漸次,得修行目?詣何方所,名入地中?云何名為等覺菩薩?作是語已,五體投地;大眾一心,佇佛慈音,瞪瞢瞻仰。

爾時,世尊讚阿難言:善哉!善哉!汝等乃能普為大眾,及諸末世一切眾生修三摩地、求大乘者,從於凡夫,終大涅槃,懸示無上正修行路。汝今諦聽,當為汝說。阿難大眾,合掌刳心,默然受教。佛言:阿難當知,妙性圓明,離諸名相;本來無有世界、眾生,因妄有生,因生有滅,生滅名妄;滅妄名眞,是稱如來無上菩提及大涅槃二轉依號。阿難,汝今欲修眞三摩地,直詣如來大涅槃者。先當識此眾生、世界,二顚倒因。顚倒不生,斯則如來眞三摩地。

阿難,云何名為眾生顚倒?阿難,由性明心、性明圓故。因明發性,性妄見生;從畢竟無,成究竟有。此有所有,非因所因;住所住相,了無根本。本此無住,建立世界及諸眾生。迷本圓明,是生虛妄;妄性無體,非有所依。將欲復眞,欲眞已非眞眞如性。非眞求復,宛成非相。非生、非住、非心、非法,展轉發生;生力發明,熏以成業。同業相感;因有感業,相滅相生。由是故有眾生顚倒。

阿難,云何名為世界顚倒?是有所有,分段妄生,因此界立;非因所因,無住所住,遷流不住,因此世成。三世四方,和合相涉,變化眾生成十二類。是故世界因動有聲、因聲有色、因色有香、因香有觸、因觸有味、因味知法,六亂妄想成業性故,十二區分由此輪轉。是故世間聲、香、味、觸,窮十二變,為一旋復。乘此輪轉顚倒相故,是有世界,卵生、胎生、濕生、化生、有色、無色、有想、無想、若非有色、若非無色、若非有想、若非無想。

阿難,由因世界虛妄輪迴動顚倒故,和合氣成八萬四千飛沈亂想,如是故有卵羯邏藍流轉國土,魚、鳥、龜、蛇,其類充塞。

由因世界雜染輪迴欲顚倒故,和合滋成八萬四千橫豎亂想,如是故有胎遏蒲曇流轉國土,人、畜、龍、仙,其類充塞。

由因世界執著輪迴趣顚倒故,和合煖成八萬四千翻覆亂想,如是故有濕相蔽尸流轉國土,含蠢輭動,其類充塞。

由因世界變易輪迴假顚倒故,和合觸成八萬四千新故亂想,如是故有化相羯南流轉國土,轉蛻飛行,其類充塞。

由因世界留礙輪迴障顚倒故,和合著成八萬四千精矅亂想,如是故有色相羯南流轉國土,休咎精明,其類充塞。

由因世界銷散輪迴惑顚倒故,和合暗成八萬四千陰隱亂想,如是故有無色羯南流轉國土,空散銷沈,其類充塞。

由因世界罔象輪迴影顚倒故,和合憶成八萬四千潛結亂想,如是故有想相羯南流轉國土,神鬼精靈,其類充塞。

由因世界愚鈍輪迴癡顚倒故,和合頑成八萬四千枯槁亂想,如是故有無想羯南流轉國土,精神化為土、木、金、石,其類充塞。

由因世界相待輪迴偽顚倒故,和合染成八萬四千因依亂想,如是故有非有色相成色羯南流轉國土,諸水母等以蝦為目,其類充塞。

由因世界相引輪迴性顚倒故,和合呪成八萬四千呼召亂想,由是故有非無色相無色羯南流轉國土,呪詛厭生,其類充塞。

由因世界合妄輪迴罔顚倒故,和合異成八萬四千迴互亂想,如是故有非有想相成想羯南流轉國土,彼蒲盧等異質相成,其類充塞。

由因世界怨害輪迴殺顚倒故,和合怪成八萬四千食父母想,如是故有非無想相無想羯南流轉國土;如土梟等附塊為兒,及破鏡鳥以毒樹果抱為其子,子成,父母皆遭其食,其類充塞。是名眾生十二種類。

%\pagebreak

%\pagestyle{empty}
%\begin{center}
%\setlength{\fboxsep}{1mm} 
%\fbox{\includegraphics[width=\textwidth]{pictures/yn.png}}
%\end{center}
