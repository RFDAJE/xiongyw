% $Id: juan06.tex,v 1.4 2011/08/09 09:08:36 xiongyw Exp $


爾時,觀世音菩薩卽從座起,頂禮佛足而白佛言:世尊,憶念我昔無數恆河沙劫,於時有佛出現於世,名觀世音。我於彼佛發菩提心;彼佛教我,從聞、思、修,入三摩地。初於聞中,入流亡所;所入旣寂,動靜二相,了然不生。如是漸增,聞所聞盡。盡聞不住,覺所覺空;空覺極圓,空所空滅。生滅旣滅,寂滅現前。忽然超越世出世間,十方圓明,獲二殊勝。一者、上合十方諸佛本妙覺心,與佛如來同一慈力。二者、下合十方一切六道眾生,與諸眾生同一悲仰。

世尊,由我供養觀音如來,蒙彼如來授我如幻聞熏聞修金剛三昧;與佛如來同慈力故,令我身成三十二應,入諸國土。

世尊,若諸菩薩入三摩地,進修無漏,勝解現圓;我現佛身而為說法,令其解脫。

若諸有學,寂靜妙明,勝妙現圓;我於彼前現獨覺身而為說法,令其解脫。

若諸有學斷十二緣,緣斷勝性,勝妙現圓;我於彼前現緣覺身而為說法,令其解脫。

若諸有學得四諦空,修道入滅,勝性現圓;我於彼前現聲聞身而為說法,令其解脫。

若諸眾生欲心明悟,不犯欲塵,欲身清淨;我於彼前現梵王身而為說法,令其解脫。

若諸眾生欲為天主,統領諸天;我於彼前現帝釋身而為說法,令其成就。

若諸眾生欲身自在,遊行十方;我於彼前現自在天身而為說法,令其成就。

若諸眾生欲身自在,飛行虛空;我於彼前現大自在天身而為說法,令其成就。

若諸眾生愛統鬼神,救護國土;我於彼前現天大將軍身而為說法,令其成就。

若諸眾生愛統世界,保護眾生;我於彼前現四天王身而為說法,令其成就。

若諸眾生愛生天宮,驅使鬼神;我於彼前現四天王國太子身而為說法,令其成就。

若諸眾生樂為人王,我於彼前現人王身而為說法,令其成就。

若諸眾生愛主族姓,世間推讓;我於彼前現長者身而為說法,令其成就。

若諸眾生愛談名言,清淨自居;我於彼前現居士身而為說法,令其成就。

若諸眾生愛治國土,剖斷邦邑;我於彼前現宰官身而為說法,令其成就。

若諸眾生愛諸數術,攝衛自居;我於彼前現婆羅門身而為說法,令其成就。

若有男子,好學出家,持諸戒律;我於彼前現比丘身而為說法,令其成就。

若有女人,好學出家,持諸禁戒;我於彼前現比丘尼身而為說法,令其成就。

若有男子,樂持五戒;我於彼前現優婆塞身而為說法,令其成就。

若有女子,五戒自居;我於彼前現優婆夷身而為說法,令其成就。

若有女人,內政立身,以修家國;我於彼前現女主身,及國夫人、命婦大家,而為說法,令其成就。

若有眾生,不壞男根;我於彼前現童男身而為說法,令其成就。

若有處女,愛樂處身,不求侵暴;我於彼前現童女身而為說法,令其成就。

若有諸天,樂出天倫;我現天身而為說法,令其成就。

若有諸龍,樂出龍倫;我現龍身而為說法,令其成就。

若有藥叉,樂度本倫;我於彼前現藥叉身而為說法,令其成就。

若乾闥婆,樂脫其倫;我於彼前現乾闥婆身而為說法,令其成就。

若阿脩羅,樂脫其倫;我於彼前現阿脩羅身而為說法,令其成就。

若緊那羅,樂脫其倫;我於彼前現緊那羅身而為說法,令其成就。

若摩呼羅伽,樂脫其倫;我於彼前現摩呼羅伽身而為說法,令其成就。

若諸眾生,樂人修人;我現人身而為說法,令其成就。

若諸非人,有形無形、有想無想,樂脫其倫;我於彼前皆現其身而為說法,令其成就。

是名妙淨三十二應入國土身;皆以三昧聞熏聞修,無作妙力,自在成就。

世尊,我復以此聞熏聞修金剛三昧無作妙力,與諸十方三世六道一切眾生同悲仰故;令諸眾生於我身心獲十四種無畏功德。

一者,由我不自觀音,以觀觀者。令彼十方苦惱眾生,觀其音聲,卽得解脫。

二者,知見旋復。令諸眾生設入大火,火不能燒。

三者,觀聽旋復。令諸眾生,大水所漂,水不能溺。

四者,斷滅妄想,心無殺害。令諸眾生入諸鬼國,鬼不能害。

五者,熏聞成聞;六根銷復,同於聲聽。能令眾生臨當被害,刀段段壞;使其兵戈猶如割水,亦如吹光,性無摇動。

六者,聞熏精明,明徧法界;則諸幽暗,性不能全。能令眾生,藥叉、羅剎、鳩槃茶鬼,及毗舍遮、富單那等,雖近其旁,目不能視。

七者,音性圓銷,觀聽返入,離諸塵妄。能令眾生,禁繫枷鎖所不能著。

八者,滅音圓聞,徧生慈力。能令眾生經過險路,賊不能劫。

九者,熏聞離塵,色所不劫。能令一切多婬眾生,遠離貪欲。

十者,純音無塵;根境圓融,無對所對。能令一切忿恨眾生,離諸瞋恚。

十一者,銷塵旋明;法界身心,猶如璢璃,朗徹無礙。能令一切昏鈍性障、諸阿顚迦,永離癡暗。

十二者,融形復聞,不動道場,涉入世間,不壞世界;能徧十方供養微塵諸佛如來,各各佛邊,為法王子。能令法界無子眾生欲求男者,誕生福德智慧之男。

十三者,六根圓通,明照無二,含十方界,立大圓鏡,空如來藏;承順十方微塵如來祕密法門,受領無失。能令法界無子眾生欲求女者,誕生端正、福德柔順、眾人愛敬有相之女。

十四者,此三千大千世界百億日月,現住世間諸法王子,有六十二恆河沙數,修法垂範,教化眾生。隨順眾生,方便智慧,各各不同。由我所得圓通本根發妙耳門,然後身心微妙含容周徧法界。能令眾生持我名號,與彼共持六十二恆河沙諸法王子,二人福德,正等無異。世尊,我一名號,與彼眾多名號無異;由我修習得眞圓通。是名十四施無畏力,福備眾生。

世尊,我又獲是圓通,修證無上道故,又能善獲四不思議無作妙德。

一者,由我初獲妙妙聞心,心精遺聞;見、聞、覺、知,不能分隔,成一圓融清淨寶覺。故我能現眾多妙容,能說無邊祕密神呪。其中或現一首、三首、五首、七首、九首、十一首,如是乃至一百八首、千首、萬首、八萬四千爍迦羅首。二臂、四臂、六臂、八臂、十臂、十二臂,十四、十六、十八、二十,至二十四,如是乃至一百八臂、千臂、萬臂、八萬四千母多羅臂。二目、三目、四目、九目,如是乃至一百八目、千目、萬目、八萬四千清淨寶目。或慈、或威、或定、或慧,救護眾生,得大自在。

二者,由我聞思,脫出六塵;如聲度垣,不能為礙。故我妙能現一一形,誦一一呪;其形其呪,能以無畏施諸眾生。是故十方微塵國土,皆名我為施無畏者。

三者,由我修習本妙圓通清淨本根;所遊世界,皆令眾生捨身珍寶,求我哀愍。

四者,我得佛心,證於究竟;能以珍寶種種,供養十方如來,傍及法界六道眾生。求妻得妻,求子得子,求三昧得三昧,求長壽得長壽,如是乃至求大涅槃得大涅槃。

佛問圓通,我從耳門圓照三昧,緣心自在;因入流相,得三摩地,成就菩提,斯為第一。世尊,彼佛如來歎我善得圓通法門,於大會中,授記我為觀世音號。由我觀聽十方圓明,故觀音名,徧十方界。

爾時,世尊於師子座,從其五體同放寶光,遠灌十方微塵如來,及法王子諸菩薩頂。彼諸如來亦於五體同放寶光,從微塵方來灌佛頂,並灌會中諸大菩薩及阿羅漢。林木池沼,皆演法音,交光相羅,如寶絲網;是諸大眾得未曾有,一切普獲金剛三昧。卽時,天雨百寶蓮華,青黃赤白間錯紛糅,十方虛空成七寶色。此娑婆界大地山河,俱時不現;唯見十方微塵國土,合成一界。梵唄詠歌,自然敷奏。

於是如來告文殊師利法王子:汝今觀此二十五無學、諸大菩薩,及阿羅漢,各說最初成道方便,皆言修習眞實圓通。彼等修行,實無優劣前後差別。我今欲令阿難開悟二十五行,誰當其根?兼我滅後,此界眾生入菩薩乘,求無上道;何方便門,得易成就?

文殊師利法王子奉佛慈旨,卽從座起,頂禮佛足,承佛威神,說偈對佛:

\begin{JISONG}
覺海性澄圓,圓澄覺元妙;

元明照生所,所立照性亡。

迷妄有虛空,依空立世界;

想澄成國土,知覺乃眾生。

空生大覺中,如海一漚發;

有漏微塵國,皆依空所生。

漚滅空本無,況復諸三有。

歸元性無二,方便有多門;

聖性無不通,順逆皆方便。

初心入三昧,遲速不同倫。

色想結成塵,精了不能徹;

如何不明徹,於是獲圓通。

音聲雜語言,但伊名句味;

一非含一切,云何獲圓通。

香以合中知,離則元無有;

不恆其所覺,云何獲圓通。

味性非本然,要以味時有;

其覺不恆一,云何獲圓通。

觸以所觸明,無所不明觸;

合離性非定,云何獲圓通。

法稱為內塵,憑塵必有所;

能所非徧涉,云何獲圓通。

見性雖洞然,明前不明後;

四維虧一半,云何獲圓通。

鼻息出入通,現前無交氣;

支離匪涉入,云何獲圓通。

舌非入無端,因味生覺了;

味亡了無有,云何獲圓通。

身與所觸同,各非圓覺觀;

涯量不冥會,云何獲圓通。

知根雜亂思,湛了終無見;

想念不可脫,云何獲圓通。

識見雜三和,詰本稱非相;

自體先無定,云何獲圓通。

心聞洞十方,生於大因力;

初心不能入,云何獲圓通。

鼻想本權機,只令攝心住;

住成心所住,云何獲圓通。

說法弄音文,開悟先成者;

名句非無漏,云何獲圓通。

持犯但束身,非身無所束;

元非徧一切,云何獲圓通。

神通本宿因,何關法分別;

念緣非離物,云何獲圓通。

若以地性觀,堅礙非通達;

有為非聖性,云何獲圓通。

若以水性觀,想念非眞實;

如如非覺觀,云何獲圓通。

若以火性觀,厭有非眞離;

非初心方便,云何獲圓通。

若以風性觀,動寂非無對;

對非無上覺,云何獲圓通。

若以空性觀,昏鈍先非覺;

無覺異菩提,云何獲圓通。

若以識性觀,觀識非常住;

存心乃虛妄,云何獲圓通。

諸行是無常,念性元生滅;

因果今殊感,云何獲圓通。

我今白世尊:佛出娑婆界,

此方眞教體,清淨在音聞。

欲取三摩提,實以聞中入;

離苦得解脫,良哉觀世音!

於恆沙劫中,入微塵佛國;

得大自在力,無畏施眾生。

妙音觀世音,梵音海潮音;

救世悉安寧,出世獲常住。

我今啟如來,如觀音所說:

譬如人靜居,十方俱擊鼓;

十處一時聞,此則圓眞實。

目非觀障外,口鼻亦復然;

身以合方知,心念紛無緒。

隔垣聽音響,遐邇俱可聞;

五根所不齊,是則通眞實。

音聲性動靜,聞中為有無;

無聲號無聞,非實聞無性。

聲無卽無滅,聲有亦非生;

生滅二圓離,是則常眞實。

縱令在夢想,不為不思無;

覺觀出思惟,身心不能及。

今此娑婆國,聲論得宣明。

眾生迷本聞,循聲故流轉;

阿難縱強記,不免落邪思。

豈非隨所淪,旋流獲無妄。

阿難汝諦聽:我承佛威力,

宣說金剛王;如幻不思議,

佛母眞三昧。汝聞微塵佛,

一切祕密門;欲漏不先除,

蓄聞成過誤。將聞持佛佛,

何不自聞聞?聞非自然生,

因聲有名字;旋聞與聲脫,

能脫欲誰名?一根旣返源,

六根成解脫。見聞如幻翳,

三界若空華;聞復翳根除,

塵銷覺圓淨。淨極光通達,

寂照含虛空;卻來觀世間,

猶如夢中事。摩登伽在夢,

誰能留汝形?如世巧幻師,

幻作諸男女。雖見諸根動,

要以一機抽;息機歸寂然,

諸幻成無性。六根亦如是。

元依一精明,分成六和合;

一處成休復,六用皆不成。

塵垢應念銷,成圓明淨妙。

餘塵尚諸學,明極卽如來。

大眾及阿難,旋汝倒聞機,

反聞聞自性;性成無上道,

圓通實如是。此是微塵佛,

一路涅槃門。過去諸如來,

斯門已成就;現在諸菩薩,

今各入圓明;未來修學人,

當依如是法。我亦從中證,

非唯觀世音。誠如佛世尊,

詢我諸方便。以救諸末劫,

求出世間人;成就涅槃心,

觀世音為最。自餘諸方便,

皆是佛威神,卽事捨塵勞。

非是常修學,淺深同說法。

頂禮如來藏,無漏不思議;

願加被未來,於此門無惑。

方便易成就,堪以教阿難,

及末劫沈淪。但以此根修,

圓通超餘者,眞實心如是。

\end{JISONG}

於是阿難及諸大眾身心了然,得大開示。觀佛菩提及大涅槃,猶如有人因事遠遊,未得歸還,明了其家所歸道路。普會大眾、天龍八部、有學二乘,及諸一切新發心菩薩,其數凡有十恆河沙,皆得本心,遠塵離垢,獲法眼淨。性比丘尼聞說偈已,成阿羅漢。無量眾生,皆發無等等阿耨多羅三藐三菩提心。

阿難整衣服,於大眾中,合掌頂禮;心跡圓明,悲欣交集。欲益未來諸眾生故,稽首白佛:大悲世尊,我今已悟成佛法門,是中修行,得無疑惑。常聞如來說如是言:自未得度,先度人者,菩薩發心;自覺已圓,能覺他者,如來應世。我雖未度,願度末劫,一切眾生。世尊,此諸眾生,去佛漸遠;邪師說法,如恆河沙。欲攝其心入三摩地,云何令其安立道場,遠諸魔事?於菩提心,得無退屈?

爾時,世尊於大眾中,稱讚阿難:善哉!善哉!如汝所問,安立道場,救護眾生末劫沈溺。汝今諦聽,當為汝說。

阿難、大眾唯然奉教。

佛告阿難:汝常聞我毗奈耶中,宣說修行三決定義,所謂攝心為戒;因戒生定,因定發慧,是則名為三無漏學。

阿難,云何攝心我名為戒?

若諸世界六道眾生,其心不婬,則不隨其生死相續。汝修三昧,本出塵勞;婬心不除,塵不可出。縱有多智,禪定現前;如不斷婬,必落魔道:上品魔王,中品魔民,下品魔女。彼等諸魔亦有徒眾,各各自謂成無上道。我滅度後,末法之中,多此魔民熾盛世間,廣行貪婬,為善知識;令諸眾生落愛見坑,失菩提路。汝教世人修三摩地,先斷心婬,是名如來先佛世尊第一決定清淨明誨。是故阿難,若不斷婬,修禪定者,如蒸沙石欲其成飯,經百千劫,只名熱沙。何以故?此非飯本,沙石成故。汝以婬身,求佛妙果,縱得妙悟,皆是婬根。根本成婬,輪轉三途,必不能出;如來涅槃,何路修證?必使婬機,身心俱斷,斷性亦無,於佛菩提斯可希冀。如我此說,名為佛說;不如此說,卽波旬說。

阿難,又諸世界,六道眾生,其心不殺,則不隨其生死相續。汝修三昧,本出塵勞,殺心不除,塵不可出。縱有多智,禪定現前,如不斷殺,必落神道:上品之人,為大力鬼;中品則為飛行夜叉,諸鬼帥等;下品當為地行羅剎。彼諸鬼神,亦有徒眾,各各自謂,成無上道。我滅度後,末法之中,多此鬼神熾盛世間,自言食肉得菩提路。阿難,我令比丘食五淨肉,此肉皆我神力化生,本無命根。汝婆羅門,地多蒸濕;加以沙石,草菜不生。我以大悲神力所加,因大慈悲,假名為肉,汝得其味。奈何如來滅度之後,食眾生肉,名為釋子!汝等當知,是食肉人,縱得心開,似三摩地,皆大羅剎;報終必沈生死苦海,非佛弟子。如是之人,相殺相吞,相食未已,云何是人得出三界?汝教世人修三摩地,次斷殺生。是名如來先佛世尊第二決定清淨明誨。是故阿難,若不斷殺,修禪定者;譬如有人,自塞其耳,高聲大叫,求人不聞,此等名為欲隱彌露。清淨比丘,及諸菩薩,於歧路行,不蹋生草,況以手拔?云何大悲,取諸眾生血肉充食?若諸比丘,不服東方絲、緜、絹、帛,及是此土,靴、履、裘、毳、乳、酪、醍醐,如是比丘於世眞脫;酬還宿債,不遊三界。何以故?服其身分,皆為彼緣;如人食其地中百穀,足不離地。必使身心於諸眾生,若身身分,身心二途,不服不食,我說是人眞解脫者。如我此說,名為佛說;不如此說,卽波旬說。

阿難,又復世界,六道眾生,其心不偷,則不隨其生死相續。汝修三昧,本出塵勞;偷心不除,塵不可出。縱有多智,禪定現前;如不斷偷,必落邪道:上品精靈;中品妖魅;下品邪人,諸魅所著。彼等群邪,亦有徒眾,各各自謂成無上道。我滅度後,末法之中,多此妖邪熾盛世間;潛匿奸欺,稱善知識,各自謂己得上人法。詃惑無識,恐令失心;所過之處,其家耗散。我教比丘,循方乞食;令其捨貪,成菩提道。諸比丘等,不自熟食,寄於殘生,旅泊三界,示一往還,去已無返。云何賊人假我衣服、裨販如來,造種種業,皆言佛法?卻非出家,具戒比丘為小乘道。由是疑誤無量眾生,墮無間獄。若我滅後,其有比丘,發心決定修三摩提。能於如來形像之前,身然一燈,燒一指節,及於身上爇一香炷。我說是人無始宿債,一時酬畢,長揖世間,永脫諸漏。雖未卽明無上覺路,是人於法已決定心。若不為此捨身微因,縱成無為,必還生人酬其宿債,如我馬麥正等無異。汝教世人修三摩地,後斷偷盜,是名如來先佛世尊第三決定清淨明誨。是故阿難,若不斷偷,修禪定者;譬如有人,水灌漏卮,欲求其滿,縱經塵劫,終無平復。若諸比丘,衣缽之餘,分寸不蓄;乞食餘分,施餓眾生。於大集會,合掌禮眾;有人捶詈,同於稱讚。必使身心二俱捐捨;身肉骨血,與眾生共。不將如來不了義說;迴為己解以誤初學。佛印是人得眞三昧。如我所說,名為佛說;不如此說,卽波旬說。  

阿難,如是世界六道眾生,雖則身心無殺、盜、婬,三行已圓;若大妄語,卽三摩地不得清淨,成愛見魔,失如來種。所謂:未得謂得,未證言證。或求世間尊勝第一,謂前人言:我今已得須陀洹果、斯陀含果、阿那含果、阿羅漢道、辟支佛乘、十地、地前,諸位菩薩。求彼禮懺,貪其供養。是一顚迦,消滅佛種;如人以刀,斷多羅木。佛記是人,永殞善根,無復知見,沈三苦海,不成三昧。我滅度後,勑諸菩薩,及阿羅漢,應身生彼末法之中,作種種形,度諸輪轉。或作沙門,白衣居士,人王、宰官、童男、童女,如是乃至婬女、寡婦,姦、偷、屠、販,與其同事,稱讚佛乘,令其身心入三摩地。終不自言:我眞菩薩,眞阿羅漢,洩佛密因,輕言未學。唯除命終,陰有遺付。云何是人,惑亂眾生,成大妄語?汝教世人修三摩地,後復斷除諸大妄語,是名如來先佛世尊第四決定清淨明誨。是故阿難,若不斷其大妄語者,如刻人糞,為旃檀形;欲求香氣,無有是處。我教比丘直心道場,於四威儀,一切行中,尚無虛假;云何自稱得上人法?譬如窮人妄號帝王,自取誅滅;況復法王,如何妄竊?因地不眞,果招紆曲。求佛菩提,如噬臍人,欲誰成就?若諸比丘,心如直弦,一切眞實,入三摩地,永無魔事。我印是人,成就菩薩無上知覺。如我所說,名為佛說;不如此說,卽波旬說。

%\pagebreak

%\pagestyle{empty}
%\begin{center}
%\setlength{\fboxsep}{1mm} 
%\fbox{\includegraphics[width=\textwidth]{pictures/sjdyyh.png}}
%\end{center}
