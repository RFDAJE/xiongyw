% $Id: juan03.tex,v 1.3 2011/08/09 09:08:36 xiongyw Exp $


復次,阿難,云何六入,本如來藏,妙眞如性?

阿難,卽彼目睛,瞪發勞者,兼目與勞,同是菩提瞪發勞相。因於明、暗二種妄塵,發見居中,吸此塵象,名為見性。此見離彼明、暗二塵,畢竟無體。如是,阿難,當知是見,非明暗來,非於根出,不於空生。何以故?若從明來,暗卽隨滅,應非見暗;若從暗來,明卽隨滅,應無見明。若從根生,必無明暗;如是見精,本無自性。若於空出,前矚塵象,歸當見根;又空自觀,何關汝入?是故當知眼入虛妄,本非因緣,非自然性。

阿難,譬如有人,以兩手指急塞其耳,耳根勞故,頭中作聲。兼耳與勞,同是菩提瞪發勞相。因於動、靜二種妄塵,發聞居中,吸此塵象,名聽聞性。此聞離彼動、靜二塵,畢竟無體。如是,阿難,當知是聞,非動靜來,非於根出,不於空生。何以故?若從靜來,動卽隨滅,應非聞動。若從動來,靜卽隨滅,應無覺靜。若從根生,必無動靜;如是聞體,本無自性。若於空出,有聞成性,卽非虛空;又空自聞,何關汝入?是故當知耳入虛妄,本非因緣,非自然性。

阿難,譬如有人,急畜其鼻,畜久成勞,則於鼻中聞有冷觸;因觸分別通、塞、虛、實,如是乃至諸香臭氣。兼鼻與勞,同是菩提瞪發勞相。因於通、塞二種妄塵,發聞居中,吸此塵象,名齅聞性。此聞離彼通、塞二塵,畢竟無體。當知是聞,非通塞來,非於根出,不於空生。何以故?若從通來,塞則聞滅,云何知塞?如因塞有,通則無聞,云何發明香、臭等觸?若從根生,必無通、塞;如是聞機,本無自性。若從空出,是聞自當迴齅汝鼻;空自有聞,何關汝入?是故當知鼻入虛妄,本非因緣,非自然性。

阿難,譬如有人,以舌舐吻,熟舐令勞。其人若病,則有苦味;無病之人,微有甜觸。由甜與苦,顯此舌根;不動之時,淡性常在。兼舌與勞,同是菩提瞪發勞相。因甜苦、淡二種妄塵,發知居中,吸此塵象,名知味性。此知味性,離彼甜苦及淡二塵,畢竟無體。如是,阿難,當知如是嚐苦、淡知,非甜、苦來,非因淡有,又非根出,不於空生。何以故?若甜、苦來,淡則知滅,云何知淡?若從淡出,甜卽知亡,復云何知甜、苦二相?若從舌生,必無甜、淡,及與苦塵;斯知味根,本無自性。若從空出,虛空自味,非汝口知;又空自知,何關汝入?是故當知舌入虛妄,本非因緣,非自然性。

阿難,譬如有人,以一冷手,觸於熱手,若冷勢多,熱者從冷;若熱功勝,冷者成熱。如是以此合覺之觸,顯於離知;涉勢若成,因於勞觸。兼身與勞,同是菩提瞪發勞相。因於離、合二種妄塵,發覺居中,吸此塵象,名知覺性。此知覺體,離彼離、合違順二塵,畢竟無體。如是,阿難,當知是覺,非離、合來,非違、順有,不於根出,又非空生。何以故?若合時來,離當已滅,云何覺離?違、順二相,亦復如是。若從根出,必無離、合、違、順四相;則汝身知,元無自性。必於空出,空自知覺,何關汝入?是故當知身入虛妄,本非因緣,非自然性。

阿難,譬如有人,勞倦則眠,睡熟便寤;覽塵斯憶,失憶為忘。是其顚倒生、住、異、滅;吸習中歸,不相踰越,稱意知根。兼意與勞,同是菩提瞪發勞相。因於生、滅二種妄塵,集知居中;吸撮內塵,見、聞逆流,流不及地,名覺知性。此覺知性,離彼寤、寐生滅二塵,畢竟無體。如是,阿難,當知如是覺知之根,非寤寐來,非生滅有,不於根出,亦非空生。何以故?若從寤來,寐卽隨滅,將何為寐?必生時有,滅卽同無,令誰受滅?若從滅有,生卽滅無,誰知生者?若從根出,寤、寐二相,隨身開合,離斯二體;此覺知者,同於空華,畢竟無性。若從空出,自是空知,何關汝入?是故當知意入虛妄,本非因緣,非自然性。

復次,阿難,云何十二處,本如來藏,妙眞如性?

阿難,汝且觀此祇陀樹林,及諸泉池。於意云何?此等為是色生眼見?眼生色相?阿難,若復眼根生色相者,見空非色,色性應銷,銷則顯發一切都無;色相旣無,誰明空質?空亦如是。若復色塵生眼見者,觀空非色,見卽銷亡,亡則都無,誰明空、色?是故當知見與色、空,俱無處所。卽色與見二處虛妄,本非因緣,非自然性。

阿難,汝更聽此祇陀園中,食辦擊鼓,眾集撞鐘;鐘鼓音聲,前後相續。於意云何?此等為是聲來耳邊?耳往聲處?阿難,若復此聲來於耳邊,如我乞食室羅筏城,在祇陀林則無有我;此聲必來阿難耳處,目連、迦葉應不俱聞?何況其中一千二百五十沙門,一聞鐘聲,同來食處?若復汝耳往彼聲邊,如我歸住祇陀林中,在室羅城則無有我。汝聞鼓聲,其耳已往擊鼓之處;鐘聲齊出,應不俱聞,何況其中象、馬、牛、羊種種音響?若無來往,亦復無聞。是故當知,聽與音聲俱無處所。卽聽與聲二處虛妄,本非因緣,非自然性。

阿難,汝又齅此爐中旃檀,此香若復然於一銖,室羅筏城四十里內,同時聞氣。於意云何?此香為復生旃檀木,生於汝鼻,為生於空?阿難,若復此香生於汝鼻,稱鼻所生,當從鼻出。鼻非旃檀,云何鼻中有旃檀氣?稱汝聞香,當於鼻入,鼻中出香,說聞非義?若生於空,空性常恆,香應常在;何藉爐中,爇此枯木?若生於木,則此香質因爇成煙;若鼻得聞,合蒙煙氣。其煙騰空,未及遙遠,四十里內云何已聞?是故當知,香鼻與聞俱無處所。卽齅與香二處虛妄,本非因緣,非自然性。

阿難,汝常二時眾中持缽,其間或遇酥、酪、醍醐,名為上味。於意云何?此味為復生於空中?生於舌中?為生食中?阿難,若復此味生於汝舌,在汝口中只有一舌;其舌爾時已成酥味,遇黑石蜜,應不推移?若不變移,不名知味;若變移者,舌非多體。云何多味,一舌之知?若生於食,食非有識,云何知味?又食自知,卽同他食,何預於汝名味之知?若生於空,汝噉虛空,當作何味?必其虛空若作鹹味;旣鹹汝舌,亦鹹汝面?則此界人,同於海魚,旣常受鹹,了不知淡;若不識淡,亦不覺鹹,必無所知,云何名味?是故當知,味、舌與嚐俱無處所。卽嚐與味二處虛妄,本非因緣,非自然性。

阿難,汝常晨朝以手摩頭。於意云何?此摩所知,誰為能觸?能為在手?為復在頭?若在於手,頭則無知,云何成觸?若在於頭,手則無用,云何名觸?若各各有,則汝阿難,應有二身?若頭與手,一觸所生,則手與頭當為一體。若一體者,觸則無成。若二體者,觸誰為在?在能非所?在所非能?不應虛空與汝成觸?是故當知,覺觸與身俱無處所。卽身與觸二處虛妄,本非因緣,非自然性。

阿難,汝常意中所緣,善、惡、無記三性,生成法則。此法為復卽心所生?為當離心別有方所?阿難,若卽心者,法則非塵,非心所緣,云何成處?若離於心別有方所,則法自性,為知非知?知則名心,異汝非塵,同他心量;卽汝卽心,云何汝心更二於汝?若非知者,此塵旣非色、聲、香、味、離、合、冷、暖,及虛空相,當於何在?今於色、空都無表示,不應人間更有空外;心非所緣,處從誰立?是故當知,法則與心俱無處所。則意與法二俱虛妄,本非因緣,非自然性。

復次,阿難,云何十八界,本如來藏,妙眞如性?

阿難,如汝所明,眼、色為緣,生於眼識。此識為復因眼所生,以眼為界?因色所生,以色為界?阿難,若因眼生,旣無色空,無可分別;縱有汝識,欲將何用?汝見又非青、黃、赤、白,無所表示,從何立界?若因色生,空無色時,汝識應滅?云何識知是虛空性?若色變時,汝亦識其色相遷變,汝識不遷,界從何立?從變則變,界相自無;不變則恆,旣從色生,應不識知虛空所在?若兼二種眼、色共生,合則中離,離則兩合,體性雜亂,云何成界?是故當知眼色為緣,生眼識界,三處都無。則眼與色及色界三,本非因緣,非自然性。

阿難,又汝所明,耳、聲為緣,生於耳識。此識為復,因耳所生,以耳為界?因聲所生,以聲為界?阿難,若因耳生,動靜二相,旣不現前,根不成知,必無所知,知尚無成,識何形貌?若取耳聞,無動靜故,聞無所成;云何耳形,雜色觸塵,名為識界?則耳識界,復從誰立?若生於聲,識因聲有,則不關聞。無聞則亡聲相所在?識從聲生,許聲因聞,而有聲相;聞應聞識;不聞非界;聞則同聲;識已被聞,誰知聞識?若無知者,終如草木。不應聲聞,雜成中界。界無中位,則內外相,復從何成?是故當知耳聲為緣,生耳識界,三處都無。則耳與聲及聲界三,本非因緣,非自然性。

阿難,又汝所明,鼻、香為緣,生於鼻識。此識為復因鼻所生,以鼻為界?因香所生,以香為界?阿難,若因鼻生,則汝心中以何為鼻?為取肉形雙爪之相?為取齅知動摇之性?若取肉形,肉質乃身,身知卽觸。名身非鼻,名觸卽塵;鼻尚無名,云何立界?若取齅知,又汝心中,以何為知?以肉為知,則肉之知,元觸非鼻。以空為知,空則自知,肉應非覺?如是則應虛空是汝。汝身非知,今日阿難應無所在。以香為知,知自屬香,何預於汝?若香臭氣必生汝鼻,則彼香、臭二種流氣,不生伊蘭及旃檀木?二物不來,汝自嗅鼻為香為臭?臭則非香,香應非臭;若香、臭二俱能聞者,則汝一人應有兩鼻?對我問道,有二阿難,誰為汝體?若鼻是一,香臭無二;臭旣為香,香復成臭,二性不有,界從誰立?若因香生,識因香有。如眼有見,不能觀眼;因香有故,應不知香?知卽非生,不知非識。香非知有,香界不成;識不知香,因界則非從香建立。旣無中間,不成內外;彼諸聞性,畢竟虛妄。是故當知鼻香為緣,生鼻識界,三處都無。則鼻與香及香界三,本非因緣,非自然性。

阿難,又汝所明,舌、味為緣,生於舌識。此識為復因舌所生,以舌為界?因味所生,以味為界?阿難,若因舌生,則諸世間甘蔗、烏梅、黃連、石鹽、細辛、薑、桂,都無有味。汝自嚐舌,為甜為苦?若舌性苦,誰來嚐舌?舌不自嚐,孰為知覺?舌性非苦,味自不生,云何立界?若因味生,識自為味,同於舌根,應不自嚐;云何識知是味非味?又一切味,非一物生,味旣多生,識應多體?識體若一,體必味生,鹹、淡、甘、辛,和合俱生,諸變異相,同為一味,應無分別?分別旣無,則不名識;云何復名舌味識界?不應虛空生汝心識?舌味和合,卽於是中,元無自性,云何界生?是故當知舌味為緣,生舌識界,三處都無。則舌與味卽舌界三,本非因緣,非自然性。

阿難,又汝所明,身、觸為緣,生於身識。此識為復因身所生,以身為界?因觸所生,以觸為界?阿難,若因身生,必無合、離二覺觀緣,身何所識?若因觸生,必無汝身,誰有非身知合離者?阿難,物不觸知,身知有觸。知身卽觸,知觸卽身;卽觸非身,卽身非觸。身觸二相,元無處所。合身,卽為身自體性;離身,卽是虛空等相。內外不成,中云何立?中不復立,內外性空。則汝識生,從誰立界?是故當知身觸為緣,生身識界,三處都無。則身與觸及身界三,本非因緣,非自然性。

阿難,又汝所明,意、法為緣,生於意識。此識為復因意所生,以意為界?因法所生,以法為界?阿難,若因意生,於汝意中,必有所思,發明汝意。若無前法,意無所生;離緣無形,識將何用?又汝識心,與諸思量,兼了別性,為同為異?同意卽意,云何所生?異意不同,應無所識。若無所識,云何意生?若有所識,云何識意?唯同與異,二性無成,界云何立?若因法生,世間諸法,不離五塵。汝觀色法,及諸聲法、香法、味法,及與觸法,相狀分明,以對五根,非意所攝。汝識決定依於法生,汝今諦觀,法法何狀?若離色空,動靜、通塞、合離、生滅,越此諸相,終無所得。生,則色空諸法等生;滅,則色空諸法等滅。所因旣無,因生有識,作何形相?相狀不有,界云何生?是故當知意法為緣,生意識界,三處都無。則意與法及意界三,本非因緣,非自然性。

阿難白佛言:世尊,如來常說,和合因緣,一切世間種種變化,皆因四大和合發明。云何如來因緣、自然二俱排擯?我今不知斯義所屬,惟垂哀愍,開示眾生中道了義,無戲論法。

爾時,世尊告阿難言:汝先厭離聲聞、緣覺、諸小乘法,發心勤求無上菩提,故我今時,為汝開示第一義諦。如何復將世間戲論,妄想因緣,而自纏繞,汝雖多聞,如說藥人,眞藥現前,不能分別,如來說為眞可憐愍。汝今諦聽,吾當為汝分別開示;亦令當來修大乘者,通達實相。

阿難默然,承佛聖旨。  

阿難,如汝所言:四大和合,發明世間種種變化。阿難,若彼大性,體非和合,則不能與諸大雜和,猶如虛空不和諸色。若和合者,同於變化,始終相成,生滅相續;生死死生,生生死死,如旋火輪,未有休息。阿難,如水成冰,冰還成水。

汝觀地性,麤為大地,細為微塵,至鄰虛塵。析彼極微,色邊際相,七分所成,更析鄰虛,卽實空性。阿難,若此鄰虛析成虛空,當知虛空出生色相。汝今問言:由和合故,出生世間諸變化相。汝且觀此一鄰虛塵,用幾虛空和合而有?不應鄰虛合成鄰虛。又鄰虛塵析入空者,用幾色相合成虛空?若色合時,合色非空;若空合時,合空非色;色猶可析,空云何合?汝元不知如來藏中,性色眞空,性空眞色,清淨本然,周徧法界;隨眾生心,應所知量。循業發現。世間無知,惑為因緣及自然性,皆是識心分別計度,但有言說,都無實義。

阿難,火性無我,寄於諸緣。汝觀城中未食之家,欲炊爨時,手執陽燧,日前求火。阿難,名和合者,如我與汝,一千二百五十比丘,今為一眾;眾雖為一,詰其根本,各各有身,皆有所生氏族名字。如舍利弗,婆羅門種;優樓頻螺,迦葉波種;乃至阿難,瞿曇種姓。阿難,若此火性,因和合有。彼手執鏡,於日求火,此火為從鏡中而出?為從艾出?為於日來?阿難,若日來者,自能燒汝手中之艾,來處林木,皆應受焚?若鏡中出,自能於鏡,出然於艾,鏡何不鎔?紆汝手執,尚無熱相,云何融泮?若生於艾,何藉日鏡光明相接,然後火生?汝又諦觀:鏡因手執,日從天來,艾本地生,火從何方,遊歷於此?日、鏡相遠,非和、非合,不應火光無從自有?汝猶不知如來藏中,性火眞空,性空眞火,清淨本然,周徧法界。隨眾生心,應所知量。阿難,當知世人,一處執鏡,一處火生。徧法界執,滿世間起。起徧世間,寧有方所。循業發現。世間無知,惑為因緣及自然性,皆是識心分別計度,但有言說,都無實義。

阿難,水性不定,流息無恆。如室羅城,迦毗羅仙、斫迦羅仙、及缽頭摩、訶薩多等諸大幻師,求太陰精,用和幻藥;是諸師等,於白月晝,手執方諸,承月中水。此水為復從珠中出?空中自有?為從月來?阿難,若從月來,尚能遠方令珠出水,所經林木皆應吐流。流,則何待方諸所出?不流,明水非從月降。若從珠出,則此珠中常應流水,何待中宵承白月晝?若從空生,空性無邊,水當無際,從人洎天,皆同滔溺。云何復有水、陸、空行?汝更諦觀:月從天陟,珠因手持,承珠水盤,本人敷設,水從何方流注於此?月、珠相遠,非和、非合;不應水精,無從自有。汝尚不知如來藏中,性水眞空,性空眞水,清淨本然,周徧法界。隨眾生心,應所知量。一處執珠,一處水出;徧法界執,滿法界生,生滿世間,寧有方所?循業發現。世間無知,惑為因緣及自然性,皆是識心分別計度,但有言說,都無實義。

阿難,風性無體,動、靜不常。汝常整衣入於大眾,僧伽梨角,動及傍人,則有微風拂彼人面。此風為復出袈裟角?發於虛空?生彼人面?阿難,此風若復出袈裟角,汝乃披風,其衣飛摇,應離汝體。我今說法,會中垂衣,汝看我衣,風何所在?不應衣中有藏風地?若生虛空,汝衣不動,何因無拂?空性常住,風應常生,若無風時,虛空當滅;滅風可見,滅空何狀?若有生滅,不名虛空;名為虛空,云何風出?若風自生,被拂之面;從彼面生,當應拂汝?自汝整衣,云何倒拂?汝審諦觀:整衣在汝,面屬彼人。虛空寂然,不參流動;風自誰方,鼓動來此?風、空性隔,非和、非合;不應風性無從自有?汝宛不知如來藏中,性風眞空,性空眞風,清淨本然,周徧法界。隨眾生心,應所知量。阿難,如汝一人,微動服衣,有微風出;徧法界拂,滿國土生,周徧世間,寧有方所?循業發現。世間無知,惑為因緣及自然性,皆是識心分別計度,但有言說,都無實義。

阿難,空性無形,因色顯發:如室羅城去河遙處,諸剎利種,及婆羅門,毗舍、首陀,兼頗羅墮、旃陀羅等。新立安居,鑿井求水,出土一尺,於中則有一尺虛空,如是乃至出土一丈,中間還得一丈虛空,虛空淺深,隨出多少。此空為當因土所出?因鑿所有?無因自生?阿難,若復此空無因自生,未鑿土前,何不無礙?唯見大地,迥無通達。若因土出,則土出時,應見空入,若土先出,無空入者,云何虛空,因土而出?若無出入,則應空土元無異因;無異則同,則土出時,空何不出?若因鑿出,則鑿出空,應非出土?不因鑿出,鑿自出土,云何見空?汝更審諦,諦審諦觀:鑿從人手,隨方運轉,土因地移,如是虛空因何所出?鑿、空虛實,不相為用,非和、非合,不應虛空無從自出。若此虛空,性圓周徧,本不動摇,當知現前地、水、火、風,均名五大,性眞圓融,皆如來藏,本無生滅。阿難,汝心昏迷,不悟四大元如來藏,當觀虛空為出?為入?為非出入?汝全不知如來藏中,性覺眞空,性空眞覺,清淨本然,周徧法界。隨眾生心,應所知量。阿難,如一井空,空生一井;十方虛空亦復如是,圓滿十方,寧有方所。循業發現。世間無知,惑為因緣及自然性,皆是識心分別計度,但有言說,都無實義。

阿難,見覺無知,因色空有。如汝今者在祇陀林,朝明夕昏;設居中宵,白月則光,黑月便暗,則明暗等,因見分析。此見為復與明暗相,並太虛空,為同一體?為非一體?或同、非同?或異、非異?阿難,此見若復與明與暗,及與虛空,元一體者,則明與暗,二體相亡;暗時無明,明時無暗。若與暗一,明則見亡;必一於明,暗時當滅。滅則云何見明見暗?若明暗殊,見無生滅,一云何成?若此見精與暗與明,非一體者,汝離明暗,及與虛空,分析見元,作何形相?離明離暗,及離虛空,是見元同龜毛、兔角;明、暗、虛空,三事俱異,從何立見?明暗相背,云何或同?離三元無,云何或異?分空、分見,本無邊畔,云何非同?見暗、見明,性非遷改,云何非異?汝更細審,微細審詳,審諦審觀:明從太陽,暗隨黑月,通屬虛空,壅歸大地,如是見精,因何所出?見覺空頑,非和非合;不應見精,無從自出?若見、聞、知,性圓周徧,本不動摇,當知無邊不動虛空,併其動摇地、水、火、風,均名六大。性眞圓融,皆如來藏,本無生滅。阿難,汝性沈淪,不悟汝之見、聞、覺、知,本如來藏。汝當觀此見、聞、覺、知,為生為滅?為同為異?為非生滅?為非同異?汝曾不知如來藏中,性見覺明,覺精明見,清淨本然,周徧法界。隨眾生心,應所知量。如一見根,見周法界;聽、嗅、嚐觸、覺觸、覺知,妙德瑩然,徧周法界。圓滿十虛,寧有方所?循業發現。世間無知,惑為因緣及自然性,皆是識心分別計度,但有言說,都無實義。

阿難,識性無源,因於六種根塵妄出。汝今徧觀此會聖眾,用目循歷。其目周視,但如鏡中,無別分析。汝識於中,次第標指:此是文殊,此富樓那,此目犍連,此須菩提,此舍利弗。此識了知,為生於見?為生於相?為生虛空?為無所因,突然而出?阿難,若汝識性生於見中,如無明暗及與色空,四種必無。元無汝見,見性尚無,從何發識?若汝識性,生於相中,不從見生?旣不見明,亦不見暗,明暗不矚,卽無色空,彼相尚無,識從何發?若生於空,非相非見。非見無辨,自不能知明暗色空?非相滅緣,見、聞、覺、知,無處安立?處此二非,空則同無,有非同物,縱發汝識,欲何分別?若無所因,突然而出,何不日中,別識明月?汝更細詳,微細詳審:見託汝睛,相推前境,可狀成有,不相成無,如是識緣,因何所出?識動、見澄,非和、非合,聞、聽、覺、知,亦復如是;不應識緣,無從自出?若此識心,本無所從。當知了別見、聞、覺、知,圓滿湛然,性非從所,兼彼虛空、地、水、火、風,均名七大。性眞圓融,皆如來藏,本無生滅。阿難,汝心麤浮,不悟見聞,發明了知,本如來藏。汝應觀此六處識心,為同為異?為空為有?為非同異?為非空有?汝元不知如來藏中,性識明知,覺明眞識,妙覺湛然,徧周法界。含吐十虛,寧有方所?循業發現。世間無知,惑為因緣及自然性,皆是識心分別計度,但有言說,都無實義。

爾時,阿難及諸大眾,蒙佛如來微妙開示,身心蕩然,得無罣礙。是諸大眾,各各自知心徧十方,見十方空,如觀手中所持葉物。一切世間諸所有物,皆卽菩提妙明元心。心精徧圓,含裹十方。反觀父母所生之身,猶彼十方虛空之中,吹一微塵,若存若亡。如湛巨海,流一浮漚,起滅無從。了然自知,獲本妙心,常住不滅。禮佛合掌,得未曾有。於如來前,說偈讚佛:

~\\[-22mm]

\begin{JISONG}
妙湛總持不動尊,首楞嚴王世希有。

銷我億劫顚倒想,不歷僧祇獲法身。

願今得果成寶王,還度如是恆沙眾。

將此深心奉塵剎,是則名為報佛恩。

伏請世尊為證明,五濁惡世誓先入。

如一眾生未成佛,終不於此取泥洹。

大雄大力大慈悲,希更審除微細惑。

令我早登無上覺,於十方界坐道場。

舜若多性可銷亡,爍迦羅心無動轉。

\end{JISONG}

%\pagebreak
%\pagestyle{empty}
%\begin{center}
%\setlength{\fboxsep}{1mm} 
%\fbox{\includegraphics[width=\textwidth]{pictures/zs.png}}
%\end{center}
