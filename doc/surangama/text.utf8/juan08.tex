% $Id: juan08.tex,v 1.4 2011/08/09 09:08:36 xiongyw Exp $


阿難,如是眾生一一類中,亦各各具十二顚倒,猶如捏目,亂華發生。顚倒妙圓眞淨明心,具足如斯虛妄亂想。汝今修證佛三摩地,於是本因元所亂想,立三漸次,方得除滅。如淨器中,除去毒蜜,以諸湯水並雜灰香,洗滌其器,後貯甘露。云何名為三種漸次?一者修習,除其助因;二者眞修,刳其正性;三者增進,違其現業。

云何助因?阿難,如是世界十二類生,不能自全,依四食住;所謂段食、觸食、思食、識食。是故佛說,一切眾生,皆依食住。阿難,一切眾生食甘故生,食毒故死;是諸眾生求三摩地,當斷世間五種辛菜。是五種辛,熟食發淫,生噉增恚。如是世界食辛之人,縱能宣說十二部經,十方天仙嫌其臭穢,咸皆遠離;諸餓鬼等,因彼食次,舐其唇吻,常與鬼住。福德日消,長無利益。是食辛人修三摩地,菩薩天仙、十方善神不來守護;大力魔王得其方便,現作佛身來為說法,非毀禁戒,讚婬、怒、癡。命終自為魔王眷屬,受魔福盡,墮無間獄。阿難,修菩提者永斷五辛,是則名為第一增進修行漸次。

云何正性?阿難,如是眾生入三摩地,要先嚴持清淨戒律。永斷婬心,不餐酒肉;以火淨食,無噉生氣。阿難,是修行人若不斷婬及與殺生,出三界者,無有是處。當觀婬欲猶如毒蛇,如見怨賊。先持聲聞四棄八棄,執身不動;後行菩薩清淨律儀,執心不起。禁戒成就,則於世間永無相生、相殺之業;偷劫不行,無相負累,亦於世間不還宿債。是清淨人修三摩地,父母肉身不須天眼,自然觀見十方世界,睹佛聞法,親奉聖旨。得大神通,遊十方界;宿命清淨,得無艱險。是則名為第二增進修行漸次。

云何現業?阿難,如是清淨持禁戒人,心無貪婬,於外六塵,不多流逸;因不流逸,旋元自歸。塵旣不緣,根無所偶;反流全一,六用不行。十方國土,皎然清淨;譬如璢璃,內懸明月。身心快然,妙圓平等,獲大安隱。一切如來密圓淨妙,皆現其中。是人卽獲無生法忍;從是漸修,隨所發行,安立聖位。是則名為第三增進修行漸次。

阿難,是善男子欲愛乾枯,根境不偶,現前殘質不復續生。執心虛明,純是智慧,慧性明圓,鎣十方界。乾有其慧,名乾慧地。欲習初乾,未與如來法流水接。卽以此心中中流入,圓妙開敷,從眞妙圓,重發眞妙,妙信常住;一切妄想滅盡無餘,中道純眞,名信心住。眞信明了,一切圓通,陰、處、界三不能為礙;如是乃至過去未來無數劫中,捨身受身一切習氣皆現在前。是善男子皆能憶念得無遺忘,名念心住。妙圓純眞,眞精發化,無始習氣通一精明,唯以精明進趣眞淨,名精進心。心精現前,純以智慧,名慧心住。執持智明,周徧寂湛,寂妙常凝,名定心住。定光發明,明性深入,唯進無退,名不退心住。心進安然,保持不失,十方如來氣分交接,名護法心。覺明保持,能以妙力迴佛慈光,向佛安住;猶如雙鏡光明相對,其中妙影重重相入,名迴向心。心光密迴,護佛常凝無上妙淨,安住無為,得無遺失,名戒心住。住戒自在,能遊十方,所去隨願,名願心住。

阿難,是善男子以眞方便發此十心,心精發輝,十用涉入,圓成一心,名發心住。心中發明如淨璢璃,內現精金;以前妙心,屨以成地,名治地住。心地涉知,俱得明了,遊履十方,得無留礙,名修行住。行與佛同,受佛氣分,如中陰身自求父母,陰信冥通,入如來種,名生貴住。旣遊道胎,親奉覺胤,如胎已成,人相不缺,名方便具足住。容貌如佛,心相亦同,名正心住。身心合成,日益增長,名不退住。十身靈相,一時具足,名童眞住。形成出胎,親為佛子,名法王子住。表以成人,如國大王以諸國事分委太子,彼剎利王世子長成,陳列灌頂,名灌頂住。

阿難,是善男子成佛子已,具足無量如來妙德,十方隨順,名歡喜行。善能利益一切眾生,名饒益行。自覺覺他,得無違拒,名無瞋恨行。種類出生窮未來際,三世平等,十方通達,名無盡行。一切合同種種法門,得無差誤,名離癡亂行。則於同中顯現群異,一一異相各各見同,名善現行。如是乃至十方虛空滿諸微塵,一一塵中現十方界,現塵現界,不相留礙,名無著行。種種現前,咸是第一波羅蜜多,名尊重行。如是圓融,能成十方諸佛軌則,名善法行。一一皆是清淨無漏,一眞無為,性本然故,名眞實行。

阿難,是善男子滿足神通成佛事已,純潔精眞,遠諸留患。當度眾生滅除度相,迴無為心,向涅槃路,名救一切眾生離眾生相迴向。壞其可壞,遠離諸離,名不壞迴向。本覺湛然,覺齊佛覺,名等一切佛迴向。精眞發明,地如佛地,名至一切處迴向。世界如來互相涉入,得無罣礙,名無盡功德藏迴向。於同佛地,地中各各生清淨因;依因發輝,取涅槃道,名隨順平等善根迴向。眞根旣成,十方眾生皆我本性;性圓成就,不失眾生,名隨順等觀一切眾生迴向。卽一切法,離一切相,唯卽與離,二無所著,名眞如相迴向。眞得所如,十方無礙,名無縛解脫迴向。性德圓成,法界量滅,名法界無量迴向。

阿難,是善男子,盡是清淨四十一心,次成四種妙圓加行。卽以佛覺用為己心,若出未出,猶如鑽火欲然其木,名為煖地。又以己心,成佛所履,若依非依,如登高山,身入虛空,下有微礙,名為頂地。心佛二同,善得中道,如忍事人,非懷非出,名為忍地。數量銷滅,迷、覺中道,二無所目,名世第一地。

阿難,是善男子,於大菩提善得通達,覺通如來,盡佛境界,名歡喜地。異性入同,同性亦滅,名離垢地。淨極明生,名發光地。明極覺滿,名燄慧地。一切同異所不能至,名難勝地。無為眞如,性淨明露,名現前地。盡眞如際,名遠行地。一眞如心,名不動地。發眞如用,名善慧地。

阿難,是諸菩薩從此已往,修習畢功,功德圓滿,亦目此地,名修習位。慈陰妙雲,覆涅槃海,名法雲地。如來逆流,如是菩薩順行而至,覺際入交,名為等覺。

阿難,從乾慧心至等覺已,是覺始獲金剛心中初乾慧地。如是重重單複十二,方盡妙覺,成無上道。是種種地,皆以金剛觀察如幻十種深喻;奢摩他中,用諸如來毗婆舍那清淨修證,漸次深入。

阿難,如是皆以三增進故,善能成就五十五位眞菩提路。作是觀者,名為正觀;若他觀者,名為邪觀。

爾時,文殊師利法王子在大眾中,卽從座起,頂禮佛足而白佛言:當何名是經?我及眾生云何奉持?

佛告文殊師利,是經名「大佛頂悉怛多般怛囉無上寶印,十方如來清淨海眼」。亦名「救護親因,度脫阿難,及此會中性比丘尼,得菩提心,入徧知海」。亦名「如來密因修證了義」。亦名「大方廣妙蓮華王,十方佛母陀羅尼呪」。亦名「灌頂章句,諸菩薩萬行首楞嚴」。汝當奉持。

說是語已,卽時,阿難及諸大眾,得蒙如來開示密印般怛囉義,兼聞此經了義名目。頓悟禪那,修進聖位,增上妙理,心慮虛凝。斷除三界修心六品微細煩惱。卽從座起,頂禮佛足,合掌恭敬而白佛言:大威德世尊,慈音無遮,善開眾生微細沈惑,令我今日身心快然,得大饒益。世尊,若此妙明眞淨妙心本來徧圓,如是乃至大地草木、蝡動含靈,本元眞如,卽是如來成佛眞體。佛體眞實,云何復有地獄、餓鬼、畜生、脩羅、人、天等道?世尊,此道為復本來自有?為是眾生妄習生起?世尊,如寶蓮香比丘尼持菩薩戒,私行婬欲,妄言行婬非殺非偷,無有業報。發是語已,先於女根生大猛火,後於節節猛火燒然,墮無間獄。璢璃大王、善星比丘,璢璃為誅瞿曇族姓,善星妄說一切法空,生身陷入阿鼻地獄。此諸地獄為有定處?為復自然彼彼發業各各私受?惟垂大慈開發童蒙,令諸一切持戒眾生,聞決定義,歡喜頂戴,謹潔無犯。

佛告阿難,快哉此問!令諸眾生不入邪見,汝今諦聽,當為汝說。阿難,一切眾生實本眞淨,因彼妄見,有妄習生,因此分開內分、外分。

阿難,內分卽是眾生分內。因諸愛染,發起妄情;情積不休,能生愛水。是故眾生心憶珍饈,口中水出;心憶前人,或憐或恨,目中淚盈;貪求財寶,心發愛涎,舉體光潤;心著行婬,男女二根自然流液。阿難,諸愛雖別,流結是同;潤濕不升,自然從墜,此名內分。

阿難,外分卽是眾生分外。因諸渴仰,發明虛想;想積不休,能生勝氣。是故眾生心持禁戒,舉身輕清;心持呪印,顧盼雄毅;心欲生天,夢想飛舉;心存佛國,聖境冥現;事善知識,自輕身命。阿難,諸想雖別,輕舉是同;飛動不沈,自然超越,此名外分。

阿難,一切世間生死相續,生從順習,死從變流。臨命終時,未捨煖觸,一生善惡俱時頓現;死逆生順,二習相交。純想卽飛,必生天上。若飛心中兼福兼慧及與淨願,自然心開,見十方佛,一切淨土,隨願往生。

情少想多,輕舉非遠,卽為飛仙、大力鬼王、飛行夜叉、地行羅剎,遊於四天,所去無礙。

其中若有善願善心,護持我法;或護禁戒,隨持戒人;或護神呪,隨持呪者;或護禪定,保綏法忍;是等親住如來座下。

情想均等,不飛不墜,生於人間;想明斯聰,情幽斯鈍。

情多想少,流入橫生;重為毛群,輕為羽族。

七情三想,沈下水輪;生於火際,受氣猛火。身為餓鬼常被焚燒,水能害己,無食無飲經百千劫。

九情一想,下洞火輪;身入風火,二交過地。輕生有間,重生無間,二種地獄。

純情,卽沈入阿鼻獄。

若沈心中,有謗大乘、毀佛禁戒、誑妄說法、虛貪信施、濫膺恭敬、五逆十重,更生十方阿鼻地獄。

循造惡業,雖則自招;眾同分中,兼有元地。

阿難,此等皆是彼諸眾生自業所感,造十習因,受六交報。

云何十因?

阿難,一者,婬習交接,發於相磨,研磨不休。如是故有大猛火光,於中發動;如人以手自相摩觸,煖相現前。二習相然,故有鐵床、銅柱諸事。是故十方一切如來,色目行婬,同名欲火;菩薩見欲,如避火坑。

二者,貪習交計,發於相吸,吸攬不止。如是故有積寒堅冰,於中凍冽;如人以口吸縮風氣,有冷觸生。二習相陵,故有吒吒、波波、羅羅、青赤白蓮、寒冰等事。是故十方一切如來,色目多求,同名貪水;菩薩見貪,如避瘴海。

三者,慢習交陵,發於相恃,馳流不息。如是故有騰逸奔波,積波為水;如人口舌自相緜味,因而水發。二習相鼓,故有血河、灰河、熱砂、毒海、融銅、灌吞諸事。是故十方一切如來,色目我慢,名飲癡水;菩薩見慢,如避巨溺。

四者,瞋習交衝,發於相忤;忤結不息,心熱發火,鑄氣為金。如是故有刀山、鐵橛、劍樹、劍輪、斧、鉞、鎗、鋸;如人銜冤,殺氣飛動。二習相擊,故有宮割、斬斫、剉、刺、槌、擊諸事。是故十方一切如來,色目瞋恚,名利刀劍;菩薩見瞋,如避誅戮。

五者,詐習交誘,發於相調,引起不住。如是故有繩木、絞校;如水浸田,草木生長。二習相延,故有杻、械、枷、鎖、鞭、杖、檛、棒諸事。是故十方一切如來,色目奸偽,同名讒賊;菩薩見詐,如畏豺狼。

六者,誑習交欺,發於相罔;誣罔不止,飛心造奸。如是故有塵土、屎、尿,穢污不淨;如塵隨風,各無所見。二習相加,故有沒溺、騰擲、飛墬、漂淪諸事。是故十方一切如來,色目欺誑,同名劫殺;菩薩見誑,如踐蛇虺。

七者,怨習交嫌,發於銜恨。如是故有飛石、投礫、匣貯、車檻、甕盛、囊撲;如陰毒人,懷抱畜惡。二習相吞,故有投擲、擒捉、擊射、拋撮諸事。是故十方一切如來,色目怨家,名違害鬼;菩薩見怨,如飲鴆酒。

八者,見習交明,如薩迦耶見戒禁取,邪悟諸業;發於違拒,出生相反。如是故有王使、主吏、證執、文籍;如行路人,來往相見。二習相交,故有勘問、權詐、考訊、推鞠、察訪、披究、照明;善惡童子手執文簿,辭辯諸事。是故十方一切如來,色目惡見,同名見坑;菩薩見諸虛妄偏執,如臨毒壑。

九者,枉習交加,發於誣謗。如是故有合山、合石、碾、磑、耕、磨;如讒賊人,逼枉良善。二習相排,故有押、捺、搥、按、蹙漉、衡度諸事。是故十方一切如來,色目怨謗,同名讒虎;菩薩見枉,如遭霹靂。

十者,訟習交諠,發於藏覆。如是故有鑑見、照燭;如於日中,不能藏影。二習相陳,故有惡友、業鏡、火珠,披露宿業,對驗諸事。是故十方一切如來,色目覆藏,同名陰賊;菩薩觀覆,如戴高山,履於巨海。

云何六報?阿難,一切眾生六識造業,所招惡報從六根出。

云何惡報從六根出?

一者見報,招引惡果。此見業交,則臨終時,先見猛火滿十方界;亡者神識飛墜乘煙,入無間獄。發明二相。一者明見,則能徧見種種惡物,生無量畏。二者暗見,寂然不見,生無量恐。如是見火。燒聽,能為鑊湯、烊銅。燒息,能為黑煙、紫焰。燒味,能為焦丸、鐵糜。燒觸,能為熱灰、爐炭。燒心,能生星火迸灑,煽鼓空界。

二者聞報,招引惡果。此聞業交,則臨終時,先見波濤沒溺天地;亡者神識降注乘流,入無間獄。發明二相。一者開聽,聽種種鬧,精神愗亂。二者閉聽,寂無所聞,幽魄沈沒。如是聞波。注聞,則能為責、為詰。注見,則能為雷、為吼、為惡毒氣。注息,則能為雨、為霧,灑諸毒蟲,周滿身體。注味,則能為膿、為血,種種雜穢。注觸,則能為畜、為鬼、為糞、為尿。注意,則能為電、為雹,摧碎心魄。

三者齅報,招引惡果。此齅業交,則臨終時,先見毒氣充塞遠近;亡者神識從地涌出,入無間獄。發明二相。一者通聞,被諸惡氣,熏極心擾。二者塞聞,氣掩不通,悶絕於地。如是齅氣。衝息,則能為質、為履。衝見,則能為火、為炬。衝聽,則能為沒、為溺、為烊、為沸。衝味,則能為餒、為爽。衝觸,則能為綻、為爛、為大肉山,有百千眼無量𠯗食。衝思,則能為灰、為瘴、為飛沙礰,擊碎身體。

四者味報,招引惡果。此味業交,則臨終時,先見鐵網猛燄熾烈,周覆世界;亡者神識,下透掛網,倒懸其頭,入無間獄。發明二相。一者吸氣,結成寒冰,凍冽肉身。二者吐氣,飛為猛火,焦爛骨髓。如是嚐味。歷嚐,則能為承、為忍。歷見,則能為然金石。歷聽,則能為利兵刃。歷息,則能為大鐵籠,彌覆國土。歷觸,則能為弓、為箭、為弩、為射。歷思,則能為飛熱鐵,從空雨下。

五者觸報招引惡果。此觸業交,則臨終時,先見大山四面來合,無復出路。亡者神識見大鐵城,火蛇、火狗、虎、狼、獅子、牛頭獄卒、馬頭羅剎,手執鎗矟,驅入城門,向無間獄。發明二相。一者合觸,合山逼體,骨肉血潰。二者離觸,刀劍觸身,心肝屠裂。如是合觸。歷觸,則能為撞、為擊、為剚、為射。歷見,則能為燒、為爇。歷聽,則能為為道、為觀、為廳、為案。歷息,則能為括、為袋、為考、為縛。歷嚐,則能為耕、為鉗、為斬、為截。歷思,則能為墜、為飛、為煎、為炙。

六者思報,招引惡果。此思業交,則臨終時,先見惡風吹壞國土;亡者神識被吹上空,旋落乘風,墜無間獄。發明二相。一者不覺,迷極則荒,奔赴不息。二者不迷,覺知則苦,無量煎燒,痛深難忍。如是邪思。結思,則能為方、為所。結見,則能為鑑、為證。結聽,則能為大合石、為冰、為霜、為土、為霧。結息,則能為大火車、火船、火檻。結嚐,則能為大叫喚、為悔、為泣。結觸,則能為大、為小、為一日中萬生萬死、為偃、為仰。

阿難,是名地獄十因六果,皆是眾生迷妄所造。若諸眾生惡業同造,入阿鼻獄,受無量苦,經無量劫。六根各造,及彼所作兼境兼根,是人則入八無間獄。身口意三作殺盜婬,是人則入十八地獄。三業不兼,中間或為一殺一盜,是人則入三十六地獄。見見一根,單犯一業,是人則入一百八地獄。由是眾生別作別造,於世界中入同分地,妄想發生,非本來有。

復次,阿難,是諸眾生非破律儀,犯菩薩戒,毀佛涅槃,諸餘雜業,歷劫燒然;後還罪畢,受諸鬼形。

若於本因,貪物為罪;是人罪畢,遇物成形,名為怪鬼。

貪色為罪;是人罪畢,遇風成形,名為魃鬼。

貪惑為罪;是人罪畢,遇畜成形,名為魅鬼。

貪恨為罪;是人罪畢,遇蟲成形,名蠱毒鬼。

貪憶為罪;是人罪畢,遇衰成形,名為癘鬼。

貪傲為罪;是人罪畢,遇氣成形,名為餓鬼。

貪罔為罪;是人罪畢,遇幽為形,名為魘鬼。

貪明為罪;是人罪畢,遇精為形,名魍魎鬼。

貪成為罪;是人罪畢,遇明為形,名役使鬼。

貪黨為罪;是人罪畢,遇人為形,名傳送鬼。

阿難,是人皆以純情墜落,業火燒乾,上出為鬼。此等皆是自妄想業之所招引,若悟菩提,則妙圓明本無所有。

復次,阿難。鬼業旣盡,則情與想二俱成空。方於世間與元負人冤對相值,身為畜生,酬其宿債。

物怪之鬼,物銷報盡,生於世間多為梟類。

風魃之鬼,風銷報盡,生於世間多為咎徵一切異類。

畜魅之鬼,畜死報盡,生於世間多為狐類。

蟲蠱之鬼,蠱滅報盡,生於世間多為毒類。

衰癘之鬼,衰窮報盡,生於世間多為蛔類。

受氣之鬼,氣銷報盡,生於世間多為食類。

緜幽之鬼,幽銷報盡,生於世間多為服類。

和精之鬼,和銷報盡,生於世間多為應類。

明靈之鬼,明滅報盡,生於世間多為休徵一切諸類。

依人之鬼,人亡報盡,生於世間多為循類。

阿難,是等皆以業火乾枯,酬其宿債,旁為畜生。此等亦皆自虛妄業之所招引,若悟菩提,則此妄緣本無所有。如汝所言,寶蓮香等,及璢璃王、善星比丘,如是惡業,本自發明,非從天降,亦非地出,亦非人與;自妄所招,還自來受。菩提心中,皆為浮虛妄想凝結。

復次,阿難。從是畜生酬償先債。若彼酬者,分越所酬;此等眾生還復為人,反徵其剩。如彼有力,兼有福德,則於人中不捨人身,酬還彼力。若無福者,還為畜生,償彼餘直。阿難當知,若用錢物,或役其力,償足自停。如其中間殺彼身命,或食其肉,如是乃至經微塵劫相食相誅。猶如轉輪,互為高下,無有休息。除奢摩他,及佛出世,不可停寢。

汝今應知,彼梟倫者,酬足復形,生人道中,參合頑類。

彼咎徵者,酬足復形,生人道中,參合異類。

彼狐倫者,酬足復形,生人道中,參於庸類。

彼毒倫者,酬足復形,生人道中,參合狠類。

彼蛔倫者,酬足復形,生人道中,參合微類。

彼食倫者,酬足復形,生人道中,參合柔類。

彼服倫者,酬足復形,生人道中,參合勞類。

彼應倫者,酬足復形,生人道中,參於文類。

彼休徵者,酬足復形,生人道中,參合明類。

彼諸循倫,酬足復形,生人道中,參於達類。

阿難,是等皆以宿債酬畢,復形人道;皆無始來業計顚倒,相生相殺。不遇如來、不聞正法,於塵勞中,法爾輪轉。此輩名為可憐愍者。

阿難,復有從人不依正覺修三摩地,別修妄念,存想固形。遊於山林人不及處,有十種仙。

阿難,彼諸眾生,堅固服餌而不休息,食道圓成,名地行仙。

堅固草木而不休息,藥道圓成,名飛行仙。

堅固金石而不休息,化道圓成,名遊行仙。

堅固動止而不休息,氣精圓成,名空行仙。

堅固津液而不休息,潤德圓成,名天行仙。

堅固精色而不休息,吸粹圓成,名通行仙。

堅固呪禁而不休息,術法圓成,名道行仙。

堅固思念而不休息,思憶圓成,名照行仙。

堅固交遘而不休息,感應圓成,名精行仙。

堅固變化而不休息,覺悟圓成,名絕行仙。

阿難,是等皆於人中煉心,不修正覺,別得生理,壽千萬歲;休止深山,或大海島,絕於人境。斯亦輪迴妄想流轉,不修三昧,報盡還來,散入諸趣。

阿難,諸世間人,不求常住,未能捨諸妻妾恩愛。於邪婬中,心不流逸,澄瑩生明。命終之後,鄰於日月。如是一類,名四天王天。

於己妻房,婬愛微薄;於淨居時,不得全味。命終之後,超日月明,居人間頂。如是一類,名忉利天。

逢欲暫交,去無思憶;於人間世,動少靜多。命終之後,於虛空中朗然安住,日月光明上照不及,是諸人等自有光明。如是一類,名須燄摩天。

一切時靜,有應觸來,未能違戾。命終之後,上昇精微,不接下界諸人天壞,乃至劫壞,三災不及。如是一類,名兜率陀天。

我無欲心,應汝行事;於橫陳時,味如嚼蠟。命終之後,生越化地。如是一類,名樂變化天。

無世間心,同世行事;於行事交,了然超越。命終之後,徧能出超化無化境。如是一類,名他化自在天。

阿難,如是六天,形雖出動,心跡尚交,自此已還,名為欲界。

%\vspace{1cm}
%\begin{center}
%\setlength{\fboxsep}{1mm} 
%\fbox{%
%\includegraphics[width=.6\textwidth]{pictures/lw.png}
%}
%\end{center}