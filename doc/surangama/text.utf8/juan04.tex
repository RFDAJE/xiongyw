% $Id: juan04.tex,v 1.3 2011/08/09 09:08:36 xiongyw Exp $

爾時,富樓那彌多羅尼子,在大眾中,卽從座起,偏袒右肩,右膝著地,合掌恭敬而白佛言:大威德世尊,善為眾生,敷演如來第一義諦。世尊常推說法人中,我為第一。今聞如來微妙法音,猶如聾人,逾百步外,聆於蚊蚋;本所不見,何況得聞?佛雖宣明,令我除惑。今猶未詳斯義究竟無疑惑地。世尊,如阿難輩,雖則開悟,習漏未除。我等會中,登無漏者,雖盡諸漏;今聞如來所說法音,尚紆疑悔。世尊,若復世間一切根塵,陰、處、界等,皆如來藏清淨本然;云何忽生山河大地諸有為相,次第遷流,終而復始?又如來說:地、水、火、風,本性圓融,周徧法界,湛然常住。世尊,若地性徧,云何容水?水性周徧,火則不生;復云何明,水、火二性俱徧虛空,不相陵滅?世尊,地性障礙,空性虛通,云何二俱周徧法界?而我不知,是義攸往。惟願如來宣流大慈,開我迷雲,及諸大眾。作是語已,五體投地,欽渴如來無上慈誨。

爾時,世尊告富樓那,及諸會中漏盡無學諸阿羅漢:如來今日,普為此會,宣勝義中,眞勝義性。令汝會中定性聲聞,及諸一切未得二空,回向上乘阿羅漢等,皆獲一乘寂滅場地,眞阿練若正修行處。汝今諦聽,當為汝說。

富樓那等,欽佛法音,默然承聽。

佛言:富樓那,如汝所言清淨本然,云何忽生山河大地?汝常不聞如來宣說,性覺妙明,本覺明妙?富樓那言:唯然,世尊。我常聞佛宣說斯義。佛言:汝稱覺明,為復性明,稱名為覺?為覺不明,稱為明覺?富樓那言:若此不明,名為覺者,則無所明?

佛言:若無所明,則無明覺;有所非覺,無所非明。無明又非覺湛明性。性覺必明,妄為明覺。覺非所明,因明立所;所旣妄立,生汝妄能。無同異中,熾然成異;異彼所異,因異立同。同異發明,因此復立無同無異。如是擾亂,相待生勞;勞久發塵,自相渾濁,由是引起塵勞煩惱。起為世界,靜成虛空;虛空為同,世界為異,彼無同異,眞有為法。覺明空昧,相待成摇,故有風輪執持世界。因空生摇,堅明立礙;彼金寶者,明覺立堅,故有金輪保持國土。堅覺寶成,摇明風出,風金相摩,故有火光為變化性。寶明生潤,火光上蒸,故有水輪含十方界。火騰水降,交發立堅,濕為巨海,乾為洲潬。以是義故,彼大海中,火光常起,彼洲潬中,江河常注。水勢劣火,結為高山;是故山石擊則成燄,融則成水。土勢劣水,抽為草木;是故林藪遇燒成土,因絞成水。交妄發生,遞相為種,以是因緣,世界相續。

復次,富樓那,明妄非他,覺明為咎。所妄旣立,明理不踰;以是因緣,聽不出聲,見不超色。色、香、味、觸,六妄成就,由是分開,見、覺、聞、知。同業相纏,合離成化。見明色發,明見想成。異見成憎,同想成愛;流愛為種,納想為胎。交遘發生,吸引同業,故有因緣,生羯羅藍、遏蒲曇等。胎、卵、濕、化,隨其所應。卵惟想生,胎因情有,濕以合感,化以離應。情想合離,更相變易。所有受業,逐其飛沈,以是因緣,眾生相續。

富樓那,想愛同結,愛不能離,則諸世間父、母、子、孫,相生不斷。是等則以欲貪為本。貪愛同滋,貪不能止,則諸世間卵、化、溼、胎,隨力強弱,遞相吞食,是等則以殺貪為本。以人食羊;羊死為人,人死為羊。如是乃至十生之類,死死生生,互來相噉;惡業俱生,窮未來際。是等則以盜貪為本。汝負我命,我還汝債;以是因緣,經百千劫,常在生死。汝愛我心,我憐汝色;以是因緣,經百千劫,常在纏縛。唯殺、盜、婬三為根本,以是因緣,業果相續。

富樓那,如是三種顚倒相續,皆是覺明,明了知性;因了發相,從妄見生。山河大地,諸有為相,次第遷流;因此虛妄,終而復始。

富樓那言:若此妙覺,本妙覺明,與如來心,不增不減。無狀忽生山河大地諸有為相。如來今得妙空明覺,山河大地,有為習漏,何當復生?佛告富樓那:譬如迷人,於一聚落,惑南為北。此迷為復因迷而有?因悟而出?富樓那言:如是迷人,亦不因迷,又不因悟。何以故?迷本無根,云何因迷?悟非生迷,云何因悟?佛言:彼之迷人,正在迷時;倏有悟人,指示令悟。富樓那,於意云何?此人縱迷,於此聚落更生迷不?不也,世尊。富樓那,十方如來,亦復如是。此迷無本,性畢竟空。昔本無迷,似有迷覺;覺迷迷滅,覺不生迷。亦如翳人,見空中華;翳病若除,華於空滅。忽有愚人,於彼空華所滅空地,待華更生。汝觀是人為愚?為慧?富樓那言:空元無華,妄見生滅。見華滅空,已是顚倒;勑令更出,斯實狂癡。云何更名如是狂人為愚?為慧?佛言:如汝所解,云何問言諸佛如來,妙覺明空,何當更出山河大地?又如金礦,雜於精金。其金一純,更不成雜;如木成灰,不重為木。諸佛如來菩提涅槃,亦復如是。

富樓那,又汝問言:地、水、火、風,本性圓融,周徧法界,疑水、火性,不相陵滅。又徵虛空及諸大地,俱徧法界,不合相容。富樓那,譬如虛空,體非群相,而不拒彼諸相發揮。所以者何?富樓那,彼太虛空,日照則明,雲屯則暗,風摇則動,霽澄則清,氣凝則濁,土積成霾,水澄成映。於意云何?如是殊方諸有為相,為因彼生?為復空有?若彼所生,富樓那,且日照時,旣是日明,十方世界同為日色,云何空中更見圓日?若是空明,空應自照,云何中宵雲霧之時,不生光矅?當知是明,非日非空,不異空日。眞妙覺明,亦復如是。汝以空明,則有空現。地、水、火、風,各各發明,則各各現;若俱發明,則有俱現。云何俱現?富樓那,如一水中,現於日影;兩人同觀水中之日,東西各行,則各有日隨二人去。一東一西先無準的。不應難言此日是一,云何各行?各日旣雙,云何現一?宛轉虛妄,無可憑據。觀相元妄,無可指陳;猶邀空華,結為空果,云何詰其相陵滅義?觀性元眞,唯妙覺明;妙覺明心,先非水火,云何復問不相容者?

富樓那,汝以色、空相傾相奪於如來藏。而如來藏隨為色、空,周徧法界;是故於中,風動、空澄、日明、雲暗。眾生迷悶,背覺合塵;故發塵勞,有世間相。我以妙明,不滅不生,合如來藏;而如來藏,唯妙覺明,圓照法界。是故於中,一為無量,無量為一;小中現大、大中現小。不動道場,徧十方界;身含十方,無盡虛空;於一毛瑞現寶王剎,坐微塵裏轉大法輪。滅塵合覺,故發眞如妙覺明性。而如來藏,本妙圓心。非心,非空,非地,非水,非風,非火;非眼,非耳、鼻、舌、身、意;非色,非聲、香、味、觸、法;非眼識界,如是乃至非意識界。非明、無明、明無明盡,如是乃至非老,非死,非老死盡;非苦,非集,非滅,非道,非智,非得;非檀那,非尸羅,非毗梨耶,非羼提,非禪那,非般刺若,非波羅蜜多;如是乃至非怛闥阿竭,非阿羅訶三耶三菩,非大涅槃;非常,非樂,非我,非淨。以是俱非,世、出世故。卽如來藏,元明心妙。卽心,卽空,卽地,卽水,卽風,卽火;卽眼,卽耳、鼻、舌、身、意;卽色,卽聲、香、味、觸、法;卽眼識界,如是乃至卽意識界。卽明、無明、明無明盡,如是乃至卽老,卽死,卽老死盡;卽苦,卽集,卽滅,卽道,卽智,卽得;卽檀那,卽尸羅,卽毗梨耶,卽羼提,卽禪那,卽般刺若,卽波羅蜜多;如是乃至卽怛闥阿竭,卽阿羅訶三耶三菩,卽大涅槃;卽常,卽樂,卽我,卽淨。以是俱卽,世、出世故。卽如來藏,元明心妙。離卽、離非,是卽、非卽。如何世間,三有眾生,及出世間聲聞、緣覺,以所知心,測度如來無上菩提,用世語言,入佛知見。譬如琴、瑟、箜篌、琵琶,雖有妙音,若無妙指,終不能發。汝與眾生,亦復如是。寶覺眞心,各各圓滿;如我按指,海印發光,汝暫舉心,塵勞先起。由不勤求無上覺道,愛念小乘,得少為足。

富樓那言:我與如來,寶覺圓明,眞妙淨心,無二圓滿。而我昔遭無始妄想,久在輪迴;今得聖乘,猶末究竟。世尊諸妄,一切圓滅,獨妙眞常。敢問如來:一切眾生,何因有妄,自蔽妙明,受此淪溺?佛告富樓那:汝雖除疑,餘惑未盡;吾以世間現前諸事,今復問汝。汝豈不聞?室羅城中,演若達名,忽於晨朝,以鏡照面。愛鏡中頭,眉目可見;瞋責己頭,不見面目,以為魑魅,無狀狂走。於意云何?此人何因,無故狂走?富樓那言:是人心狂,更無他故。佛言:妙覺明圓,本圓明妙,旣稱為妄,云何有因?若有所因,云何名妄?自諸妄想展轉相因,從迷積迷,以歷塵劫;雖佛發明,猶不能返。如是迷因,因迷自有;識迷無因,妄無所依。尚無有生,欲何為滅?得菩提者,如寤時人,說夢中事;心縱精明,欲何因緣,取夢中物?況復無因,本無所有;如彼城中演若達多,豈有因緣,自怖頭走?忽然狂歇,頭非外得;縱末歇狂,亦何遺失?富樓那,妄性如是,因何為在?汝但不隨分別世間、業果、眾生,三種相續;三緣斷故,三因不生。則汝心中,演若達多;狂性自歇,歇卽菩提。勝淨明心,本周法界,不從人得,何藉劬勞,肯綮修證。譬如有人,於自衣中,繫如意珠,不自覺知。窮露他方,乞食馳走,雖實貧窮,珠不曾失;忽有智者,指示其珠,所願從心,致大饒富。方悟神珠,非從外得。

卽時,阿難在大眾中,頂禮佛足,起立白佛:世尊,現說殺、盜、婬業,三緣斷故,三因不生。心中達多,狂性自歇;歇卽菩提,不從人得。斯則因緣,皎然明白,云何如來頓棄因緣?我從因緣,心得開悟。世尊,此義何獨我等年少,有學聲聞;今此會中,大目犍連,及舍利弗、須菩提等,從老梵志,聞佛因緣,發心開悟,得成無漏。今說菩提不從因緣。則王舍城、拘舍梨等,所說自然,成第一義。惟垂大悲,開發迷悶。

佛告阿難:卽如城中,演若達多,狂性因緣,若得滅除,則不狂性,自然而出,因緣、自然,理窮於是。阿難,演若達多,頭本自然,本自其然,無然非自;何因緣故,怖頭狂走?若自然頭,因緣故狂;何不自然,因緣故失?本頭不失,狂怖妄出,曾無變易,何藉因緣?本狂自然,本有狂怖;末狂之際,狂何所潛?不狂自然,頭本無妄,何為狂走?若悟本頭,識知狂走,因緣、自然,俱為戲論。是故我言,三緣斷故,卽菩提心。菩提心生,生滅心滅。此但生滅,滅生俱盡,無功用道。若有自然,如是則明自然心生,生滅心滅,此亦生滅。無生滅者,名為自然,猶如世間,諸相雜和成一體者,名和合性;非和合者,稱本然性。本然非然,和合非合;合然俱離,離合俱非。此句方名,無戲論法。菩提、涅槃,尚在遙遠,非汝歷劫辛勤修證。雖復憶持十方如來十二部經,清淨妙理如恆河沙,只益戲論。汝雖談說因緣、自然,決定明了,人間稱汝多聞第一。以此積劫,多聞熏習,不能免離摩登伽難。何須待我佛頂神呪,摩登伽心,婬火頓歇,得阿那含。於我法中,成精進林;愛河乾沽,今汝解脫。是故阿難,汝雖歷劫憶持如來秘密妙嚴,不如一日修無漏業,遠離世間憎、愛二苦。如摩登伽,宿為婬女;由神呪力鎖其愛欲,法中今名性比丘尼。與羅喉母耶輸陀羅,同悟宿因。知歷世因,貪愛為苦,一念薰修無漏善故,或得出纏,或蒙授記。如何自欺,尚留觀聽?

阿難及諸大眾,聞佛示誨,疑惑銷除,心悟實相;身意輕安,得未曾有。重復悲淚,頂禮佛足,長跪合掌而白佛言:無上大悲清淨寶王,善開我心;能以如是種種因緣,方便提獎,引諸沈冥,出於苦海。世尊,我今雖承如是法音,知加來藏妙覺明心徧十方界,含育如來十方國土,清淨寶嚴妙覺王剎。如來復責多聞無功,不逮修習。我今猶如旅泊之人,忽蒙天王賜以華屋;雖獲大宅,要因門入。惟願如來不捨大悲,示我在會諸蒙暗者,捐捨小乘,畢獲如來無餘涅槃本發心路。令有學者,從何攝伏疇昔攀緣,得陀羅尼,入佛知見。作是語已,五體投地。在會一心,佇佛慈旨。

爾時,世尊哀愍會中緣覺、聲聞,於菩提心末自在者,及為當來佛滅度後,末法眾生發菩薩心,開無上乘妙修行路。宣示阿難及諸大眾:汝等決定發菩提心,於佛如來妙三摩提,不生疲倦。應當先明發覺初心,二決定義。云何初心二義決定?

阿難,第一義者,汝等若欲捐捨聲聞,修菩薩乘,入佛知見。應當審觀因地發心,與果地覺為同?為異?阿難,若於因地,以生滅心為本修因,而求佛乘不生不滅,無有是處。以是義故,汝當照明諸器世間。可作之法,皆從變滅。阿難,汝觀世間可作之法,誰為不壞?然終不聞爛壞虛空,何以故?空非可作,由是始終無壞滅故。則汝身中,堅相為地,潤濕為水,煖觸為火,動摇為風。由此四纏,分汝湛圓妙覺明心,為視、為聽、為覺、為察,從始入終,五疊渾濁。云何為濁?阿難,譬如清水,清潔本然。卽彼塵土灰砂之倫,本質留礙;二體法爾,性不相循。有世間人,取彼土塵,投於淨水;土失留礙,水亡清潔,容貌汨然名之為濁。汝濁五重,亦復如是。阿難,汝見虛空徧十方界,空見不分;有空無體,有見無覺,相織妄成。是第一重,名為劫濁。汝身現搏四大為體。見、聞、覺、知,壅令留礙;水、火、風、土,旋今覺知,相織妄成。是第二重,名為見濁。又汝心中,憶識誦習,性發知見,容現六塵;離塵無相,離覺無性,相織妄成。是第三重,名煩惱濁。又汝朝夕生滅不停,知見每欲留於世間,業運每常遷於國土,相織妄成。是第四重,名眾生濁。汝等見聞,元無異性;眾塵隔越,無狀異生。性中相知,用中相背,同異失準,相織妄成。是第五重,名為命濁。

阿難,汝今欲今見、聞、覺、知,遠契如來常、樂、我、淨;應當先擇死生根本,依不生滅圓湛性成。以湛旋其虛妄滅生,復還元覺;得元明覺,無生滅性為因地心。然後圓成果地修證。如澄濁水,貯於淨器。靜深不動,沙土自沈,清水現前,名為初伏客塵煩惱。去泥純水,名為永斷根本無明。明相精純,一切變現,不為煩惱,皆合涅槃清淨妙德。

第二義者,汝等必欲發菩提心,於菩薩乘,生大勇猛,決定棄捐諸有為相。應當審詳煩惱根本,此無始來,發業潤生,誰作?誰受?阿難,汝修菩提,若不審觀煩惱根本,則不能知虛妄根塵何處顚倒。處尚不知,云何降伏取如來位?阿難,汝觀世間解結之人,不見所結,云何知解?不聞虛空被汝隳裂,何以故?空無相形,無結解故。則汝現前眼、耳、鼻、舌,及與身心,六為賊媒,自劫家寶。由此無始眾生世界,生纏縛故,於器世間不能超越。

阿難,云何名為眾生世界?世為遷流,界為方位。汝今當知東、西、南、北、東南、西南、東北、西北,上、下為界,過去、末來、現在為世;方位有十,流數有三。一切眾生,織妄相成,身中貿遷,世界相涉。而此界性,設雖十方,定位可明。世間只目東、西、南、北,上、下無位,中無定方。四數必明,與世相涉,三四、四三,宛轉十二;流變三疊,一十百千。總括始終,六根之中,各各功德有千二百。阿難,汝復於中,克定優劣。如眼觀見,後暗前明;前方全明,後方全暗。左右旁觀,三分之二,統論所作,功德不全。三分言功,一分無德,當知眼唯八百功德。如耳周聽,十方無遺;動若邇遙,諍無邊際。當知耳根圓滿一千二百功德。如鼻齅聞,通出入息;有出有入,而闕中交。驗於鼻根,三分闕一,當知鼻唯八百功德。如舌宣揚,盡諸世間、出世間智;言有方分,理無窮盡。當知舌根圓滿一千二百功德。如身覺觸,識於違順;合時能覺,離中不知,離一合雙。驗於身根三分闕一,當知身唯八百功德。如意默容,十方三世一切世間出世間法,唯聖與凡,無不包容,盡其涯際。當知意根圓滿一千二百功德。

阿難,汝今欲逆生死欲流,返窮流根,至不生滅。當驗此等六受用根:誰合?誰離?誰深?誰淺?誰為圓通?誰不圓滿?若能於此,悟圓通根,逆彼無始織妄業流,得循圓通與不圓根,日劫相倍。我今備顯六湛圓明本所功德,數量如是。隨汝詳擇其可入者,吾當發明,今汝增進。十方如來於十八界,一一修行,皆得圓滿無上菩提,於其中間亦無優劣。但汝下劣,末能於中,圓自在慧;故我宣揚,令汝但於一門深入。入一無妄,彼六知根,一時清淨。

阿難白佛言:世尊,云何逆流深入一門,能令六根一時清淨?佛告阿難:汝今已得須陀洹果,已滅三界眾生世間見所斷惑;然猶末知根中積生無始虛習,彼習要因修所斷得。何況此中,生、住、異、滅,分劑頭數?今汝且觀現前六根為一?為六?阿難,若言一者,耳何不見?目何不聞?頭奚不履?足奚無語?若此六根決定成六,如我今會與汝宣揚微妙法門,汝之六根,誰來領受?阿難言:我用耳聞。

佛言:汝耳自聞,何關身口?口來問義,身起欽承?是故應知,非一終六,非六終一;終不汝根,元一元六。阿難,當知是根非一、非六,由無始來顚倒淪替,故於圓湛一六義生。汝須陀洹雖得六銷,猶末亡一。如太虛空,參合群器,由器形異,名之異空。除器觀空,說空為一。彼太虛空,云何為汝成同、不同?何況更名是一、非一?則汝了知六受用根,亦復如是。由明、暗等二種相形,於妙圓中,黏湛發見。見精映色,結色成根;根元目為清淨四大,因名眼體如蒲萄朵。浮根四塵,流逸奔色。由動、靜等二種相擊,於妙圓中,黏湛發聽。聽精映聲,卷聲成根;根元目為清淨四大,因名耳體如新卷葉。浮根四塵,流逸奔聲。由通、塞等二種相發,於妙圓中,黏湛發齅。齅精映香,納香成根;根元目為清淨四大,因名鼻體如雙垂爪。浮根四塵,流逸奔香。由恬、變等二種相參,於妙圓中,黏湛發嚐。嚐精映味,絞味成根;根元目為清淨四大,因名舌體如初偃月。浮根四塵,流逸奔味。由離、合等二種相摩,於妙圓中,黏湛發覺。覺精映觸,搏觸成根,根元目為清淨四大,因名身體如腰鼓顙。浮根四塵,流逸奔觸。由生、滅等二種相續,於妙圓中,黏湛發知。知精映法,覽法成根;根元目為清淨四大,因名意思如幽室見。浮根四塵,流逸奔法。

阿難,如是六根,由彼覺明有明明覺,失彼精了,黏妄發光。是以汝今,離暗、離明,無有見體;離動、離靜,元無聽質;無通、無塞,齅性不生;非變、非恬,嚐無所出;不離、不合,覺觸本無;無滅、無生,了知安寄?汝但不循動、靜、合、離,恬、變、通、塞,生、滅、明、暗,如是十二諸有為相。隨拔一根,脫黏內伏;伏歸元眞,發本明矅。矅性發明,諸餘五黏應拔圓脫。

不由前塵所起知見,明不循根,寄根明發;由是六根,互相為用。阿難,汝豈不知今此會中,阿那律陀,無目而見?跋難陀龍,無身而聽?殑伽神女,非鼻聞香?驕梵缽提,異舌知味?舜若多神,無身有觸?如來光中,映令暫現,旣為風質,其體元無。諸滅盡定得寂聲聞,如此會中摩訶迦葉,久滅意根,圓明了知,不因心念?阿難,今汝諸根若圓拔已,內瑩發光,如是浮塵,及器世間諸變化相,如湯銷冰,應念化成無上知覺。阿難,如彼世人,聚見於眼。若令急合,暗相現前;六根黯然,頭足相類。彼人以手,循體外繞;彼雖不見,頭足一辨,知覺是同。緣見因明,暗成無見;不明自發,則諸暗相永不能昏。根塵旣銷,云何覺明不成圓妙?

阿難白佛言:世尊,如佛說言,因地覺心,欲求常住,要與果位,名目相應。世尊,如果位中,菩提、涅槃、眞如、佛性、菴摩羅識、空如來藏、大圓鏡智。是七種名,稱謂雖別;清淨圓滿,體性堅凝,如金剛王常住不壞。若此見聽,離於明暗、動靜、通塞,畢竟無體;猶如念心,離於前塵,本無所有。云何將此畢竟斷滅,以為修因,欲獲如來七常住果?世尊,若離明、暗,見畢竟空;如無前塵,念自性滅。進退循環,微細推求;本無我心,及我心所。將誰立因,求無上覺?如來先說湛精圓常。違越誠言,終成戲論,云何如來眞實語者?惟垂大慈,開我蒙恡。

佛告阿難:汝學多聞,未盡諸漏。心中徒知顚倒所因,眞倒現前,實末能識。恐汝誠心,猶未信伏;吾今試將塵俗諸事,當除汝疑。卽時,如來勑羅喉羅,擊鐘一聲。問阿難言:汝今聞不?阿難、大眾俱言:我聞。鐘歇無聲,佛又問言:汝今聞不?阿難、大眾俱言:不聞。時羅喉羅又擊一聲,佛又問言:汝今聞不?阿難、大眾又言:俱聞。佛問阿難:汝云何聞?云何不聞?阿難、大眾俱白佛言:鐘聲若擊,則我得聞;擊久聲銷,音響雙絕,則名無聞。如來又勑羅喉羅擊鐘,問阿難言:汝今聲不?阿難、大眾俱言:有聲。少頃聲銷,佛又問言:爾今聲不?阿難、大眾答言:無聲。有頃羅喉更來撞鐘,佛又問言:爾今聲不?阿難、大眾俱言:有聲。佛問阿難:汝云何聲?云何無聲?阿難、大眾俱白佛言:鐘聲若擊,則名有聲;擊久聲銷,音響雙絕,則名無聲。

佛語阿難及諸大眾:汝今云何自語矯亂?大眾、阿難俱時問佛:我今云何名為矯亂?佛言:我問汝聞,汝則言聞;又問汝聲,汝則言聲。唯聞與聲,報答無定,如是云何不名矯亂?阿難,聲銷無響,汝說無聞。若實無聞,聞性已滅,同於枯木;鐘聲更擊,汝云何知?知有、知無;自是聲塵或無、或有。豈彼聞性為汝有無?聞實云無,誰知無者?是故阿難,聲於聞中,自有生滅;非為汝聞聲生、聲滅,令汝聞性為有、為無。汝尚顚倒,惑聲為聞;何怪昏迷,以常為斷?終不應言離諸動、靜、閉塞、開通,說聞無性。如重睡人,眠熟床枕。其家有人,於彼睡時,擣練舂米。其人夢中,聞舂擣聲,別作他物:或為擊鼓,或為撞鐘。卽於夢時,自怪其鐘為木石響。於時忽寤,遄知舂音,自告家人:我正夢時,惑此舂音,將為鼓響。阿難,是人夢中,豈憶靜摇、開閉、通塞?其形雖寐,聞性不昏。縱汝形銷,命光遷謝,此性云何為汝銷滅?

以諸眾生,從無始來,循諸色聲,逐念流轉;曾不開悟,性淨妙常。不循所常,逐諸生滅,由是生生雜染流轉。若棄生滅,守於眞常;常光現前,根、塵、識心,應時銷落。想相為塵,識情為垢。二俱遠離,則汝法眼,應時清明,云何不成無上知覺?

%\vspace{3cm}
%\begin{center}
%\includegraphics[width=.8\textwidth]{pictures/yzjs.png}
%\end{center}
