% $Id: juan01.tex,v 1.4 2011/08/09 09:08:36 xiongyw Exp $

如是我聞。

一時,佛在室羅筏城祇桓精舍。與大比丘眾千二百五十人俱,皆是無漏大阿羅漢。佛子住持,善超諸有;能於國土,成就威儀。從佛轉輪,妙堪遺囑;嚴淨毗尼,弘範三界。應身無量,度脫眾生;拔濟未來,越諸塵累。其名曰:大智舍利弗、摩訶目犍連、摩訶拘絺羅、富樓那彌多羅尼子、須菩提、優波尼沙陀等,而為上首。復有無量辟支無學,並其初心,同來佛所。

屬諸比丘,休夏自恣。十方菩薩,咨決心疑,欽奉慈嚴,將求密義。卽時,如來敷座宴安,為諸會中宣示深奧。法筵清眾得未曾有,迦陵仙音徧十方界。恆沙菩薩來聚道場,文殊師利而為上首。

時波斯匿王為其父王諱日營齋,請佛宮掖,自迎如來。廣設珍羞無上妙味,兼復親延諸大菩薩。城中復有長者居士同時飯僧,佇佛來應。佛勑文殊分領菩薩及阿羅漢,應諸齋主。唯有阿難先受別請,遠遊未還,不遑僧次。

旣無上座及阿闍黎,途中獨歸,其日無供。卽時,阿難執持應器,於所遊城次第循乞,心中初求最後檀越,以為齋主。無問淨穢、剎利尊姓及旃陀羅,方行等慈,不擇微賤,發意圓成一切眾生無量功德。阿難已知,如來世尊訶須菩提及大迦葉,為阿羅漢心不均平,欽仰如來開闡無遮,度諸疑謗。經彼城隍,徐步郭門。嚴整威儀,肅恭齋法。

爾時,阿難因乞食次經歷婬室,遭大幻術。摩登伽女以娑毗迦羅先梵天呪攝入婬席。婬躬撫摩,將毀戒體。如來知彼婬術所加,齋畢旋歸。王及大臣、長者居士,俱來隨佛願聞法要。於時,世尊頂放百寶無畏光明,光中出生千葉寶蓮,有佛化身結跏趺坐,宣說神呪。勑文殊師利將呪往護。惡呪銷滅,提獎阿難及摩登伽歸來佛所。

阿難見佛,頂禮悲泣,恨無始來一向多聞,未全道力。殷勤啟請,十方如來得成菩提,妙奢摩他、三摩、禪那最初方便。於時,復有恆沙菩薩及諸十方大阿羅漢、辟支佛等俱願樂聞,退坐默然,承受聖旨。

爾時,世尊在大眾中舒金色臂摩阿難頂,告示阿難及諸大眾:“有三摩提,名大佛頂首楞嚴王,具足萬行。十方如來,一門超出妙莊嚴路。汝今諦聽。”

阿難頂禮,伏受慈旨。

佛告阿難:“汝我同氣,情均天倫。當初發心,於我法中見何勝相,頓捨世間深重恩愛?”

阿難白佛:“我見如來三十二相,勝妙殊絕。形體映徹,猶如璢璃。常自思惟,此相非是欲愛所生。何以故?欲氣麤濁,腥臊交遘,膿血雜亂,不能發生勝淨妙明紫金光聚。是以渴仰,從佛剃落。”

佛言:“善哉!阿難,汝等當知,一切眾生從無始來,生死相續,皆由不知常住眞心性淨明體,用諸妄想。此想不眞,故有輪轉。汝今欲研無上菩提,眞發明性,應當直心詶我所問。十方如來同一道故,出離生死,皆以直心。心言直故,如是乃至終始地位,中間永無諸委曲相。阿難,我今問汝,當汝發心,緣於如來三十二相,將何所見?誰為愛樂?”

阿難白佛言:“世尊,如是愛樂,用我心目。由目觀見如來勝相,心生愛樂,故我發心願捨生死。”

佛告阿難:“如汝所說,眞所愛樂,因於心目。若不識知心目所在,則不能得降伏塵勞。譬如國王為賊所侵,發兵討除,是兵要當知賊所在。使汝流轉,心目為咎。吾今問汝,唯心與目今何所在?”

阿難白佛言:“世尊,一切世間十種異生,同將識心居在身內。縱觀如來青蓮華眼,亦在佛面。我今觀此浮根四塵,祇在我面。如是識心,實居身內。”

佛告阿難:“汝今現坐如來講堂,觀祇陀林,今何所在?”

“世尊,此大重閣清淨講堂在給孤園,今祇陀林實在堂外。”

“阿難,汝今堂中,先何所見?”

“世尊,我在堂中,先見如來,次觀大眾。如是外望,方矚林園。”

“阿難,汝矚林園,因何有見?”

“世尊,此大講堂戶牖開豁,故我在堂得遠瞻見。”

佛告阿難:“如汝所言,身在講堂,戶牖開豁,遠矚林園。亦有眾生在此堂中,不見如來,見堂外者?”

阿難答言:“世尊,在堂不見如來,能見林泉,無有是處。”

“阿難,汝亦如是。汝之心靈,一切明了。若汝現前所明了心,實在身內,爾時先合了知內身。頗有眾生先見身中,後觀外物?縱不能見心、肝、脾、胃,爪生、髮長、筋轉、脈摇誠合明了,如何不知?必不內知,云何知外?是故應知,汝言覺了能知之心,住在身內,無有是處。”

阿難稽首而白佛言:“我聞如來如是法音,悟知我心實居身外。所以者何?譬如燈光然於室中,是燈必能先照室內,從其室門後及庭際。一切眾生,不見身中,獨見身外,亦如燈光居在室外,不能照室。是義必明,將無所惑,同佛了義,得無妄耶?”

佛告阿難:“是諸比丘,適來從我室羅筏城循乞摶食,歸祇陀林。我已宿齋,汝觀比丘,一人食時,諸人飽不?”

阿難答言:“不也,世尊。何以故?是諸比丘,雖阿羅漢,軀命不同。云何一人能令眾飽?”

佛告阿難:“若汝覺了知見之心,實在身外,身心相外,自不相干。則心所知,身不能覺;覺在身際,心不能知。我今示汝兜羅綿手,汝眼見時,心分別不?”

阿難答言:“如是,世尊。”

佛告阿難:“若相知者,云何在外?是故應知,汝言覺了能知之心,住在身外,無有是處。”

阿難白佛言:“世尊,如佛所言,不見內故,不居身內;身心相知,不相離故,不在身外。我今思惟,知在一處。”

佛言:“處今何在?”

阿難言:“此了知心,旣不知內而能見外,如我思忖,潛伏根裏。猶如有人,取璢璃椀合其兩眼。雖有物合,而不留礙;彼根隨見,隨卽分別。然我覺了能知之心,不見內者,為在根故;分明矚外無障礙者,潛根內故。”

佛告阿難:“如汝所言,潛根內者,猶如璢璃。彼人當以璢璃籠眼,當見山河,見璢璃不?”

“如是,世尊。是人當以璢璃籠眼,實見璢璃。”

佛告阿難:“汝心若同璢璃合者,當見山河,何不見眼?若見眼者,眼卽同境,不得成隨。若不能見,云何說言此了知心潛在根內,如璢璃合?是故應知,汝言覺了能知之心,潛伏根裏,如璢璃合,無有是處。”

阿難白佛言:“世尊,我今又作如是思惟,是眾生身,腑藏在中,竅穴居外;有藏則暗,有竅則明。今我對佛,開眼見明,名為見外;閉眼見暗,名為見內。是義云何?”

佛告阿難:“汝當閉眼見暗之時,此暗境界,為與眼對?為不眼對?若與眼對,暗在眼前,云何成內?若成內者,居暗室中,無日、月、燈,此室暗中,皆汝焦腑?若不對者,云何成見?若離外見,內對所成。合眼見暗,名為身中;開眼見明,何不見面?若不見面,內對不成;見面若成,此了知心及與眼根,乃在虛空,何成在內?若在虛空,自非汝體,卽應如來今見汝面,亦是汝身。汝眼已知,身合非覺,必汝執言身、眼兩覺,應有二知。卽汝一身,應成兩佛。是故應知,汝言見暗,名見內者,無有是處。”

阿難言:“我嘗聞佛開示四眾:‘由心生故,種種法生;由法生故,種種心生。’我今思惟,卽思惟體,實我心性。隨所合處,心則隨有。亦非內、外、中間三處。”

佛告阿難:“汝今說言,由法生故,種種心生,隨所合處,心隨有者,是心無體,則無所合。若無有體而能合者,則十九界因七塵合。是義不然。若有體者,如汝以手自挃其體,汝所知心為復內出?為從外入?若復內出,還見身中;若從外來,先合見面。”

阿難言:“見是其眼,心知非眼,為見非義。”

佛言:“若眼能見,汝在室中,門能見不?則諸已死,尚有眼存,應皆見物。若見物者,云何名死?阿難,又汝覺了能知之心,若必有體,為復一體?為有多體?今在汝身,為復徧體?為不徧體?若一體者,則汝以手挃一支時,四支應覺。若咸覺者,挃應無在。若挃有所,則汝一體自不能成。若多體者,則成多人,何體為汝?若徧體者,同前所挃。若不徧者,當汝觸頭,亦觸其足,頭有所覺,足應無知。今汝不然。是故應知,隨所合處,心則隨有,無有是處。”

阿難白佛言:“世尊,我亦聞佛與文殊等諸法王子談實相時,世尊亦言‘心不在內,亦不在外’。如我思惟,內無所見,外不相知。內無知故,在內不成;身心相知,在外非義。今相知故,復內無見,當在中間。”

佛言:“汝言中間,中必不迷,非無所在。今汝推中,中何為在?為復在處?為當在身?若在身者,在邊非中,在中同內。若在處者,為有所表?為無所表?無表同無,表則無定。何以故?如人以表,表為中時,東看則西,南觀成北。表體旣混,心應雜亂。”

阿難言:“我所說中,非此二種。如世尊言,眼色為緣生於眼識,眼有分別,色塵無知。識生其中,則為心在。”

佛言:“汝心若在根塵之中,此之心體,為復兼二?為不兼二?若兼二者,物體雜亂。物非體知,成敵兩立,云何為中?兼二不成,非知、不知,卽無體性,中何為相?是故應知,當在中間,無有是處。”

阿難白佛言:“世尊,我昔見佛與大目連、須菩提、富樓那、舍利弗四大弟子共轉法輪,常言覺知分別心性,旣不在內亦不在外,不在中間,俱無所在。一切無著,名之為心。則我無著,名為心不?”

佛告阿難:“汝言覺知分別心性,俱無在者,世間虛空水陸飛行,諸所物象,名為一切。汝不著者,為在?為無?無則同於龜毛、兔角,云何不著?有不著者,不可名無。無相則無,非無則相。相有則在,云何無著?是故應知,一切無著,名覺知心,無有是處。”

爾時,阿難在大眾中卽從座起,偏袒右肩,右膝著地,合掌恭敬,而白佛言:“我是如來最小之弟,蒙佛慈愛。雖今出家,猶恃憍憐,所以多聞未得無漏。不能折伏娑毗羅呪,為彼所轉,溺於婬舍,當由不知眞際所詣。惟願世尊大慈哀愍,開示我等奢摩他路,令諸闡提隳彌戾車。”作是語已,五體投地。及諸大眾,傾渴翹佇,欽聞示誨。

爾時,世尊從其面門放種種光,其光晃矅如百千日。普佛世界六種震動,如是十方微塵國土一時開現,佛之威神令諸世界合成一界。其世界中所有一切諸大菩薩,皆住本國,合掌承聽。

佛告阿難:“一切眾生從無始來種種顚倒,業種自然如惡叉聚。諸修行人不能得成無上菩提,乃至別成聲聞、緣覺,及成外道、諸天魔王及魔眷屬,皆由不知二種根本,錯亂修習。猶如煮沙欲成嘉饌,縱經塵劫終不能得。云何二種?阿難,一者,無始生死根本。則汝今者,與諸眾生用攀緣心為自性者。二者,無始菩提涅槃元清淨體。則汝今者,識精元明,能生諸緣,緣所遺者。由諸眾生遺此本明,雖終日行而不自覺,枉入諸趣。阿難,汝今欲知奢摩他路,願出生死。今復問汝。”

卽時,如來舉金色臂,屈五輪指,語阿難言:“汝今見不?”

阿難言:“見。”

佛言:“汝何所見?”

阿難言:“我見如來舉臂屈指,為光明拳,矅我心目。”

佛言:“汝將誰見?”

阿難言:“我與大眾同將眼見。”

佛告阿難:“汝今答我,如來屈指為光明拳,矅汝心目,汝目可見。以何為心,當我拳矅?”

阿難言:“如來現今徵心所在,而我以心推窮尋逐。卽能推者,我將為心。”

佛言:“咄!阿難,此非汝心。”

阿難矍然,避座合掌,起立白佛:“此非我心,當名何等?”

佛告阿難:“此是前塵虛妄相想,惑汝眞性。由汝無始至於今生,認賊為子,失汝元常,故受輪轉。”

阿難白佛言:“世尊,我佛寵弟心愛佛故,令我出家。我心何獨供養如來,乃至徧歷恆沙國土,承事諸佛及善知識,發大勇猛,行諸一切難行法事,皆用此心。縱令謗法,永退善根,亦因此心。若此發明不是心者,我乃無心,同諸土木?離此覺知,更無所有。云何如來說此非心?我實驚怖,兼此大眾無不疑惑。惟垂大悲,開示未悟。”

爾時,世尊開示阿難及諸大眾,欲令心入無生法忍。於師子座摩阿難頂而告之言:“如來常說‘諸法所生,唯心所現’。一切因果,世界、微塵,因心成體。阿難,若諸世界一切所有,其中乃至草葉縷結,詰其根元,咸有體性。縱令虛空,亦有名貌,何況清淨妙淨明心,性一切心,而自無體?若汝執恡,分別覺觀所了知性,必為心者,此心卽應離諸一切色、香、味、觸諸塵事業,別有全性。如汝今者,承聽我法,此則因聲而有分別。縱滅一切見、聞、覺、知,內守幽閑,猶為法塵分別影事。我非勑汝執為非心,但汝於心,微細揣摩。若離前塵有分別性,卽眞汝心。若分別性離塵無體,斯則前塵分別影事。塵非常住,若變滅時,此心則同龜毛、兔角,則汝法身同於斷滅,其誰修證無生法忍?”

卽時,阿難與諸大眾默然自失。

佛告阿難:“世間一切諸修學人,現前雖成九次第定,不得漏盡成阿羅漢,皆由執此生死妄想,誤為眞實。是故汝今雖得多聞,不成聖果。”

阿難聞已,重復悲淚,五體投地,長跪合掌而白佛言:“自我從佛發心出家,恃佛威神,常自思惟,無勞我修,將謂如來惠我三昧。不知身心本不相代,失我本心。身雖出家,心不入道;譬如窮子,捨父逃逝。今日乃知,雖有多聞,若不修行,與不聞等;如人說食,終不能飽。世尊,我等今者二障所纏,良由不知寂常心性。惟願如來哀愍窮露,發妙明心,開我道眼。”

卽時,如來從胷卍字涌出寶光。其光晃昱,有百千色。十方微塵普佛世界,一時周徧。徧灌十方所有寶剎諸如來頂,旋至阿難及諸大眾。告阿難言:“吾今為汝建大法幢,亦令十方一切眾生,獲妙微密性淨明心,得清淨眼。阿難,汝先答我見光明拳。此拳光明,因何所有?云何成拳?汝將誰見?”

阿難言:“由佛全體閻浮檀金,赩如寶山,清淨所生,故有光明;我實眼觀五輪指端屈握示人,故有拳相。”

佛告阿難:“如來今日實言告汝。諸有智者要以譬喻而得開悟。阿難,譬如我拳,若無我手不成我拳,若無汝眼不成汝見。以汝眼根例我拳理,其義均不?”

阿難言:“唯然,世尊。旣無我眼,不成我見。以我眼根例如來拳,事義相類。”

佛告阿難:“汝言相類,是義不然。何以故?如無手人,拳畢竟滅;彼無眼者,非見全無。所以者何?汝試於途詢問盲人:‘汝何所見?’彼諸盲人必來答汝:‘我今眼前唯見黑暗,更無他矚。’以是義觀,前塵自暗,見何虧損?”

阿難言:“諸盲眼前惟睹黑暗,云何成見?”

佛告阿難:“諸盲無眼,唯睹黑暗,與有眼人處於暗室,二黑有別?為無有別?”

“如是,世尊。此暗中人與彼群盲,二黑較量,曾無有異。”

“阿難,若無眼人全見前黑,忽得眼光,還於前塵見種種色,名眼見者。彼暗中人全見前黑,忽獲燈光,亦於前塵見種種色,應名燈見。若燈見者,燈能有見,自不名燈;又則燈觀,何關汝事?是故當知,燈能顯色,如是見者,是眼非燈。眼能顯色,如是見性,是心非眼。”

阿難雖復得聞是言,與諸大眾口已默然,心未開悟。猶冀如來慈音宣示,合掌清心,佇佛悲誨。

爾時,世尊舒兜羅綿網相光手,開五輪指,誨勑阿難及諸大眾:“我初成道,於鹿園中為阿若多五比丘等及汝四眾言:‘一切眾生不成菩提及阿羅漢,皆由客塵煩惱所誤。’汝等當時因何開悟,今成聖果?”

時憍陳那起立白佛:“我今長老,於大眾中獨得解名,因悟‘客塵’二字成果。世尊,譬如行客投寄旅亭,或宿或食,宿食事畢,俶裝前途,不遑安住。若實主人,自無攸往。如是思惟,不住名‘客’,住名‘主人’。以不住者,名為‘客’義。又如新霽,清暘升天,光入隙中,發明空中諸有塵相。塵質摇動,虛空寂然。如是思惟,澄寂名‘空’,摇動名‘塵’。以摇動者,名為‘塵’義。”

佛言:“如是。”

卽時,如來於大眾中屈五輪指,屈已,復開;開已,又屈。謂阿難言:“汝今何見?”

阿難言:“我見如來百寶輪掌,眾中開合。”

佛告阿難:“汝見我手眾中開合,為是我手有開有合?為復汝見有開有合?”

阿難言:“世尊寶手眾中開合。我見如來手自開合,非我見性有開有合。”

佛言:“誰動?誰靜?”

阿難言:“佛手不住。而我見性尚無有靜,誰為無住?”

佛言:“如是。”

如來於是從輪掌中飛一寶光在阿難右,卽時阿難迴首右盼。又放一光在阿難左,阿難又則迴首左盼。佛告阿難:“汝頭今日何因摇動?”

阿難言:“我見如來出妙寶光來我左右,故左右觀,頭自摇動。”

“阿難,汝盼佛光,左右動頭。為汝頭動?為復見動?”

“世尊,我頭自動。而我見性尚無有止,誰為摇動?”

佛言:“如是。”

於是,如來普告大眾:“若復眾生以摇動者名之為‘塵’,以不住者名之為‘客’。汝觀阿難,頭自動摇,見無所動;又汝觀我,手自開合,見無舒卷。云何汝今以動為身?以動為境?從始洎終,念念生滅,遺失眞性,顚倒行事。性心失眞,認物為己。輪迴是中,自取流轉。”

%\vspace{1cm}
%\begin{center}
%\setlength{\fboxsep}{1mm} 
%\fbox{%
%\includegraphics[width=.5\textwidth]{pictures/hxxzy.png}
%}
%\end{center}