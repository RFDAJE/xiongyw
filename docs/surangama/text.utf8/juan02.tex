% $Id: juan02.tex,v 1.3 2011/08/09 09:08:36 xiongyw Exp $

爾時,阿難及諸大眾聞佛示誨,身心泰然。念無始來失卻本心,妄認緣塵,分別影事。今日開悟,如失乳兒忽遇慈母。合掌禮佛,願聞如來顯出身心眞妄虛實,現前生滅與不生滅二發明性。

時波斯匿王起立白佛:“我昔未承諸佛誨勑,見迦旃延、毗羅胝子,咸言此身死後斷滅,名為涅槃。我雖值佛,今猶狐疑。云何發揮證知此心不生滅地?今此大眾諸有漏者,咸皆願聞。”

佛告大王:“汝身現在,今復問汝,汝此肉身,為同金剛常住不朽?為復變壞?”

“世尊,我今此身終從變滅。”

佛言:“大王,汝未曾滅,云何知滅?”

“世尊,我此無常變壞之身雖未曾滅,我觀現前,念念遷謝,新新不住。如火成灰,漸漸銷殞,殞亡不息。決知此身,當從滅盡。”

佛言:“如是,大王。汝今生齡已從衰老,顏貌何如童子之時?”

“世尊,我昔孩孺,膚腠潤澤;年至長成,血氣充滿;而今頹齡,迫於衰耄,形色枯悴,精神昏昧,髮白面皺,逮將不久。如何見比充盛之時?”

佛言:“大王,汝之形容應不頓朽?”

王言:“世尊,變化密移,我誠不覺;寒暑遷流,漸至於此。何以故?我年二十,雖號年少,顏貌已老初十歲時;三十之年,又衰二十;於今六十又過於二,觀五十時,宛然強壯。世尊,我見密移,雖此殂落,其間流易,且限十年。若復令我微細思惟,其變寧唯一紀、二紀,實唯年變。豈唯年變?亦兼月化。何直月化?兼又日遷。沈思諦觀,剎那剎那,念念之間,不得停住。故知我身,終從變滅。”

佛告大王:“汝見變化遷改不停,悟知汝滅。亦於滅時,汝知身中有不滅耶?”

波斯匿王合掌白佛:“我實不知。”

佛言:“我今示汝不生滅性。大王,汝年幾時見恆河水?”

王言:“我生三歲,慈母攜我謁耆婆天,經過此流。爾時卽知是恆河水。”

佛言:“大王,如汝所說,二十之時衰於十歲,乃至六十,日月歲時,念念遷變。則汝三歲見此河時,至年十三,其水云何?”

王言:“如三歲時,宛然無異。乃至於今,年六十二,亦無有異。”

佛言:“汝今自傷髮白面皺,其面必定皺於童年。則汝今時觀此恆河,與昔童時觀河之見,有童耄不?”

王言:“不也,世尊。”

佛言:“大王,汝面雖皺,而此見精性未曾皺。皺者為變,不皺非變。變者受滅;彼不變者,元無生滅。云何於中,受汝生死?而猶引彼末伽梨等,都言此身死後全滅。”

王聞是言,信知身後捨生趣生。與諸大眾踊躍歡喜,得未曾有。

阿難卽從座起,禮佛合掌,長跪白佛:“世尊,若此見聞必不生滅,云何世尊名我等輩遺失眞性,顚倒行事?願興慈悲,洗我塵垢。”

卽時,如來垂金色臂,輪手下指,示阿難言:“汝今見我母陀羅手,為正為倒?”

阿難言:“世間眾生以此為倒,而我不知誰正誰倒。”

佛告阿難:“若世間人以此為倒,卽世間人將何為正?”

阿難言:“如來豎臂,兜羅綿手上指於空,則名為正。”

佛卽豎臂,告阿難言:“若此顚倒,首尾相換,諸世間人,一倍瞻視。則知汝身,與諸如來清淨法身,比類發明。如來之身名正徧知,汝等之身號性顚倒。隨汝諦觀,汝身佛身,稱顚倒者,名字何處號為顚倒?”

於時,阿難與諸大眾瞪瞢瞻佛,目睛不瞬,不知身心顚倒所在。

佛興慈悲,哀愍阿難,及諸大眾,發海潮音,徧告同會:諸善男子,我常說言:色心諸緣,及心所使,諸所緣法,唯心所現。汝身汝心,皆是妙明,眞精妙心,中所現物。云何汝等,遺失本妙,圓妙明心,寶明妙性,認悟中迷?晦昧為空;空晦暗中,結暗為色。色雜妄想,想相為身。聚緣內摇,趣外奔逸,昏擾擾相,以為心性。一迷為心,決定惑為色身之內。不知色身,外洎山河、虛空、大地,咸是妙明,眞心中物。譬如澄清百千大海,棄之,唯認一浮漚體。目為全潮,窮盡瀛渤。汝等卽是迷中倍人。如我垂手,等無差別,如來說為可憐愍者。

阿難承佛悲救深誨,垂泣叉手,而白佛言:我雖承佛如是妙音,悟妙明心,元所圓滿,常住心地。而我悟佛,現說法音,現以緣心,允所瞻仰。徒獲此心,未敢認為本元心地。願佛哀愍,宣示圓音;拔我疑根,歸無上道。

佛告阿難:汝等尚以緣心聽法,此法亦緣,非得法性。如人以手,指月示人,彼人因指,當應看月。若復觀指,以為月體。此人豈唯亡失月輪,亦亡其指。何以故?以所標指,為明月故。豈唯亡指,亦復不識明之與暗。何以故?卽以指體為月明性,明暗二性無所了故。汝亦如是。若以分別我說法音為汝心者,此心自應離分別音,有分別性;譬如有客寄宿旅亭,暫止便去,終不常住。而掌亭人都無所去,名為亭主。此亦如是。若眞汝心,則無所去;云何離聲無分別性?斯則豈唯生分別心;分別我容,離諸色相,無分別性。如是乃至分別都無,非色非空,拘舍離等昧為冥諦,離諸法緣無分別性。則汝心性各有所還,云何為主?

阿難言:若我心性各有所還;則如來說妙明元心,云何無還?惟垂哀愍,為我宣說。

佛告阿難:且汝見我見精明元,此見雖非妙精明心,如第二月,非是月影。汝應諦聽,今當示汝無所還地。阿難,此大講堂,洞開東方。日輪升天,則有明矅;中夜黑月,雲霧晦瞑,則復昏暗;戶牖之隙,則復見通;牆宇之間,則復觀壅;分別之處,則復見緣;頑虛之中,徧是空性;鬱𡋯之象,則紆昏塵,澄霽斂氛,又觀清淨。阿難,汝咸看此諸變化相;吾今各還本所因處。云何本因?阿難,此諸變化,明還日輪。何以故?無日不明,明因屬日,是故還日。暗還黑月,通還戶牖,壅還牆宇,緣還分別,頑虛還空,鬱𡋯還塵,清明還霽。則諸世間,一切所有,不出斯類。汝見八種見精明性,當欲誰還?何以故?若還於明,則不明時,無復見暗;雖明暗等種種差別,見無差別。諸可還者,自然非汝;不汝還者,非汝而誰?則知汝心,本妙明淨。汝自迷悶,喪本受淪,於生死中,常被漂溺。是故如來名可憐愍。

阿難言:我雖識此見性無還,云何得知是我眞性?

佛告阿難:吾今問汝,汝今未得無漏清淨,承佛神力,見於初禪,得無障礙;而阿那律,見閻浮提,如觀掌中菴摩羅果。諸菩薩等,見百千界;十方如來,窮盡微塵清淨國土,無所不矚;眾生洞視,不過分寸。

阿難,且吾與汝,觀四天王所住宮殿,中間徧覽,水、陸、空行,雖有昏明種種形像,無非前塵分別留礙。汝應於此,分別自他;今吾將汝擇於見中,誰是我體?誰為物象?阿難,極汝見源,從日月宮,是物非汝;至七金山,周徧諦觀,雖種種光亦物非汝;漸漸更觀,雲騰鳥飛,風動塵起,樹木山川,草芥人畜,咸物非汝。阿難,是諸近遠,諸有物性,雖復差殊。同汝見精,清淨所矚,則諸物類自有差別,見性無殊。此精妙明,誠汝見性。

若見是物,則汝亦可見吾之見?若同見者,名為見吾;吾不見時,何不見吾不見之處?若見不見,自然非彼不見之相;若不見吾不見之地,自然非物,云何非汝?

又則汝今見物之時,汝旣見物,物亦見汝。體性紛雜,則汝與我,並諸世間,不成安立。阿難,若汝見時,是汝非我。見性周徧,非汝而誰?云何自疑汝之眞性,性汝不眞,取我求實?

阿難白佛言:世尊,若此見性,必我非餘。我與如來,觀四天王勝藏寶殿,居日月宮。此見周圓徧娑婆國,退歸精舍,只見伽藍;清心戶堂,但瞻簷廡。世尊,此見如是。其體本來周徧一界,今在室中,唯滿一室?為復此見,縮大為小?為當牆宇,夾令斷絕?我今不知斯義所在,願垂弘慈,為我敷演。

佛告阿難:一切世間,大小內外諸所事業,各屬前塵;不應說言,見有舒縮。譬如方器,中見方空。吾復問汝:此方器中,所見方空,為復定方?為不定方?若定方者,別安圓器,空應不圓?若不定者,在方器中,應無方空?汝言不知斯義所在。義性如是,云何為在?阿難,若復欲令入無方圓;但除器方,空體無方。不應說言,更除虛空方相所在。若如汝問,入室之時,縮見令小。仰觀日時,汝豈挽見齊於日面?若築牆宇,能夾見斷;穿為小竇,寧無續跡?是義不然。一切眾生從無始來,迷己為物;失於本心,為物所轉,故於是中,觀大觀小。若能轉物,則同如來。身心圓明,不動道場。於一毛端,徧能含受,十方國土。

阿難白佛言:世尊,若此見精必我妙性,今此妙性現在我前。見必我眞,我今身心復是何物?而今身心分別有實,彼見無別分辨我身。若實我心,令我今見;見性實我,而身非我?何殊如來先所難言:物能見我。惟垂大慈,開發未悟。

佛告阿難:今汝所言,見在汝前,是義非實。若實汝前,汝實見者;則此見精旣有方所,非無指示。且今與汝坐祇陀林,徧觀林渠,及與殿堂,上至日月,前對恆河。汝今於我師子座前,舉手指陳。是種種相:陰者是林,明者是日,礙者是壁,通者是空,如是乃至草樹纖毫,大小雖殊,但可有形,無不指著。若必其見現在汝前,汝應以手確實指陳,何者是見?阿難當知,若空是見,旣已成見,何者是空?若物是見,旣已是見,何者為物?汝可微細披剝萬象,析出精明淨妙見元,指陳示我。同彼諸物,分明無惑。

阿難言:我今於此重閣講堂,遠洎恆河,上觀日月,舉手所指,縱目所觀;指皆是物,無是見者。世尊,如佛所說,況我有漏初學聲聞;乃至菩薩,亦不能於萬物象前,剖出精見,離一切物別有自性。

佛言:如是,如是。

佛復告阿難:如汝所言,無有見精離一切物,別有自性;則汝所指,是物之中無是見者。今復告汝:汝與如來坐祇陀林。更觀林苑,乃至日月,種種象殊。必無見精,受汝所指;汝又發明,此諸物中何者非見?  

阿難言:我實徧見此祇陀林,不知是中何者非見。何以故?若樹非見,云何見樹?若樹卽見,復云何樹?如是乃至若空非見,云何見空?若空卽見,復云何空?我又思惟:是萬象中,微細發明,無非見者。

佛言:如是,如是。

於是大眾非無學者,聞佛此言,茫然不知是義終始。一時惶悚,失其所守。如來知其魂慮變慴,心生憐愍,安慰阿難及諸大眾:諸善男子,無上法王,是眞實語。如所如說,不誑不妄,非末伽梨四種不死矯亂論議;汝諦思惟,無忝哀慕。

是時文殊師利法王子,愍諸四眾。在大眾中,卽從座起,頂禮佛足,合掌恭敬而白佛言:世尊,此諸大眾,不悟如來發明二種精見,色、空,是、非是義。世尊,若此前緣色空等象,若是見者,應有所指;若非見者,應無所矚。而今不知是義所歸,故有驚怖。非是疇昔善根輕尠。惟願如來大慈發明,此諸物象,與此見精,元是何物?於其中間無是非是。

佛告文殊及諸大眾:十方如來,及大菩薩,於其自住三摩地中。見與見緣,並所想相;如虛空華,本無所有。此見及緣,元是菩提妙淨明體;云何於中,有是非是?文殊,吾今問汝:如汝文殊,更有文殊,是文殊者?為無文殊?  

如是,世尊,我眞文殊。無是文殊,何以故?若有是者,則二文殊。然我今日非無文殊,於中實無是非二相。

佛言:此見妙明,與諸空塵,亦復如是。本是妙明無上菩提淨圓眞心,妄為色、空,及與聞見。如第二月。誰為是月?又誰非月?文殊,但一月眞,中間自無是月非月。是以汝今觀見與塵,種種發明,名為妄想;不能於中,出是非是。由是眞精妙覺明性,故能令汝出指非指。

阿難白佛言:世尊,誠如法王所說,覺緣徧十方界,湛然常住,性非生滅。與先梵志娑毗迦羅,所談冥諦,及投灰等諸外道種,說有眞我徧滿十方,有何差別?世尊亦曾於楞伽山,為大慧等敷演斯義:彼外道等,常說自然;我說因緣,非彼境界。我今觀此:覺性自然,非生非滅,遠離一切虛妄顚倒,似非因緣。與彼自然,云何開示,不入群邪,獲眞實心妙覺明性?

佛告阿難:我今如是開示方便,眞實告汝;汝猶未悟,惑為自然。阿難,若必自然,自須甄明有自然體。汝且觀此妙明見中,以何為自?此見為復以明為自?以暗為自?以空為自?以塞為自?阿難,若明為自,應不見暗;若復以空為自體者,應不見塞;如是乃至諸暗等相,以為自者;則於明時見性斷滅,云何見明?

阿難言:必此妙見,性非自然,我今發明是因緣生。心猶未明;咨詢如來,是義云何合因緣性?

佛言:汝言因緣,吾復問汝:汝今因見,見性現前。此見為復因明有見?因暗有見?因空有見?因塞有見?阿難,若因明有,應不見暗;如因暗有,應不見明;如是乃至因空、因塞,同於明、暗。復次,阿難,此見又復緣明有見?緣暗有見?緣空有見?緣塞有見?阿難,若緣空有,應不見塞;若緣塞有,應不見空;如是乃至緣明、緣暗,同於空、塞。當知如是精覺妙明,非因、非緣,亦非自然。非不自然,無非、不非,無是、非是。離一切相,卽一切法。汝今云何於中措心,以諸世間戲論名相而得分別?如以手掌撮摩虛空,只益自勞,虛空云何隨汝執捉?

阿難白佛言:世尊,必妙覺性,非因、非緣。世尊云何常與比丘宣說:見性具四種緣?所謂因空、因明、因心、因眼,是義云何?

阿難,我說世間諸因緣相,非第一義。阿難,吾復問汝:諸世間人說我能見,云何名見?云何不見?阿難言:世人因於日、月、燈光,見種種相,名之為見;若復無此三種光明,則不能見。阿難,若無明時,名不見者,應不見暗。若必見暗,此但無明,云何無見?阿難,若在暗時,不見明故,名為不見;今在明時,不見暗相,還名不見。如是二相,俱名不見。若復二相,自相陵奪,非汝見性於中暫無。如是則知二俱名見,云何不見?

是故,阿難,汝今當知,見明之時,見非是明;見暗之時,見非是暗;見空之時,見非是空;見塞之時,見非是塞。四義成就。汝復應知,見見之時,見非是見。見猶離見,見不能及;云何復說因緣、自然,及和合相?汝等聲聞,狹劣無識,不能通達清淨實相。吾今誨汝,當善思惟,無得疲怠妙菩提路。

阿難白佛言:世尊,如佛世尊,為我等輩宣說因緣,及與自然;諸和合相,與不和合,心猶未開。而今更聞見見非見,重增迷悶。伏願弘慈,施大慧目,開示我等,覺心明淨。作是語已,悲淚頂禮,承受聖旨。

爾時,世尊憐愍阿難,及諸大眾,將欲敷演大陀羅尼諸三摩提妙修行路。告阿難言:汝雖強記,但益多聞;於奢摩他微密觀照,心猶未了。汝今諦聽,吾當為汝分別開示;亦令將來諸有漏者,獲菩提果。阿難,一切眾生輪迴世間,由二顚倒分別見妄,當處發生,當業輪轉。云何二見?一者、眾生別業妄見;二者、眾生同分妄見。

云何名為別業妄見?

阿難,如世間人,目有赤眚,夜見燈光,別有圓影,五色重疊。於意云何?此夜燈明所現圓光,為是燈色?為當見色?阿難,此若燈色,則非眚人何不同見?而此圓影唯眚之觀?若是見色,見已成色,則彼眚人見圓影者名為何等?復次,阿難,若此圓影,離燈別有,則合傍觀屏、幛、几、筵有圓影出?離見別有,應非眼矚,云何眚人目見圓影?是故當知,色實在燈,見病為影。影、見俱眚,見眚非病;終不應言是燈是見,於是中有非燈非見。如第二月,非體非影。何以故?第二之觀,捏所成故。諸有智者,不應說言此捏根元是形、非形,離見、非見。此亦如是,目眚所成,今欲名誰是燈是見?何況分別非燈非見?

云何名為同分妄見?

阿難,此閻浮提,除大海水,中間平陸有三千洲。正中大洲,東西括量,大國凡有二千三百;其餘小洲在諸海中,其間或有三兩百國,或一、或二,至於三十、四十、五十。阿難,若復此中有一小洲,只有兩國。唯一國人,同感惡緣;則彼小洲當土眾生,睹諸一切不祥境界:或見二日,或見兩月,其中乃至暈、適、珮、玦、彗、孛、飛、流、負、耳、虹、蜺,種種惡相。但此國見,彼國眾生本所不見,亦復不聞。阿難,吾今為汝,以此二事,進、退合明。阿難,如彼眾生別業妄見。矚燈光中所現圓影,雖似前境;終彼見者目眚所成,眚卽見勞,非色所造。然見眚者,終無見咎。例汝今日:以目觀見山河國土及諸眾生,皆是無始見病所成。見與見緣,似現前境;元我覺明,見所緣眚,覺見卽眚。本覺明心,覺緣非眚。覺所覺眚,覺非眚中。此實見見,云何復名覺、聞、知、見?是故汝今見我及汝,並諸世間十類眾生,皆卽見眚,非見眚者。彼見眞精,性非眚者,故不名見。

阿難,如彼眾生同分妄見,例彼妄見別業一人。一病目人,同彼一國;彼見圓影,眚妄所生。此眾同分所見不祥,同見業中,瘴惡所起。俱是無始,見妄所生。例閻浮提三千洲中,兼四大海,娑婆世界,並洎十方,諸有漏國,及諸眾生。同是覺明無漏妙心,見、聞、覺、知,虛妄病緣;和合妄生,和合妄死。若能遠離諸和合緣,及不和合,則復滅除諸生死因。圓滿菩提,不生滅性。清淨本心,本覺常住。

阿難,汝雖先悟本覺妙明,性非因緣,非自然性;而猶未明如是覺元,非和合生,及不和合。阿難,吾今復以前塵問汝:汝今猶以一切世間,妄想和合,諸因緣性而自疑惑證菩提心和合起者。則汝今者,妙淨見精,為與明和?為與暗和?為與通和?為與塞和?若明和者,且汝觀明,當明現前,何處雜見?見相可辨,雜何形像?若非見者,云何見明?若卽見者,云何見見?必見圓滿,何處和明?若明圓滿,不合見和。見必異明,雜則失彼性、明名字;雜失明、性,和明非義。彼暗與通,及諸群塞,亦復如是。

復次,阿難,又汝今者妙淨見精,為與明合?為與暗合?為與通合?為與塞合?若明合者,至於暗時,明相已滅,此見卽不與諸暗合,云何見暗?若見暗時,不與暗合,與明合者,應非見明。旣不見明,云何明合,了明非暗?彼暗與通,及諸群塞,亦復如是。

阿難白佛言:世尊,如我思惟:此妙覺元,與諸緣塵,及心念慮,非和合耶?

佛言:汝今又言覺非和合,吾復問汝:此妙見精非和合者,為非明和?為非暗和?為非通和?為非塞和?若非明和,則見與明,必有邊畔。汝且諦觀:何處是明?何處是見?在見在明,自何為畔?阿難,若明際中必無見者,則不相及,自不知其明相所在,畔云何成?彼暗與通,及諸群塞,亦復如是。

又妙見精非和合者,為非明合?為非暗合?為非通合?為非塞合?若非明合,則見與明,性、相乖角;如耳與明,了不相觸。見且不知明相所在,云何甄明合、非合理?彼暗與通,及諸群塞,亦復如是。

阿難,汝猶未明一切浮塵諸幻化相,當處出生,隨處滅盡;幻妄稱相,其性眞為妙覺明體。如是乃至:五陰、六入,從十二處至十八界。因緣和合,虛妄有生;因緣別離,虛妄名滅。殊不能知生滅去來,本如來藏常住妙明,不動周圓妙眞如性。性眞常中,求於去、來、迷、悟、生、死,了無所得。阿難,云何五陰本如來藏妙眞如性?

阿難,譬如有人以清淨目,觀晴明空,唯一晴虛,迥無所有。其人無故不動目睛,瞪以發勞,則於虛空別見狂華,復有一切狂亂非相。色陰當知,亦復如是。阿難,是諸狂華,非從空來,非從目出。如是,阿難,若空來者,旣從空來,還從空入。若有出入,卽非虛空。空若非空,自不容其華相起滅;如阿難體,不容阿難。若目出者,旣從目出,還從目入。卽此華性,從目出故,當合有見。若有見者,去旣華空,旋合見眼?若無見者,出旣翳空,旋當翳眼。又見華時,目應無翳,云何晴空號清明眼?是故當知色陰虛妄,本非因緣,非自然性。

阿難,譬如有人,手足宴安,百骸調適。忽如忘生,性無違順。其人無故以二手掌,於空相摩,於二手中妄生澀、滑、冷、熱諸相。受陰當知,亦復如是。阿難,是諸幻觸,不從空來,不從掌出。如是,阿難,若空來者,旣能觸掌,何不觸身?不應虛空選擇來觸。若從掌出,應非待合?又掌出故,合則掌知,離則觸入,臂、晼、骨、髓,應亦覺知入時蹤跡。必有覺心知出知入,自有一物身中往來,何待合知,要名為觸?是故當知受陰虛妄,本非因緣,非自然性。

阿難,譬如有人,談說酢梅,口中水出;思蹋懸崖,足心酸澀。想陰當知,亦復如是。阿難,如是酢說,不從梅生,非從口入。如是,阿難,若梅生者,梅合自談,何待人說?若從口入,自合口聞,何須待耳?若獨耳聞,此水何不耳中而出?思蹋懸崖,與說相類。是故當知想陰虛妄,本非因緣,非自然性。

阿難,譬如暴流,波浪相續,前際後際,不相踰越。行陰當知,亦復如是。阿難,如是流性,不因空生;不因水有;亦非水性;非離空、水。如是,阿難,若因空生,則諸十方無盡虛空,成無盡流,世界自然俱受淪溺。若因水有,則此暴流,性應非水;有所有相,今應現在。若卽水性,則澄清時,應非水體。若離空水,空非有外,水外無流。是故當知行陰虛妄,本非因緣,非自然性。

阿難,譬如有人,取頻伽瓶,塞其兩孔;滿中擎空,千里遠行,用餉他國。識陰當知,亦復如是。阿難,如是虛空,非彼方來,非此方入。如是,阿難,若彼方來,則本瓶中旣貯空去,於本瓶地,應少虛空。若此方入,開孔倒瓶,應見空出。是故當知識陰虛妄,本非因緣,非自然性。

%\vspace{1cm}
%\begin{center}
%\setlength{\fboxsep}{1mm} 
%\fbox{%
%\includegraphics[width=.6\textwidth]{pictures/cdzyy.png}
%}
%\end{center}

%\pagebreak

%\pagestyle{empty}
%\begin{center}
%\setlength{\fboxsep}{1mm} 
%\fbox{\includegraphics[width=\textwidth]{pictures/ch.png}}
%\end{center}
